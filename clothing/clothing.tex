\documentclass[oneside, article]{memoir}
\input{../../../work/packages}
\input{../../../work/packagesMemoir}
\usepackage{fontspec, xunicode}
%\setmainfont[Script=Devanagari]{Chandas}
\setmainfont[Script=Devanagari]{Kalimati}

\input{../../../work/packagesMemoir}
\usepackage{fontspec, xunicode}
%\setmainfont[Script=Devanagari]{Chandas}
\setmainfont[Script=Devanagari]{Kalimati}

\input{../../../work/packagesMemoir}
\usepackage{fontspec, xunicode}
%\setmainfont[Script=Devanagari]{Chandas}
\setmainfont[Script=Devanagari]{Kalimati}

\input{../../../work/packagesMemoir}
\title{Clothing strategies}
\author{vishvAs}

\begin{document}
\maketitle

\chapter{Prelude}
\section{Importance of right clothing}
Norwegian proverb: There is no such thing as bad weather, only bad clothing.

\subsection{Importance of right attitude}
The body can function well in extreme conditions, when the right attitude is adapted. It is possible to wear an attitude suitable for the cold, as eskimos do: one can welcome the cold temperature rather than shun it. Similarly, one can welcome dryness and hot weather.

However, one must guard against the risk of inflicting damage to the body by so tolerating extreme conditions.

\section{Goodness of fit}
For full-arm clothes, sleeves should fit well: otherwise, it returns in discomfort in the wrist as a result of repeated strain resulting from putting an over-sized sleeve out of the way while working.


\subsection{Things to check while trying out pants}
Breath should not be forced or restricted while wearing the pants and sitting.

Weave:

\section{Materials}
\subsection{Breathability}
It is the weave of the fabric (the size and number of holes) that determines breathability or resistance to air movement. Any woven or knit fabric will breathe - even if the weave is made of rubber strands.

If the yarn on the outside of the garment is thinner than the yarn on the inside of the garment, capillary action will pull water to the outside. The increased surface area (not hollow fibre cores) of the thin yarn gives the water more space to spread out. This type of construction can be used in either Nylon or Polyester fabrics.

\subsection{Wool}
Wool is a good insulator, even when wet. Expensive, heavier, slow drying, can be itchy. 

\subsubsection{Merino}
Merino wool is very fine and does not irritate the skin.

\subsection{Polyester}
Artificial Fleece, aka polar fleece or microfleece, is a good insulator. Usually made of PET of the polyester family. Suitable for outer and middle layers. Can be made breathable: So suitable for physical exertion involving perspiration.

\subsubsection{Advantages}
It is light, soft, comfortable, hydrophobic (holding less than 1\% of its weight in water when fully soaked), highly breathable. Dries fast. 
    
\subsubsection{Disadvantages}
It tends to generate very high static electricity charges. This makes it a magnet for pet hairs and other dust and fluff. Not windproof (although some more expensive grades are denser and designed to be windproof. Can be damaged by high-temperature washing and drying. But looses insulation when wet. The down-side for polyester is odor retention, and lower durability.

\subsection{Cotton}
Cotton absorbs water, and does not repel water. heavy.

Denim is cotton. Denim is a good wind breaker.

Material which is a suitable blend of cotton and polyester is desirable for use as an insulator.

\subsection{Goose down}
Light, but breaks down with use. Excellent warmth to weight ratio. the Achilles heel of down is that it loses all insulating properties when wet; so good in combination with waterproof material.

\subsection{Other materials}
Silk has good wicking ability.

Lycra/ spandex: like wetsuits, very elastic. When putting a garment with spandex in the dryer, it tends to dry out and the strands break meaning the garment loses compression power.

Nylon: light, tough. Easily melts. Hydrophilic.

Nylon / lycra blend.

\subsection{Rain proofing}
Rainclothes made out of tightly knit nylon are rain and wind proof, and are breathable.

Material which is water proof under 30000 mm of water is labelled water proof. Highly water proof material is also highly windproof.

\section{Things to carry}

A wallet with credit/ debit cards, cash, ID.

Keys

Multifunction Phone: with camera, internet abilities.

A handkerchief or some tissues.

A paper and pen for note taking.

\section{Repairs}
Use good fabric glue : common ones do not stick well. Glue tapes are hard to peel and use.

\chapter{Low temperature clothing}
\section{General strategy}
\subsection{Layer up}
Dressing with multiple layers of clothing essential, due to the wide use of heated buildings. It also insulates better.

Wear a warm clothing a few minutes before stepping out - then, air will be trapped inside the clothes.

\subsection{Accesorize}
use faceguards, gloves, etc.. for extremities. This provides additional comfort. Eg: a hood not attached to the jacket provides improved mobility.

\subsection{Maintenance}
Look for machine washable and dryable clothes.

\section{Base layer}
See warm/ normal weather clothing.

\subsection{Purpose}
Warmth, moisture control while maintaining ease of movement.
Fibers will wick (move) moisture away from your skin and pass it through the fabric so it will evaporate.

\subsubsection{Material choice}
Underwear made of polyester fiber do this, but they irritate the skin.

Merino wool thermals and balaclavas have been tried successfully.

\section{Insulation layer}
\subsection{Materials}
Cold weather (<50F) = a polyester-wool blend rip stop fabric.

Temperate (50-80F) = a 65\%/35\% polyester/cotton blend rip stop fabric.

\subsection{Sweatshirt}

This can be used during dry days. Choose ones with light but warm hoods, for convenience. Zipped sweatshirts' insulation is adjustable, and they are easy to remove.

80/20 and 50/50 Cotton, polyester/ fleece blends have been used before.

\subsection{Pants}
Use polyester, which wicks away moisture while providing warmth Eg: running pants.

\subsection{Long Underwear}
Wear thermal underwear over normal underwear, in order to reduce need for washing.

Avoid thermal underwear made out of artificial material, as this will lead to skin irritation. Fine merino wool thermal underwear was used successfully.

Thickness / Weight of the underwear should be chosen based on operating temperature.

Don't buy leg warmers/ sleeves: tendency to slip. Instead modify a merino wool long underwear by cutting out the genital area: ye will use it on top of normal underwear anyway.

\subsection{Pants}
Can use pants made out of thick materials, such as Jeans/ Denim, Corduroy, Cargo pants.

Corduroy pants made out of a 66/33 cotton/ polyester blend have been successfully used for insulation.

Or can use down pant for excellent warmth. Eg: Made by Cabela's.

28 inch waist, 30 inch inseam relaxed fit Denim jeans were too tight around the crotch, restricted breath while sitting down. 29 inch and 30 inch waist relaxed fit corduroy pants have been successfully used.

\section{Outer layer}
\subsection{Purpose}
Protection from rain and wind. Must be easily packable.

This can be same as rain clothing.

\subsection{Overcoat}
Buy slightly larger size to accommodate multiple layers of clothing.

Choose coats with large pockets.

This could be a rain and wind stopping shell, without insulation.

Choose coats with storm hoods, preferably detachable.

Choose clothes with strings to close sleeves.

belt for heavy coat - the waist is designed to carry weight.

\subsection{Pants}
Choose clothes with strings to close sleeves.

Avoid cotton or denim, as they get wet easily.

\section{Insulation and protection of extremities}
\subsection{Headgear}
An estimated 30\% of your body heat escapes through your head.

Hoods are more effective than caps and hats, as they protect the ears. can take balaclava and alter it, so that you don't have to pull it over.

Protecting face: Cold wind hits you while cycling. Use face guard with hole for the nose: nose drips when you return to a warm place from the cold.

\subsubsection{Material choice}
Wool can be irritating. Polyester and merino wool balaclava has been tried successfully.

\subsection{Scarf}
Better for mobility than attached hood. wind can get in between coat and hood.

\subsection{Gloves}

Leather gloves are breathable on hot days, while protecting from the wind.

All weather light weight gloves provide good insulation too.

\section{Shoes}
Instep crampons are useful for walking on ice.


\section{Sleeping gear}
Down goose: Down bags are considered superior because of their phenomenal warmth-to-weight and warmth-to-bulk ratios. While a synthetic bag will weigh somewhat more than a down bag at an equivalent temperature rating, synthetic bags perform better when wet.

Use sleeping bags with the right temperature rating. it is more difficult to stay warm in an insufficiently insulated bag than it is to vent a bag designed for cooler temperatures.

Prefer water resistant ones. Look for small stuff size and light weight.

Line the inside of the sleeping bag with a sheet, in order to reduce need to wash. Wear long underwear and layers to sleep.

Consider sleeping bag liners to add insulation.

Mateable bags can be zipped together.

\chapter{High/ normal temperature clothing}
\section{Materials}
Cotton is popular. Absorbs sweat etc.. Better is something which wicks away moisture: like nylon weave.

\subsection{Underwear}
\subsection{Purpose}
Prevent outer garments from being soiled by body fluids, body oils and perspiration. Provide support for genitalia.

\subsection{Strategy}
Use good (Jockey) boxer briefs with the elastic band embedded inside the cloth.

\section{Upper body}
Use T shirts next to the body. Use collarless, light colored ones if wearing a formal shirt above.

\section{Pants}
Buy pants which sit on the waist.

\chapter{Extremities}
\section{Shoes}
\subsection{Heavy or formal use}
Use water proof, breathable boots, with minimal heaviness. It must be usable even on formal occasions.

\subsubsection{Lace}
Shoes that don't require lace-tying are generally looser and more uncomfortable during brisk walks.

There are shoes which can be easily (un)tied by just pulling a loop. Use these.

Use easy to tie lace: lock/ elastic/ speed/ quick/ lace.

\subsubsection{Avoid plastic shoes}
Leather boots, rather than plastic boots, are breathable.

\subsubsection{Quality of manufacture}
Shoes bought from good companies, though more expensive, last longer.

\subsection{Comparison with slippers}
See slipper section.

\subsection{Necessity of socks}
See discussion about high-adherance slippers in the slippers section.

\section{Slippers}
\subsection{Comparison with shoes}
\subsubsection{Adherance}
Slippers in general, adhere more lightly to the bottom of the feet than shoes do. Among slippers too, there are different levels of adherence.

walking is more cumbersome if adherence to feet is low. But, slippers which have higher adherance usually require socks for comfort, often because of the sweat and heat generated by close contact.

\subsubsection{Adherence, protection, ease of wearing}
Slippers offer lower protection from the surroundings.

Slippers are usually easier to wear than shoes, which often require tying of lace. Shoes that don't require lace-tying are generally looser and more uncomfortable during brisk walks.

\subsubsection{Coolness}
Much cooler in summer.

\subsection{Absorbant soles}
Prefer slippers with absorbant, possibly antimicrobial soles - they are more comfortable and less smelly in summer.

\subsection{High adherance slippers}
\subsubsection{Beach socks}
Like shoes, with nylon mesh top. Require socks. Available in walmart.

\section{Socks}
Use thick jockey socks: less smelly.

Or consider materials which wick away moisture.

\section{Eye}
\subsection{Purpose}
Avoid the discomfort of having to squint at the sun. Or remove glare from either the computer screen or from reflected light. Or improve visibility at night.

\subsection{Choices}
Buy some light safety glasses which fit over prescription eye-glasses, with either smoke or yellow tinted lenses. Sunglasses should ideally cover the entire visual range where the sun strikes the eyes: even the top. For this reason, clip-on sunglasses are inadequate. 

Or Use Prescription sun glasses or sun glasses which fit over glasses. 

\section{Sun screen}
Buy sunscreen with high SPF, but avoid ones with oxytocin (a synthetic estrogen which easily penetrates the skin).

\chapter{Special occasion clothing}
\section{Rain clothing}
Breathable rain clothing, with sealed seams are good.

Full zip rain pants are easy to wear and remove.

\section{Swimwear}
Wetsuits, fast drying underwear.

\section{Hindu ceremonial dress}
KachchE panche with uttarIya or with a jubba: wearing technique explained in customs survey.


\chapter{Carrying things}
\section{Belt}
Useful not only for holding up pants tightly to the waist, ensuring good weight distribution, but also enables attachments like pouches.

Buy a synthetic material belt, ideally one which can be used in professional circumstances, with very little metal (no big belt buckle), so that it need not be removed for airport security checks.

\section{Cargo pants/ shorts}
Can be found in military stores or online. Professional look is achievable.

If heavy items are carried in lower pockets, belt is required.

Carrying items in back pant pockets can slightly affect sitting posture.

\section{Vests}
\subsection{Types and purchase spots}
Tactical vests are not suitable. Ranger vests or (photo) journalists/ travel/ carry vests have many large pockets inside and outside (both in the front and in the back). Ranger vests tend to come with full zipper. Journalist vests tend to have larger and more pockets. Those made with mesh are the lightest.

If a suitable material (cotton or polyester + cotton or mesh) is used, the vest is light and breathable, therefore is fit for use in summer.

\subsection{Neatness}
Black colored ones tend to complement light colored shirts well, and are suitable for professional use.

Goodness of fit is variable. In good vests a draw chord is provided.

Can be used to hide crumpled shirt and loose fitting pants.

\subsection{Use in Carrying/ storing}
They distribute weight over the shoulders very well; thus they are very comfortable.

One can store items in the vest while varying shirts and pants.

It is also convenient to place all carried contents together for inspection at airports.

\section{Non-clothes Attachments}
Bags.

\section{Pouches}
Often equipped with special detachable loops, so that the pouch can easily be attached to a belt. These are often marked Molle compatible pouches in Military stores, where they are often available for cheap.

\subsection{Waist pouches}
Maxpedition thermite versipack is conveniently part of a belt.

\subsection{Leg drop pouches}
The lower strap, if tied can cause chafing; so it is a good idea to leave it untied.

\end{document}
