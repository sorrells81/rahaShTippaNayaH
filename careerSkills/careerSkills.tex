\documentclass[oneside, article]{memoir}
\input{../../../work/packages}
\input{../../../work/packagesMemoir}
\usepackage{fontspec, xunicode}
%\setmainfont[Script=Devanagari]{Chandas}
\setmainfont[Script=Devanagari]{Kalimati}

\input{../../../work/packagesMemoir}
\usepackage{fontspec, xunicode}
%\setmainfont[Script=Devanagari]{Chandas}
\setmainfont[Script=Devanagari]{Kalimati}

\input{../../../work/packagesMemoir}
\usepackage{fontspec, xunicode}
%\setmainfont[Script=Devanagari]{Chandas}
\setmainfont[Script=Devanagari]{Kalimati}

\input{../../../work/packagesMemoir}
\input{../../../work/macros}

\title{Career skills strategy}
\author{vishvAs}

\begin{document}
\maketitle

\part{parichayaH}
dainikEShu kAryEShu laghu-sAdhanEShu santOShaH kathaM prApyaH ityEtat anyatra vivRRitaH.

\chapter{Ability and achievement}
\section{Achievement}
\subsection{10000 hour rule}
World-class performance in many cognitively complex fields requires around 10000 hours of focused practice.

\subsection{Role of opportunities}
Canadian national hockey players were overwhelmingly likely to be born earlier in the year - being slightly bigger/ mature, they recieved greater opportunities.

\section{Cognitive abilities}
Matters pertaining to brain health, disease, habits which affect cognition and maintenance are discussed elsewhere.

\subsection{Limited role of genes in cognitive ability}
Except in the case of specific genetic diseases, genes seem to play less of a role in the heritability of "intelligence", than does epigenetics (factors which affect the expression of genes). Skills and habits can be at once highly heritable (because of strong transmission through family and neighbors) and still be non-genetic. Genes which influence (say) decision-making under uncertainty have not been under as much selection pressure as those for color vision or those for immune systems which easily develop resistance to smallpox.

'When Eric­sson began working with a young man identified as S.F., his subject could, like most of us, hold only seven numbers in his short-term memory. By the end of the study, S.F. could correctly recall an astonishing 80-plus digits.' \href{http://www.nytimes.com/2010/03/21/books/review/Paul-t.html?ref=books}.

\part{General skills}
\chapter{Oral presentations}
Write your own introductions.

\section{Importance}
They can add to reputation and opportunities beyond what may be earned through research. It is also very important during hiring considerations. It is essential in teaching.

\section{Objective setting}
Have a clear objective. Know your audience.

Distinguish between tutorial presentations, job-talks and conference presentations. Talks at conferences are meant to encourage reading the paper, as it may not be possible to present the entire content of the paper in the given time.

Present to inform and provoke thought. Then you will impress.

\section{Content}
\subsection{Importance of preparation}
Always prepare the content, in full detail, beforehand. This is especially important for mathematical presentations.

The preparation is best when you can present naturally, straight from the short-term memory, without looking at notes. Eg: pradIp, pratIk.

Then you can anticipate questions. Then you will catch your own mistakes.

If it is unclear, email authors of the paper for such information.

\subsection{Expectation-setting}
Always set up the audience's expectations.

Have a clear outline. At any point in the talk, the audience should know what part of the talk you are covering.

\subsection{Varying levels of abstraction}
Present the intuition before the details.

Use visual content. A good standard is to have a picture in each slide.

Use analogies and examples.

In presenting mathematics, hide parameters (such as delta) unnecessary to the main result by replacing them with constant values.

Stick to the point. Do not cram in too much data and bullet slides.

Short phrases better than full sentences.

\subsection{Stimulating audience interest and prior knowledge}
Motivate the importance of the topic, with examples, problems etc..

Use live demos in lectures, if possible.

You have to insert a break in your presentation every 10 minutes to refresh the audience.

\section{Composing yourself}
\subsection{Calming turbulence}
Remember the aim: 'Relax, look at them, tell them what you know. They may not know it as well as you do.'

Defensiveness about ideas and nervousness come together. Separate yourself from the ideas you are presenting. You are presenting some ideas, not your ideas. Keep the mind supple.

Be confident and somewhat animated: very important. Don't be nervous.

If there is more nervousness, you tend to go over material faster.

\subsection{Practice}
More practice, less variance in time taken for delivery.

More practice, less nervousness.

Carry water or a refreshing drink to keep energy and enthusiasm from sagging.

Videotape yourself.

\section{The delivery}
\subsection{Guided Dialogue}
The talk should be a dialogue with the audience. The audience's message may not be in words, but it is often communicated using body language. You should look for audience reactions, look for signs of doubt or incomprehension, and adjust your talk appropriately - maybe you should ask or invite questions at certain points.

It is good to use visual aids like slides and movies, and auditory aids; but never use it as a crutch! You should, for the most part, deliver your message while looking at the audience.

\subsubsection{Asking questions}
If necessary, ask directed questions.

When you ask questions, wait for the answer. (This could take 12 to 15 seconds.)

In case of tutorials, this is especially important: Ask many leading questions; Let them discover the proof.

\subsubsection{Math case}
Mathematicians talk shop on the board; so there is heavier use of visual aids.

The ability to read out mathematical expressions in natural language and explain the intuition behind it is important.

Often it is mostly a matter of transferring ideas from notes to the board, and explaining it.

\subsection{Attention management}
Keep motivating current part of the lecture adequately.

Induce seriousness and fun in the class. Eg: RK.

\subsubsection{Emotion}
Excitement and emotion should come accross in presentations. Great teachers are highly animated. Like sanjay and lexing.

\subsection{Using presentation aids}
Pointing: Try not to touch the screen and shake it.

Animation: Make lines appear one at a time, to keep the audience's attention.

Use live demos where possible: they are always impressive.

\subsection{Beginning presentations}
Begin talk by thanking the audience for coming to the talk, and saying 'It is my pleasure to talk about XYZ today'.

\subsection{Ending presentations}
Say conclusion, acknowledgements. Thank the audience for their attention, and invite questions.

\subsubsection{Conclusion}
Have a take home message.

\subsubsection{Acknowledgements}
Praise and acknowledge roles of others. Sasha acknowledged knowledge inferred from courses of Klivans, Zuckerman and Gal (Communication complexity).

\subsection{Body language}
Use body language appropriate for a guided, honest dialogue. Use smiles when appropriate.

Make eye contact. This stimulates the amygdala and enable attachment of emotions to knowledge.

Don't put your hand in the pocket, don't cross it over the chest: they are signs of fear and anxiety.

\subsubsection{Gestures}
Use gestures, move around. Gestures can help you make points, distinguish concepts. That can aid you in captivating the audience. Great teachers are highly animated. Like sanjay and lexing.

\subsection{Voice}
Don't shout and hurt your voice and mood. Use appropriate aids or ask audience to move closer.

Venue: Voice carries well in an auditorium: no need to shout.

Do not mumble. Don't end sentences weakly.

Pauses and expression is important in delivery. Include presentation cues in your slides (eg: pausing, pitch variation, stressing etc..).

Take extra care with the beginning and the ending. Decrease pitch at the end of some sentences.

\subsection{Feedback, questions}
Most of the audience's feedback is expressed in body-language; but you should invite a good amount of verbal feedback. Especially when presenting ideas to close collaborators, questions and ideas flow more freely.

The dialogue should be a honest one: you should be able to think for a short while, and be open to the prospect of learning from the audience. You can think for a larger duration while talking to close collaborators. Don't bluff when answering questions, be able to admit that you don't know the answer.

Don't let particular members of the audience make you nervous. Shouldn't need to look at pieces of paper to answer questions. You should postpone tough questions off-line if necessary.

Even criticism is important. If they expose holes in your understanding and skills, that is especially good.

Keep a pad or a recorder handy to record important comments and questions.

\chapter{Job interviews}
This is considered in the 'economics' survey.

\chapter{Software engineering}
Software engineering and programming skills are considered in the 'Software architecture' survey.

\part{Research process}
\chapter{The career objective}
The justification for my participation in this career are described in the career strategy. Appropriate career paths are considered in the career strategy.

\chapter{Constant Training}
See Training strategy. Deliberate practice should be used, and training is continuous.

\section{Preparation: Importance}
The first great discovery may take a long time to come. The task is to relentlessly prepare oneself for that great discovery. It is important to keep working at the boundary of one’s abilities.

\chapter{Research}
\section{Central pleasure of solving problems}
The pleasure is great when you are tackling a problem,
\begin{itemize}
\item which you are certain is worth tackling - guaranteeing enthusiastic effort; 
\item requires you to invest an amount of thought and effort which is neither too small nor too huge in order to start reaping rewards; like activation energy in chemistry
\item you feel confident about making significant progress towards a solution - or even about solving it.
\end{itemize}

All other pleasurable activities associated with a research career - such as mentoring, presentation, gathering knowledge etc - depend on this for meaningful existence; so this is the primary pleasure. However, one must take pleasure in these surrounding activities; as the anticipation of the pleasure of solving problems may not provide sufficient enthusiasm.

\section{The process}
\subsection{High level process for each problem}
General loop: Find area. Find problem in a given area. Find solution strategy for a given problem. Implement solution, evaluate, communicate.

"Read many papers; write many papers; repeat."

Popular research mode: Knowing and heavily using old techniques: take a few hammers, hit everything with it. Finding new techniques is rare and hard.

\subsection{jJNAna-prApti: Learning an area}
jJNAna-prApti-sUtraM pashyatu.


\subsection{Problems: Finding candidates}
\subsubsection{Papers}
In doing incremental-improvement research, it is important to be able to critique papers, find their shortcomings, identify unsolved problems, then to be able to solve them.

\subsubsection{Talking to others}
Another person may be facing a problem in research where you, having a different perspective, may have a good solution strategy. Research shows that this mode of finding and solving problems is results in very good quality research.

\subsection{Choosing problems to pursue}
\subsubsection{Judging impact: theory}
The most applicable of all is a good theory.

\subsubsection{Importance of selectivity}
You don't have enough time and energy to pursue all interests. So, must be selective.

\subsubsection{Fresh problems: Importance}
Find your own, fresh problems and work on them. If a senior person is not actively involved, it is much harder to beat others who have worked in the field for years.

\subsubsection{High vs Low impact work: balance}
Aim for rough water. Devote adequate time to important research, like Leslie Valiant.

Short term/ competitive research should be avoided. As Dijkstra [EWD637] said, do not compete with your colleagues who are as well equipped as you are to tackle a certain problem.

Andy Yao said this: When you get a paper accepted to a good conference, the main effect is that it buys you some time: some number of weeks or months during which the pressure to publish is off and you can focus on exciting but very hard open problems. Thus, your work should be organized to achieve a balance between projects that will probably improve your vita, and projects that will most likely not get you anywhere in the short term, but that are important in the long term.

\subsection{Solving problems}
See problem solving strategy.

\subsection{Parallelism}
Execute many research projects/ threads in parallel. Switch between problems. Know when a project isn't right for the present moment.

\subsubsection{Time multiplexing}
Must have a good way of dividing time amongst various activities.

This worked well in Fall 2009: Experiments from 4 or 5, reading and classes before that.

\section{Collaboration and Team work}
Even advisor advisee relationship is partly one of collaboration.

\subsection{Benefits}
\subsubsection{Gives Encouragement}
Doing research alone is possible; but it requires huge amount of focus and determination; especially when it comes to doing less pleasant tasks like experimentations and simulations.

From collaborations, even boring aspects of research become more pleasant. Enthusiasm for research increases. It is also more pleasurable. See benefits of pair programming in 'programming strategy'.

\subsubsection{Toughness of problems solved}
With different viewpoints, the probability of the problem being solved is higher. Also more interesting problems often discovered.

It is also a source of new problems to attack.

Papers involving collaborations tend to have much higher impact and are more cited.

Between-institution collaborations had a higher average impact.

\subsection{Identifying good collaborations}
If a collaboration yields much more pain than pleasure, more discouragement than encouragement, it is probably a bad collaboration. In a bad collaboration, stress is high and productivity is low.

\subsection{Beginning collaborations}
Present many papers. You will hear/ give ideas unrelated to what you/ others were thinking. Happened in the case of the use of affiliation networks, microscopic node analysis in solving link prediction problems; and in case of use of collaborative filtering in link prediction.

By presenting papers and generating ideas in others, reap goodwill and increase reputation.

\subsection{Executing a good collaboration}
\subsubsection{Contributing}
Bring in enthusiasm, ideas and work. Show ownership, take initiative.

Be prompt in accomplishing what you promised: without it, collaborators loose patience and interest, research projects then fizzle out. So, you cannot work on too many things simultaneously.

Have clear communication, supply updates upon major milestones, inform others about tough roadblocks which you could not tackle autonomously.

\subsubsection{Feedback and setting the record}
Use feedback well, but No groveling - correct unfounded statements and suppositions about your work without fail.

\subsubsection{Getting updates}
Ask if the person who is most actively involved has new thoughts/ tried new avenues about his research project.

Fix concrete date and time to discuss project, papers.

\subsection{Technical discussions}
People form opinions about your abilities partly through your ability to communicate and collaborate well. This includes ability to propose ideas, solve problems on the spot, having a supple mind.

It often involves making oral presentations of technical work - this is described elsewhere.

Note down important things to follow up.

\subsection{Industry}
The initial meeting in industry:

"You do what I call "Shotgun Research" at some labs. You go into a meeting with a person you haven't met, and in the following one hour, you have to bond with them, size them up, get their respect, quickly formulate a problem that will match the interests of both of you, and put in place the skeleton of a novel idea or direction or even a claim. Then before the clock interferes you have to lay the foundation for how the direction will be developed, and leave with a social contract. Later of course you will refine the problem, nudge the direction and rewrite the claims, and have to do the sweat work to make it a project or a paper, but the initial shotgun meeting is the key." [Muthu]

\section{Multitasking}
You end up balancing multiple research projects and teaching and administrative tasks. You do need to find a way of making progress on various projects simultaneously. The process is probably similar to the way that we figure out how to balance career and family/ community life.

\subsection{Balancing research activities}
There should be effort and focus spent on both acquiring knowledge and on solving difficult problems. One must do both daily.

In Fall 2009, I split time thus: Before 1600 or 1700 each day I focused on theoretical thought and knowledge acquisition. After that, until 2000, I did experiments and focused on a single research problem. This split worked very well.

\section{About marketing yourself}
Note that selling your qualities is very different from selling your work.

Let your work do nearly all the talking.

Discuss ideas and papers enthusiastically with others. Like prateeka, lexing etc..

Do not hanker after credit, but rely on others' honest to credit you for your ideas. The most important thing is to find answers, not gain credit.

\section{Academic visits}
During academic visits, read their papers, and ask questions about it.

\chapter{Communicating research}
\section{Importance}
\subsection{Sell work for timely impact}
For your work to have a timely impact, others who might use your research should know about it. It is absurd to expect others to do it for you.

\section{Publishing venues}
Workshops: Some explicitly deny that a presentation in a workshop does not count as a publication. So such stuff is publishable elsewhere. Other workshops are contrary.

Publish long versions of conference papers in journals.

1 top tier paper = many 2nd tier papers. It is far more important to publish regularly in a top tier conference than to publish profusely in a 2nd tier conference. Read papers from those conference and aim for such work.

\subsection{Publicising by emailing}
If you discover a new exciting result in a hot area, it is a good idea to inform others who have tackled similar problems about your result. Eg: PratIk, raghu, inderjit did this with their compressed sensing/ matrix completion paper. Thence increase impact, get new suggestions.

\section{Writing}
\subsection{Objectives}
To publicise an important discovery, and submit it for criticism and improvement.

\subsubsection{Let it be Explanatory}
\paragraph*{Telling a story}
Write as if you are writing a story. The abstract and the conclusion should contain mostly the same text, except one may say 'we will show', and the other should say 'we have shown'.

\paragraph*{Don't try to sound impressive}
Academics have the bad habit of using jargon and stilted prose. Explanatory writing sounds less impressive, and is longer. But it does its job better. Otherwise, fewer people will be influenced by the idea, more people will throw it away.

\subsubsection{Help the reader!}
Keep writing lively, avoid turgidity.

\subsubsection{Level of effort}
Don't polish papers endlessly, just go for the deadlines.


\subsubsection{Citing others' work}
Ensure that you cite everyone in the program committee relevant to your submission.

If you are writing a grant proposal, ensure that you include a huge number of citations, to indicate to the reviewer that you have sufficient knowledge of the area.


\subsection{Planning}
\subsubsection{Itemized plan for each section}
 Make a plan document for every huge writing project. Write up section names, fill section with itemized list of things which will go in the section.
 
 Useful in collaboration: Can then discuss this outline with others, get their feedback.
 
 Can rearrange content at a high level.



\subsection{Implementation}
\subsubsection{Continuity: outline and summarize}
In writing paragraphs or articles: state the topic, write the contents, conclude by reminding the reader about what you've stated. Always start with an outline.

Sentence = Part which refers to previous sentence(s) + part introducing new idea. Paper, section, paragraph: introduction + point + discussion + emphasis of point.

Repeatedly summarize.

\subsubsection{Clarity}
Think: "In what possible way can this be misunderstood?"

Avoid using pronouns like this, that and it when there is a gap of a few sentences since prior reference to the subject: instead, be specific.

\paragraph*{Terminology}
Use consistent terminology - It will help the reader. Stop hunting for synonyms in the thesaurus.

Give names: use 'red black technique' rather than our technique.


\subsubsection{Crispness}
Let all sentences have a person-like subject.

Avoid:
\subitem Weak verbs: "is", "was".
\subitem Avoid adverbs. (Example: extremely important or very redundant.)
\subitem Avoid double negatives: "Otherwise, you sound not unlike a windbag."

\subsubsection{Formatting}
Try to ensure that figures and tables fill the width of the space allotted to them - let them not give the impression of being inserted just to make the work seem substantial.

Always present numbers in tables right-justified.

\subsubsection{Figures and tables}
In many venues for presenting experimental research (eg KDD), figures and tables can be added to make the work seem more substantial than they actually are.

Presenting research: draw graphs which swoop up: ROC curve.

\subsubsection{Dealing with space restrictions}
Avoid too much repetition.

Make everything you say substantial. Avoid 2 sentence paragraphs.

Use latex tricks like vspace to reduce space towards the end of the document.

\subsection{Review}
Do a self-review - read it aloud and read it cold.

A week before the deadline, send it out to friends and colleagues, correct bugs.

\section{Oral presentation}
Described in a separate chapter.

\part{Social responsibilities in research}
\chapter{Nurturing research talent}
See collaboration strategy.

\chapter{Networking}
This is not about research collaborations, although those should form the skeleton of a strong network. For that see elsewhere. But, it can still be done by discussing problems and current developments without worrying too much about impressing them.

\section{Importance}
Contacts, friendships, and perceptions of your professional worth will help determine your success, and that of your students.

Start the job search on day one.

Most jobs are not even advertised. 4/5 of all jobs go to personal contacts.

\section{Reputation in research community}
Do internships - It is a nice way to get variety in research. Also, internships may be important if one has to go to the industry later.

Make professional contacts by writing to people. Don't be timid.

\subsection{Conferences}
The more conferences you go to, the more people you will know and the more comfortable you will get. By the 2nd or 3rd time, you will be greeted as an old friend. Help organize conferences. (You will then be treated as a peer.)

Give lots of talks, chat with lots of people, make connections, find out where the jobs are, find out what people are working on, find out what people will be working on.

\subsection{Committees}
Ask your advisor's opinion about which committees students can serve on, and how they get appointed. Administrative responsibilities in your resume marks you as the young "assistant professor" type.

Important to be a reviewer or an area chair of a conference etc..

\section{With institutional colleagues}
Make friends in other subfields: This is better than attending talks for learning about happenings in other fields.

Be kind to support staff.

\subsection{Faculty and grad students}
Avoid eating lunch alone.

Be in the department as much as possible, interact and learn from your peers.

"Prepare something mathematical to say. Befriend a faculty member each quarter."

Spend time whenever possible, even during departmental events (including post-seminar sherry hours), with professors, including the advisor and thesis committee members. Like most people, they rarely put themselves out for strangers.

Ensure that you acquire many letter of recommendations from various professors by discussing research problems and other things with them often.

At an appropriate time, discuss career aspirations with your advisor - well in advance.




\end{document}
