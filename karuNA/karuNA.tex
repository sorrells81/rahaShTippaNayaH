\documentclass[oneside, article]{memoir}
\input{../../../work/packages}
\input{../../../work/packagesMemoir}
\usepackage{fontspec, xunicode}
%\setmainfont[Script=Devanagari]{Chandas}
\setmainfont[Script=Devanagari]{Kalimati}

\input{../../../work/packagesMemoir}
\usepackage{fontspec, xunicode}
%\setmainfont[Script=Devanagari]{Chandas}
\setmainfont[Script=Devanagari]{Kalimati}

\input{../../../work/packagesMemoir}
\usepackage{fontspec, xunicode}
%\setmainfont[Script=Devanagari]{Chandas}
\setmainfont[Script=Devanagari]{Kalimati}

\input{../../../work/packagesMemoir}
\input{../../../work/macros}

\title{karuNA}
\author{vishvAsaH}

\begin{document}
\maketitle

\part{parichayaH}
\chapter{Understanding other animals}
\section{Collection of guNas}
chittE santi guNAH. guNaH dRRishyante prasa\~NgeShu. prANinaM jJNAtuM tAn prasa\~NgAn anveShaNi.

It is your relationship with various entities which, defining you, are important, more than the entities themselves.

\section{Guessing thought processes by simulation}
In order to understand the behavior of other animals, it is essential to have empathy, to simulate their thoughts and feelings, to understand their relationships with other entities. One must use the questioning mind to discern their tastes, values and goals.

It is not possible to understand other animals by modeling them using one's own world-view, goals and thought processes. If necessary, one should question these animals and acquire further information.

\section{Correlation between traits}
Correlation between street-smartness and the ability to lie without guilt. Observed in Hector, pitA.

\section{Understanding teenagers}
\subsection{Neurological state}
The neurological developmental state of teenage brains is considered elsewhere.

Thence we learn that this is their last chance to use the remarkable adaptability of their brains to rapidly acquire new information and experiences. So, instinctively, they seek new sensations and experiences.

Thence we also know that their decision making and memory are not as tightly integrated as in adults.

\subsection{Higher level disabilities}
This affects their ability to marshal all their resources in making determined efforts.

\subsection{Decision making model}
\subsubsection{Risk bias}
Experiments indicate that they understand - even overestimate - risks, but in their decision making equation, potential loss is given far less weight compared to potential rewards, very much unlike adults. Plus, social rewards (especially peer approval) is weighted heavily.

Reasons for this strange drive include the fact that greater risk taking is correlated with greater sensation seeking and greater exposure, for which teens are wired by evolution.

\subsubsection{Age crime curve}
In every society at all historical times, the tendency to commit crimes and other risk-taking behavior rapidly increases in early adolescence, peaks in late adolescence and early adulthood, rapidly decreases throughout the 20s and 30s, and levels off in middle age. \href{http://www.psychologytoday.com/articles/pto-20070622-000002.xml}{[Ref]}

\subsection{Influence of peers}
They undertake great effort to gain respect and acceptance from their peers. They are driven to seek out new experience and expand their social circle. 

\subsubsection{(In)sincerity}
This results in insincerity. This is called being 'cool'. For example, they may be behave far more rudely and crassly in the presence of their peers than they would otherwise. 

\subsubsection{Nerds}
Nerds are intellectually disposed misfits, who tend to be more sincere than the rest.

\subsubsection{Rebellion and shame}
They are often rebellious towards their parents' generation: part of the reason is because they prefer the company of peers. Perhaps in reaction to the pressure placed upon them to perform, behave and train themselves or due to perceived hopelessness about the future.

They may be ashamed about various things about their household Eg: perceived hypocrisy.

\subsection{Dealing with them}
Considered elsewhere.

\subsection{References}
\href{http://ngm.nationalgeographic.com/2011/10/teenage-brains/dobbs-text/1}{NG}.

\section{Outlier conditions}
\subsection{Mania and depression}
\tbc

\subsection{Violent sadism without delusion}
Aka psychopathy. Sufferers appeared normal on the surface, but they lacked impulse controls and were prone to outbursts of violence -even sadism.

\subsubsection{Neurological root}
Their amygdala does not react to anticipated unpleasantness normally. Popular opinion is  that they are incurable: legal threats have no effect.

\subsubsection{Diagnosis}
Responses to electric shocks, before they were outlawed. A questionnaire (with a high false positive rate).

\chapter{Others' behaviour to you}
Others behavious to you depend on the guNas of their chitta, which is the result of their samskAras.

\section{Others' models of you}
Their understanding of you and your place in their world determines their behaviour. Ethnicity, for example, is an important feature they consider in modelling you.

hedaruvavarannu kaNDare hedarisuttAre.

\part{indrajita-guru-kula}
\chapter{TA\~Ng vey}
For long seemed satisfied with mediocrity in research, but recently seems to be improving. Has published recently, due to collaboration with jhe\~Ngdong.

Has had some conflict, tension with dhillon-AchArya.

\section{Defensiveness and condescension}
He is hot-headed, very defensive. He looses temper quickly in an argument. Learnt the lesson of delivering 'put-downs' from indrajita-AchArya. I received some of this when I was taking the Data mining course he was TA-ing in fall 2009.

\chapter{pratIka}
Last physical interaction early Dec 2010. Sent me an offline message on google talk in late February 2010, to which I had responded with a detailed, friendly status message.

\section{Attitude towards me}
Wanted a professional, but a very limited personal relationship with me.

Extent of personal relationship: Has given smart advice in the past. But, I have not penetrated his circle of friends or regard.

\section{Very high Research proclivity}
Highly immersed in research. Has high bias towards theoretical work.

\section{Sensitivity to territory}
affiliation networks vishaye mama \\
jhe\~Ngdong-saMvAdAt pratIkaH kruddhaH.

\chapter{lu-jhe\~Ngdong}
Honest.

\section{sahAyya-dAtA}
shlAghya-patra-dAnena sahAyyaM dAsyAmi \\iti avadat mama prashnasya uttare \\indrajita-saMshOdhanAlayaM tyaktvA anyatra-gamana-samaye.

upAyAn adadat indrajitena saha \\kliShTa-kAryasya viShaye cha saMshOdhana-viShaye.

\section{Adept at making high-level research plans, and at collaborating}
Adept at using discussions for collaboration, thereby gathering new ideas. Adept at processing/ discussing vague ideas, thence discerning promising ones.

\section{Sensitivity to territory}
Gently claims credit for his ideas, for example by sending email to Prof. dhillon; suggesting collaboration. He has his fair ideas about 'whose project it is', and enforces it, for example, by dropping or adding people to cc section of the email. Can be upset about meeting dhillon-AchArya alone regarding a fresh idea being explored jointly, without his presense.

\chapter{nAgarAjan naTarAjan}
\section{Capacity for focused work}
Is driven to do well in data-mining, if not to the extant to produce revolutionary results.

Is very good at programming.

\section{Social adeptness}
Good at pleasing people during interactions. By greeting warmly, appearing innocent, not venturing into debates, he is liked.

His tamiL friends constitute his primary social circle.

\section{Slight Irresponsibility}
In Fall 2009, during data-mining project, had agreed over phone to try running extracting a package and running a few experiments. But, he did not do this. This led to trouble the next day; and he called me for help a few times when I had a distributed computing exam.

\subsection{Not picking up calls}
In Fall 2009, a couple of times, he did not pick up my calls. One such incident happened in Spring 2010.

\subsection{Thought process behind this}
Seems to instinctively avoid confrontation; so agrees to nearly everything.

\section{Limited pride in tradition}
His pride in tradition is limited to wearing the yajJNOpavItaM, praising his lineage, wearing vibhUti. He does not yet feel the need to maintain traditions such as the sandhyA-vandanaM, and does not seem to have yet formed clear ideas about being a brAhmaNa by deed rather than by jAti. parantu yajurupAkarmaM Acharat 2010 varShe.

\chapter{dhillon-AchArya}
\section{guNAH}
Very clever, suspicious. A proud person.

ahaMkAra is based on research accomplishments, respect accorded by peers, diffidence accorded by shishyAs.

\subsection{Suceptibility to flattery}
He is suceptible to respect, acknowledgements of superiority and praise. He does not seek love.

He was very pleased at compliments he got from the OR student for his lecture, incidentally same as in the 'Interrogation in class' incident.

\subsection{pratikAra-pravRRitti}
If he perceives that someone, in his biased view, took advantage of him, he reacts vengefully.

\subsubsection{History}
When shun-chuan shifted to bajAj, he was removed from the list of authors from a paper whose original form included his name and which was submitted to a conference from which it was rejected.

\subsubsection{Humuliation at the presentation}
He deliberately asked me a tough question after course presentation and signalled his strong disapproval using body language, possibly as an excuse to deduct marks later or in an attempt to humiliate me in class. On the same day, he spoke to nAgarAjan and strongly suggested to him that he should do the course project with someone else - especially someone still looking for a project partner. Suggesting that nAga explore some topic, he said: 'I asked vishvAs to explore it too, but I don't know what he explored.'

\subsubsection{Interrogation in class}
Affected by some allergy, I exited the class to clear my nose during lecture. I returned. I was sitting in the back row, and was not writing notes. Another person exited and returned after some time.

Possibly irritated by this, he asked a question of me which required use of a notion introduced a page ago. I admitted that I don't remember. Introducing the notion of SVM's which allowed misclassifications, he challenged me to describe how one would modify the optimization problem to allow misclassified points. I responded correctly. He further grilled me to get a mathematical expression; and again I responded properly.


\section{Research strategy}
\subsection{Computational mathematics research skills}
Highly adept at mathematics, especially at applying to understanding and developing pattern recognition algorithms.

He heavily favors mathematically principled approaches to pattern recognition research.

\subsection{Current research strategy: director, not solver}
He sees his role as one of finding promising problems, writing grants, creating the right environment for his students to make progress on these problems/ writing up results, setting up collaborations. He reads up on pertinent literature and discusses progress and ideas with students.

While leaving it up to his students to tackling problems, writing it up, he himself does not think deeply about solutions to problems beyond time spent assessing and discussing progress in various projects. He himself does not run experiments.

After the problem is solved, he critiques the solution and presentation, as it interests him.

He thinks that pressurizing students, maintaining his authority is important in maintaining a productive environment in the lab.

\section{Peer relationships}
\subsection{Adept at professional dealings}
Adept at scheming in dealing with people: students and colleagues. Very adept at divining what others are thinking.

\subsection{Power within the department}
Very well connected. Widely respected for accomplishments. Instrumental in hiring pradIpa ravikumAra; tried hard to hire a good linear algebraist, who is now a stanford prof.

He brought pradIpa ravikumAra to the lab, and plainly pressurised him on matters of syllabus for the statistics course; where, trying to counteract syllabus suggestions introduced by the theory professors, he tried to suggest strongly that pradIpa include calculus 2 as a prerequisite, and remove the 'counting' content. The interaction was filled with condescension.

\subsection{visheSha-saMbandhAH}
duShTa-manasaH chandrajita-AchAryasya mitraM.

jayadIpena saha aneka-vAraM saMshOdhanaM \\chAlayat. parantu, purA jayadIpena indrajitasya mudraNeShu \\svanAma-prAthamikatA ninditaH. tathA, ekadA clustering-saMshOdhane \\
co-PI jayadIpaH mayA saha milituM aichChayat, parantu indrajitena nirAkRRitaH.

\section{Student strategy}
\subsection{Research direction}
indrajita alternates between the roles of aloofness and pressurizing overseer.

\subsubsection{bhAra-upayoga-pravRRittiH}
Good at intimidation and applying pressure on students, perhaps even enjoys it.

Eg: indrajitasya vartanaM reading group vyavasthA-viShaye.

Eg: Grills you about progress in projects by saying: 'I thought you would have done this by now. .. So this is what you have done in the past month. .. So you tried a couple of simple algorithms.'

\paragraph*{Extant of success}
indrajitasya saMshOdhanAlaye bhUteH saMshOdhane adhikaH dhyAnaH.

indrajita-apekShayA vegena kRRitAH saMshOdhanAH.

But, this strategy sometimes causes loss of enthusiasm for research, confidence and creativity.

\subsubsection{bhAra-upayoga for the sake of assertion of dominance}
He likes to remind everyone 'who is the boss?' frequently. Sometimes descends to bullying. indrajtia gets angry at defiance easily; but after some time, he becomes reasonable again. He imposes his will on others.

indrajitaH anAvashyakAn parIkShAn kArayati.

\paragraph*{Underlying thought processes}
Perhaps thinks that he must do this in order to retain obedience and for ensuring steady progress in research. indrajItaH tasya balAt maya kakShyAyAM uttamaM prayatnaM kRRitaM iti avadat data-mining course madhye.

tasya dur-aha\~NkArAya shishyeShu atishaya-Atma-vishvAsaH asahyaH. In 'interrogation in class' incident, in a subsequent conversation, he accused me of thinking (and claiming) that I understand technical topics just by having seen it.

\paragraph*{Strong habit}
It is also partly habitual: After I explained some technical reasoning in his office after not doing it satisfactorily in class, his body language was conciliatory.

indrajtia gets angry at defiance easily; but after some time, he becomes reasonable again. Eg: In late April 2010, over email, I said that I could not work on the affiliation network paper because I was busy with the grant proposal. Indrajita stormed to the lab, confronted nAgarAjan, but I was not around. Several hours later, he met us, saw reason and allowed us 3 days to work on the affiliation network project.

\paragraph*{History}
The professor, in insisting on my implementing the graph generation algorithm rather than immediately start work on the proposed new projects, and in doing so sternly, is clearly asserting his power over me.

indrajitaH recco na dAsyAmi yat bhavAn 1 varSha-kAryaM na karoti chet iti avadat jhe~NgdongAya.

He similarly imposes his will on Berkant and Wei at regular intervals: He insisted forcefully that berkAnt try do\~Ngmin's code.

\subsubsection{Starting off new students}
He is not very good at starting students off in research.

\subsubsection{shishya-vAk-anumAna}
Probably strengthened due to stories related by bajAj. Used me and Berkant independently to check Dongmin's code.

He suspected that I may not stay in his lab since beginning of fall 2008.

\subsubsection{Harsh criticism, Atma-vishvAsa-bha\~Nga}
Very skilled at delivering put-downs, adept at condescension. Often results in shaking others' self confidence.

vai-mahodayaM pruShTavAn: 'what was your GPA?' iti (prAyaH joke madhye). Sometimes done by revealing others' flaws in public. Eg: He expressed disapproval about my fumbling with the gradient on the blackboard after my talk by rudely beckoning me back and conspicously shaking his head. Told vai: 'do you want to go into the academia? if so, you must come up with better questions.'


\subsection{Respect: Hierarchical boss point of view}
Also see 'bhAra-upayoga-pravRRitti' section.

\subsubsection{Respect proportional to productivity}
Respects highly accomplished students, those with great potential.

Openly shows disrespect to students he suspects to be non-hard-working or unproductive.

THis is a good way of establishing a good working relationhip with him: gather research accomplishments, gather knowledge and skills with which one can answer his every challenge.

\subsubsection{Discourtesy}
Sometimes asserts dominance over students by showing discourtesy, especially by the use of body language.



\subsubsection{Expectation of diffidence}
asmAt shishyeShu atishaya-Atma-vishvAsAt \\
api kRRiddhaH. Also see pratikAra-pravRRitti section.

\paragraph*{History}
In the 'Interrogation in class' incident, as related during a later conversation, he was very irritated by the impression that, sitting in the back bench with a smile on my face, staring into space, I was not paying attention in his class, and that I was not working hard for his class.

CS398 Fall 2009 kakShyAyAM indrajita-krodhaH ChAtrAnAM apahAsya-anumAnAt.


\subsection{TA, funding}
\subsubsection{Funding strategy}
Usually funds 2 students per semester, while the rest TA.

Usually does not provide funding for the first semester.

Reputed to be stingy in paying students stipends (may not hire for 40 hours in summer).

AchArya's money pipeline: Observe CV.

\subsubsection{TA kAryaM}
Working for him as a TA requires much effort.

\section{Prospective students: strategy}
\subsection{Wooing prospective students}
Impresses prospective students by saying deep and smart things about his research. He is nicer to such students than he is to his own students.

He is also insecure about his lab getting smaller by the day (Oct 2009), and his students going elsewhere.

Hired me only when danger of my going elsewhere for research was imminent. That too, when the money had to be spent on something quickly due to expiring grant.

\subsection{shiShya-parIkShA}
Adept at gauging skills. Tested many different skills (programming, presenting, interacting, course exam) in me.

Carefully observes my performance on exams and homeworks. He had retained linear algebra exam papers and remembered my mistakes in first data mining homework.

\section{itihAsaH}
dIrgha-parIkShA. bhaya-sthApana-prayatnaH. \\bala-prayogaH. avamAna-dAnaM. sukAryAya mAna-dAnaM. ante mayA eva tasya saMshOdhanAlayaH tyaktaH.

\part{mitrANi}
\chapter{rAdhAkRRiShNaH beTTadapauraH}
\section{The obsession: writing}
Above all, he is a literary writer. That is his obsession. No other interest comes close to this. This seems to sap away his motivation for research.

\subsection{Other Interests}
He is also interested in listening to music, playing the guitar, learning about abstract mathematics. He also loves teaching, and is a successful and popular teacher.

\section{Scorn for rigidity}
\subsection{Scorn for the brAhmaNa saMpradAya}
He is scornful of the rigid brAhmaNa saMpradAya he was brought up in. He considers those social rules as an unnecessary restriciton of his freedom.
\tbc

\subsection{Subversive tendency}
Non-rigidity implies compassion for those who do not subscribe to the rigid rules; this in him sometimes translates to subversion of peoples' rigid loyalty to their ideas.

\section{Respect for staunch individuality}
He considers 'social' people somewhat superficial.

\subsection{Low regard for contrarianism}
The reason behind this is unclear.

\section{Skill at karuNA}
Literary writing involves the search for meaning, and an examination of a subject's reactions to various circumstances. He is a master at this.

\subsection{Over-confidence in his appraisal of others}
He has been wrong in his guesses about the things I care about, and about the extant to which I care about them.

He hypothesised my incapacity to see the truth due to cynicism, rather than believe my reading/ conclusions about the body language of Zuckerman, Inderjit etc..

\section{Good Social skills}
His great skill at karuNA translates to great social skills, despite his disregard for highly social people.

\chapter{sucharitA chandran}
rAdhAkRRiShNasya bhAryA. Helicopter pilot at chattisgarh.

Began training as a mechanical engineer at RV College. Loved bikes. Loves ninjitsu, and practices it when she can. Then trained to be a pilot.

The subversive influence which enabled rAdhAkRRiShNa to break out of his brAhmaNa-saMpradAya and even to the point of pre-marital sex. The muse who encouraged and reviewed his literary efforts.

Has good social graces.

\chapter{hRRishIkEshaH bhaTTAchArya}
\section{buddhi-parichayaH}
A verbose person, who tends to make long well formed sentences and communicates his ideas with surprising openness and clarity. He has slowly mastered the art of polite and sometimes enthusiastic greetings in words and gestures. Yet, he likes to think that he is private and did not join Facebook until Jan 2011.

When young, he was deeply affected by his visits to his parents' hospitals. Some patients were more capable than others to withstand the ravages and uncertainty of disease, and he wondered what made them separate.

\section{parivAraH}
His parents are both doctors, who once visited him at UT and went and spoke with Lorenzo Alvisi, when she was concerned with his inability to find an advisor at the beginning semesters of his PhD work. He greets them with 'joi mA kAlI' on phone.

His parents and the parents of a girl who studies in another university in Texas tried to arrange a match, but the girl did not seem to be very interested.

\section{maitriH}
As of 2011, lived in a small room the Goodall wooten dormitory for 5 years. He is older compared to most other inhabitants. He has had trouble with some of the proctors there; but is much loved by other inhabitants of that place; because of his caring nature and his grudging but competent help with their homeworks. He criticizes most of them for being shallow in their conversations - for being mostly concerned with 'who slept with whom', for example. With a few more conscious people, however, he has formed deeper friendships.

\section{saMshOdhanaM cha udyOgaH}
In research, he is curious - reasonably good at both theory and experimentation. At IIT, he got interested in firewalls and internet security. Later, he worked on computational biology, but he did not prosper in Tandy Warnow's lab, and was turned away by her. Then, he spent a semester with Joydeep Ghosh as an agreed upon paper-advisor. Then, he joined Mohammed Gouda's research group. He admires jayadEva mishrA in research and capacity for cold reason.

His intellectual curiosity causes him to attend various classes not related to his research topics - and even think of doing some of the homework assignments.

Knowing the role of politics in impeding peoples ambition to achieve good things in bhArata, he does not want to return there in the near future.

\section{abhiruchayaH}
He likes to play Diablo.

He reads some novels.

\part{UT anyAH}
\chapter{klaivans-AchArya}
\section{Research motivations}
Glory and respect for intelligence, as admitted in his eulogy of sAshA sherstov, are his motivations.

\section{Gossip proclivity}
Upon my sudden, yet firm decision to do research elsewhere, he discussed me with david zuckerman; created an unfavorable impression in his mind, strong evidence of which was observed by his instinctive avoidance of my eyes/ greeting accompanied by an angry frown, while passing by me in the hallways during spring and fall 2009.

\subitem Though it is currently a weak speculation, mishra-AchArya seems to have formed an ill-opinion of me for similar reasons.

Evidence, from a chance conversation with alex suggests his having discussed me with alex Tang as being someone about whom 'no one knows where he came from' during fall 2008.

\section{Provision of truthful evaluation of worth}
Greg Plaxton did a strange volte-face and decided to test my theoretical abilities critically when I approached him for research. While this could have been intrinsically motivated, it is more likely that adverse reports from Adam was the cause.

Later, when I began my attempts to begin research with Prof. Dhillon, who said that he will speak with Adam, took the sudden and surprising decision (again, a volte-face) to test my theoretical and practical skills before assigning me a project.

These incidents, particularly the suddenness and the thoroughness with which the professors decided to test me before even trying to collaborate with me suggests the following as the summary of the most likely report given by Adam to these Professors: "I suggest that you test him thoroughly before you agree to work with him. He may not realize that every professor will need to test him before accepting him as a student. He has trouble handling this fact well.".

\section{Ruthlessness}
He is ruthless in is pursuit of research results and prestige.

\section{Fierce independence}
This is apparent from his favored 'college student' attire, his usually arrogant/ fearless manner of speech. He reacts strongly to any perceived manipulation.

\section{Biases in evaluating others}
In his mind, famous schools and association with great researchers are strong predictors of a person's academic worth. One of his first questions to me was: 'Where did you study?'.

In others, above all, he respects academic worth and 'cool' academic associations. His attitude towards me improved noticably upon observing me in dhillon-AchArya's favor, despite his partly negative recommendation: he made it a point to greet me.

Low certainty: Also, after the faculty meeting during which I was discussed, his attitude has possibly reverted to 'neutral' mode.

\part{Business dealings}
\chapter{Rolando}
\section{Shrewdness and connections}
Manager of matthews properties. A homosexual. Reputed to be duke covert's boyfriend.

Very smooth in communication.

Very shrewd in business dealings. Tough negotiator: Usually makes no compromises.

\section{Desire for peace of mind}
Desirous of freedom from worries and commitments to pursue his own interests, even if this implies being irresponsible.

Does not like to be confronted about dishonesty or irresponsibility. Went incommunicado when he was to hand me over the keys, but to soothe my feelings offered to cut my first month's rent by half. Upon being notified about my anticipation of the check by SMS, letter and email, promptly sent me the check, made out to a higher amount than expected.

\section{Unreliability}
Often does not respond to voice messages. But, likes messaging. Responded promptly to SMS.

Often does not keep promises. Skipped meetings, promises to return calls.

\part{Former roommates}

\chapter{nikhilaH mAlva\~nkaraH}
asasyAhArI-kula-jAtaH api sasyAhAriH. marAThA-saMskRRityAM cha itihAsE abhimAnaM rakShati.

physics PhD prAptavAn, biology saMshOdhana-adhyakShasya sahakArEna, sva-physics-AchAryasya sahAyyaM vinA api. tatraiva post-doctoral saMshOdhanaM kurvan asti.

saha-vAsiShu svArthaM adhikaM prAdarshayat.

bhinnAbhiprAyE jAtE kadAchit sva-abhiprAyAn nEraM na prakaTayati, lamba-vAgbhiH Uhayati kEvalaM.

tarkaH cha dUrabhASha-saMvAdAH tasmai rOchatE. dUrasamparkaiH maitraM rakShituM prayatati.

turuShkAyAH physics saMsHOdhakAyAH saha AptaH jAtaH.

pAchana-viShayE tEna saha bhinnAbhiprAyaH jAtaH AsIt.

\chapter{Hector}
(Last update Jul 2009):

\section{Social adeptness}
Has a strong bond with his mother and sisters, but not with his father.

Derives pleasure from social approval, and from sex. Naturally, has a large group of friends, and a long line of girlfriends.
\subitem Pliable to some flattery, and ofcourse manipulation by his sex-partners.
\subitem Has a strong and useful social network.

Is helpful to his friends, those on whom his pleasure from social approval depends:
\subitem Often provided temporary accomodation to his friends and compatriots.

Is intelligent in negotiation and navigating the world.

\section{Extant of Untrustworthyness for non-friends}
Is very comfortable with lying, for his own convenience.
\subitem Lied to the person from Peru about finding an efficiency.
\subitem Lied to me about not being around when I once needed his help in checking out a car. Guiltily stated a poor excuse, without my asking, when I found him at home.
\subitem Probably lied to me about travelling on the 6th, when we move out.

Shows honor and conscience, especially when confronted by clear recognition of 'ignoble behavior' by others, or when his largesse is clearly visible:
\subitem Upon being faced with the prospect of us parting ways unpleasently in May 2008, upon my refusal to bend to his assertion of right to the room near the pool, he retracted his decision.

Is mostly responsible:
\subitem Pays bills on time.
\subitem But, would not follow up about dealyed return of the san gabriel square deposit.
\subitem Would be hesitant to schedule or keep appointments; whereas he would presume my acquiescence to his schedules.

\section{Cultural arrogance}
Is too comfortable in showing disrespect, as long as he is not confronted about it.
\subitem He would not pickup my calls, or answer my emails. Then one day, reminding him that I am not a fool, I confronted him on email.
\subitem He is very unskilled in writing. His true attitude was apparent in his responses to some of my emails. Naturally has lower liking to confrontations on emails, compared to face-to-face confrontations in which he is far more suave.
\subitem He is comfortable with displaying this attitude in public, and spreading his impression about my being friendless and lonely to his other friends.

Possessing a domineering attitude, he does not hesitate to assert his presumed rights. Often does not negotiate or consult my opinion.
\subitem Would turn the heater off when temperature went low, until I confronted him and put an end to it.

Thinks the world-view dominant in his civilization is superior to mine.
\subitem Once apparently felt that I feel the need to pretend that I drink.
\subitem Falsely thinks that I miss having girlfriends and company.
0\subitem Has a natural scorn for my culture.
                + He is satisfied with his continued ignorance about other cultures.
\subitem Has a misplaced sense of superiority.

\section{Social adeptness}
Has good mechanical skills: evaluated prospective cars well.

His immersion in other activity percludes possibility of his being of significant academic worth.

\section{mayi abhiprAyaH}
COnsiders me, my tastes and my values uncool, and thereby lesser than himself.

He does not consider me to be his friend.
\subitem Did not offer as much help with grocery as he might have.
\subitem Apparent with his lack of invitation to any travel or social gatherings.

\part{parivAraH}
\chapter{vishvAsa}
\section{shrutyAH ardhA\~NgI}
tayA mahat-prabhAvitaH.

\section{susaMskRRiti-bhakti}
anuShTAna-shAlI, brAhmaNa-samskRRiteH bhaktaH abhimAnI cha.

\subsection{susaMskRRiti-rakShaNa ichChA}
tasmAt sva-bhAShA prayogaH adhikatayA.


\section{yogaH}
\subsection{satya-bhaktiH}


\subsubsection{nisarga-bhaktiH}


\subsection{karma-yoga-prayatnaH}


\section{Strong intellect}

\subsection{doShAH}
chitta-sUtraM pashyatu.

\section{Suceptibility to stress}


\section{virodhI-svabhAva}


\subsection{samAje-akaushalaM}


\subsection{apahAsya-pravRRittiH}
mama apahAsya-pravRRittinA anyeShu \\
utsAha-bha\~NgaH. prAyaH anyAnAM atishaya-nindA-pravRRittiH api asti.

\subsection{Disgust at money making self help organizations/ people}
They often fool themselves and others into believing irrationality. They try to sell remedies for achieving 'eternal bliss', and don't deliver the goods in the end.

\section{itihAsaH}
\subsection{Meta-cognition, tradition}
For a long time, I was fooled by religious nonsense.

\subsubsection{Initial belief in a personal, intelligent IShvara}
First, I was brainwashed into believing in the existance of a certain psychopath called God. This was in early childhood. This belief was extremely strong. During a trip to the nIlgiris with amma's bank, for example, I remember circumabmulating thrice on the spot while batting during cricket etc..

It was clearly present when, in middle school I used to perform Arati and sandhyAvandanaM; and when my grandmother died before my eyes (in 1991) and I prayed in vain.

\subsubsection{Later, belief in convergence to a perfect state of mind}
Even when I overcame that notion, I was plagued by the notion that there is a "perfect state of mind". I used to think that "genius" is a permanent state of mind. I had no idea that it is impossible to control the mind moment-to-moment. This continued until the end of 2006.

In 2004-2005, I had read 'Zen and the Art of motorcycle maintenance' and lIla by robert pirsig very slowly. In 2006, I took a year off, purportedly to prepare for higher studies; but I actually indulged myself in my dominant aim - attaining 'the perfect state of mind'. In that year, I repeatedly fooled myself into thinking that I attained the perfect state; indulged myself in discussions and debates in online fora. I used writing and debate as a way of finding flaws in my understanding of the 'perfect state'. I failed repeatedly.

\subsubsection{Lack of ambition}
I once harbored a notion that ambition is bad. I even believed in nationalism. I had not realized the role of guile in animal society.

\subsubsection{Convergence to rationality}
By the end of 2006, I began to feel the influence of reading UG kriShNamUrti. FOr the first time, I began to truly doubt the existance of the "perfect state of mind", free from any maintenance. I saw that the belief in the existance of the unshakable, ``perfect state of mind'' was irrational.

As a side effect, I became virulently opposed to any idea espoused by tradition, culture and religion, even good and sane ideas. I embraced the biological perspective of propogation of the genes as the ultimate purpose of life.

\subsubsection{Embracing noble aspects of brAhmaNa culture}
By summer 2008, I had spent the past 9 months in USA - 4 months at UMass Amherst, Massachusetts, and the rest of the time at Austin, TX. Before I went to the USA, I was an ardent admirer of the American way of life. I still am, but my admiration is now qualified and limited. I saw the American way of life. Besides befriending other Indian students, I had a chance to know folks from USA, Turkey, Korea, Europe and Latin America.

During my vacation to India, I re-evaluated the strengths and the weaknesses of the Western culture and my culture. I was forced to do so by the dissonance between them, which struck me upon my return.

Though I did meet nice westerners (especially our super-nice department secretaries), I found that they were more likely to be ignorant (excepting Professors and graduate students, of course), somewhat angry and arrogantly condescending towards people from the east. (I even saw outright racism a couple of times, mainly from administrative/ government officials.) They seem incurably convinced of the superiority of their way of doing things. They hold very wrong opinions about the outside world. Eg: They sometimes don't know that India is a democracy, are often confused between economic class and caste.

Young westerners seem to date many people. They have no clue about Indian arranged marriages. I read that the divorce rate in USA is around 50\% compared to 1 or 2\% in India.

Many young people in India do drink and smoke - but I found that western young people (even females) are more likely to drink and smoke. The 6th street in Austin on a weekend night is filled with completely drunk youngsters yelling at and fighting each other, getting arrested etc..

To evaluate my culture, I considered our past failures and current successes. We were once more susceptible to mysticism, but now it has now come to plague the West too. Otherwise, we seem to have overcome our past follies while retaining the valuable parts of our culture. I see very bright prospects for life in the West, but I also see that I do not want to live like a native American, nor do I want anyone in my family to do so. Hence, I decided to finally adapt the increasingly popular combination of critical/ rigorous way of thinking together with our ancient tradition.

\subsubsection{Yoga}
I was initially taught yoga by my father in early childhood. I had gone to a yoga expo and Yoga class in sAdhana-sa\~Ngama. I had learned to stand on my head.

In early childhood, I used to go to karATe classes, I used to practice it with dedication for some time. I used to jog in the park, wash my own clothes after getting up at 0530 when young. I used to play badminton at the malleshvaraM association. I also used to go the gym some times, and go swimming. In early childhood, I liked soccer, I was a fan of the telesoccer 2 minute fitness test. I was reputed to be a good defender. In all these activities, I aspired for attaining a state of great grace and effortless effectiveness.

But, there were other physical activities in which I did not aspire for excellence so much. I tried going to table tennis classes in 5th or 6th grade perhaps, but gave up when I wasn't allowed to go beyond 'shadow practice' for a couple of months.

\subsection{saMshOdhana-itihAsa}
I was fascinated by what I saw in the Discovery channel and the National Georgraphic magazine. I loved to read about the lives of scientists and their discoveries in middle school and high school. I used to fantasize in my long walks with manjunAtha about being either a scientist, or a hermit- a master of my own island.

But, I was much too ignorant of "research" life. I had not seriously considered the prospects of doing research, what it involves, and the way to get there.

I spent far too much time learning Sanskrit. I was far too absorbed in fiction and poetry. I watched too many movies and television. I spent far too much time reading history, civics and non-quantitative economics. I was distracted by too many puzzles. I spent far too much time learning to stand on my head and to breathe in special ways. Unfortunately, I did not realize the need for the selection and deep study of a field of knowledge in making new discoveries.

I studied geometry, algebra and parts of physics somewhat thoroughly. Unfortunately, I did not commit big stretches of time and effort required to truly master them. Furthermore, those efforts were not sharply focused on preparing for future research. I did not realize that research and research-preparation are a full-time occupation.

For a long time, I was unaware of the true nature of computer science and bioinformatics research. The only truly fascinating fields of research were theoretical physics, organic chemistry and biology, for all I knew. I used to think that artificial intelligence is just about engineering robots. Until last year, I was unaware of most of the entire field of artificial intelligence; and of the fact that the purpose of writing programs in many parts of computer science is not just to engineer new systems, but to gather data for further theorizing. (Such astounding ignorance!)

\chapter{shrutiH}
\section{mama patnI}
ataH mama uparI \\
mahA-prabhAvashAlinI. ahaM cha tasyAH uparI prabhAvashAlI.

shrutyA saha mama AkarShaNasya a\~NgANi shArIrikaM, pArivArikaM snEhaM, prajanakaM cha. 

\section{dhairya-mitiH}
Superficially shy; vAstave dhIrA. The boldness stems from her trust in her intuition.

\subsection{nUtanatva-bhItiH}
sA nUtanatvAt sulabhatayA bhItA. yEShu pUrva-anubhavaH vartatE tat Eva bhaviShyE api anusarituM prayatati.

\subsection{adhairyaM}
avisha-sarpa-sparshAt bhayaH - \\nyUnaM nivAritaH seaTTle-shikShaNAt. saudAmini-bhayaH.

\section{buddhi-balaH}
Has a deep craving for meaning. Has a strong intellect.

Has high capacity for self discipline.

prayANE dik-anvEShaNE dOShANi anEkAni pradarshayati.

\subsection{adhyAya-balaH}
BASE tuition agachChat. snAtaka-pUrva-vidyAbhyAse prApnot padakAn.

\subsection{vikAra-nishchaya-avalambanaM}
Highly flexible - nishchayEShu anEkataH vikArAn anusarati (used to call it 'hari'). tasmAt dOShaM gachChati kadAchit.

\subsection{Mechanical kAryEShu akaushalaM}
Mechanical kAryEShu akaushalaM, anAsaktiM cha akShamAM pradarshayati.

\subsection{sUchanA-upayogaH}
\subsubsection{mitiH}
anyebhyaH tat-krodha-ullekhAt tasyAm krodhaH jAtaH. (bhaviShye mayi sahajam doSha-agopanam na parivartate iti spaShTIkRRitavAn.)

\section{asantOShaH}
\subsection{mahAkrodhaH}
tasyAH alpa-viShaya-mahA-kopa-pravRRittiH.

\subsection{ichChA-vachane asAmarthyaM}
ichChANAM spaShTatayA vachane asAmarthyaM.

\subsection{asUyA}
shrutiH AsUyAM prAkaTayat pUjA-maitre. tasmAt tAm mama google-chat-shreNyAH api niShkAsitavatI. (tat ayogyam, mayi prabhAvahInam iti spaShTIkRRitavAn.)

anaghA-sparShAt api asUyA anubhUtA tayA OM-shAnti-dhAma-yAtrAyAm.

szu-hwA-yAH sA\~NgItika-ruchi-gRRiha-shuchi-prashamsAyAH cha AsUyA-grastA jAtA.

\subsection{prabhAvaH madupari}
etat tasyAH \\sahavAsena mayi api kiMchit dRRiShTaM - etat nirodhitavyaM.

Feb 2012 saMbhAShaNaM:

A: Are you excited that your wife is coming?

B: .. i am a little scared actually.

A: Why?

B: She gets very angry easily. 


\section{laukika-kaushalaM}
laukikAn niyamAn pAlayati, sAmAnya strI-saundarya-vichArAn api. 

\subsection{AtmahimsA bhayotpAdanam cha lajjotpAdanam}
anekadA 'mamariShAmi' iti uktvA bhayam uptAditum prayatitavatI. (tat ayogyam, mayi prabhAvahInam iti spaShTIkRRitavAn.)

'bhavate anya-kAryam adhikam rochate chet aham na AgamiShyAmi' iti lajjAm utpAditum prayatitavatI. (mayi prabhAvahInam, yat tasyai rochate tat eva karotu iti spaShTIkRRitavAn.)

\subsection{vAde adhairyaM}
sahakAribhiH saha vAdaM kartuM bhIta asti. fear of confrontation.

\section{saMskArANi}
kadApi madirA-pAnaM vA dhUma-pAnaM na akarot.

\subsection{chitta-guNa-vardhanaM}
sva-chitta-guNAnAM uddhAraM ichChati. svasyAH \\chitta-vRRittiShu doShAn ('bugs') vindati anyaiH cha AtmanaH saha saMbhAShaNebhyaH.

purA AsIt paramahaMsasya yogAnandasya bhaktA, kriyA-yogiNI.

\subsection{mitiH}
sadA A\~Ngla-bhAShAM upayojayati, yatkimapi mAyayA Avaratena lokena prashaMsitaM, tat eva durabhyAsaM anusarati kadAchit.

ku\~Nkuma-dharaNE sa\~NkOchaM prAkaTayat, cha ma\~Ngala-sUtra-dharaNE aprItiH adarshayat. tarkAnantaraM tAbhyAM dharaNE siddhA abhavat.


\section{gRRiha-pAlana-kArya-vahana-niShThAyAH mitiH}
kadAchit gRRiha-pAlana-kAryAn svayaM eva vahati.

\subsection{AlasyaM}
kAryAn na vahati kiMchit api kliShTaM vA \\sharIra-bala-yAchakaM chet. udAharaNAya 'washer/ dryer' chAlituM vA cart vAhanAya AnItuM vA peTikAM nUtana gRRihaM AnItuM vA asamarthA ahaM iti kathayati.

\section{itihAsaH}
NIT sUratkal-madhye prApnot pitaroH \\
svatantra-chittaM. tatra Apta-snehitayA pallavikayA saha adhyAtmaM apaThat.

BASE-taH aneka varSha-paryantaM tayA \\
kAmitaH IIT maDrAs vidyArthiH ekaH; parantu tena nirAkRRitA.

purA AsIt paramahaMsasya yogAnandasya bhaktA, kriyA-yogiNI.

\chapter{vidyA}
\section{Set-Blindness to faults}
She strongly prefers to be around people who will respect her or be nice to her despite her faults.

\section{Irresponsibility}
She is irresponsible and incompetent in household affairs, and unrepentant about it. She does not follow through on her promises.

 \subitem Agreed to book tickets for me to India in 2008, did a bad job, with few inquiries.
 \subitem Did not clean her apartment in time, pack well and nearly caused me to miss my airplane.
 \subitem Did a bad job in preparing parents' visa interview.

She is unresponsive to emails, even those that demand follow-up.
 \subitem Did not even say 'can't do' when I emailed her about making a list of documents to be taken to visa interview by parents.


\section{Vulnerability to grief, coping techniques}
Is known to cry when faced with strong disapproval from me. When I am physically present, grief is revealed by her blushed nose, though tears are deliberately hidden.

She copes by cutting off contact, confiding in other forgiving people. She freely discloses my weaknesses to others.

\section{Great social network}
Has done me favors: Her huge network of friends has indubitably helped me during past efforts to find a job, temporary accomodation in Austin.

\section{Opinion about me}
Thinks that I am arrogant, rude, selfish, and a smart-ass. She has some limited admiration for my scholastic accomplishments.

\section{Old cause for trouble}
Grief has come on either side due to a vague ideal about what constitutes a good 'brother-sister' replationship, without being able to be clear about it to each other.


\section{sahodara-saMbandha}
\subsection{Be wary of delegating responsibility to her}
Past experience has shown that the sibling, even when she agrees to do a certain thing (such as booking an airline ticket), does not follow through. Hence, as far as possible, it is necessary to not ask any favors, even if she claims to want to take a responsibility.

\subsection{Visits to vidyA and ravi}
The intersection between their values/ tastes/ lifestyle and mine is tiny. This can cause visits to be unwelcome to them. Visits there must be made rarely (less than once in 2 years), and for very short durations (at most 5 days); upon planning to avoid distasteful activities and wastage of time.

\subsubsection{History}
In Jan 2011, travelled with vidyA and shruti to kanchi. That journey was pleasant too. 

Aug 2010 visit with shruti went well, without major incidents. Observed that ravi was very reluctant to let me drive their new Toyota prius, but drove it after he left for another city.

During visit to vidyA and ravi in Seattle: Dec 2008 - Jan 2009: I did not like being blocked access to the bathroom during the night, the fact that my visit was not desired by ravi, my opinions generally not being solicited in planning the day, the humungous waste of my time, having to tag along in many shopping mall trips, the palpable disrespect shown to me by ravi, the irresponsibility in packing/ moving/ travelling to the airport, the late bed time. I liked exposure to the lifestyle of settled and married couples and visits to 2 museums.

During visit to vidyA in Arizona in Dec 2008: I did not like the late bed time and the delay in the travel to the airport; and liked the lively company, glimpses into the lives of the students, road trip to surrounding places.

\chapter{pitA}
Has some limited respect for tradition: ati-saMkShiptaM sandhyA-vandanaM karoti.

\section{laukikatA}
Craves social approval and appreciation. Has good social skills. Can make others laugh.

\section{Bad habits}
Drinking, smoking, tobacco, unhealthy food.

\section{Competence}
Highly creative.

\section{Tendency to hide weakness}
Hid difficulty in walking to the river.

\section{(Un)Trustworthiness}
Often does not keep promises, is late to appointments, is insensitive to discomfort to 'unfashionable' others.

But, is trustwothy on fulfilling the most important responsibilities. These include faithfulness in marriage.

\section{pratiShTA}
Not skilled at benefiting from feedback, nindA, apahAsya, especially in front of others.

\subsection{shishu-vat virodhaH}
He was upset by damage to his pratiShTA in front of mAtula. Did not receive 5 phone calls from me, when he was absconding on Jan 3 2010 in Austin, while the rest of us planned pariyaTana of Zilker park and mozart's cafe, inconveniencing us. Later, he refused to join us, seemingly expecting endearing pleas to the contrary. He then realized its futility of this approach to persuading me, and returned to normal.

\section{itihAsaH}
Grew up close to poverty, where food was not abundant. Was subject to powerful bad influences.

\chapter{mAtA}
\section{vAtsalyaM}
snuShA-samEta apatyAnAM ArOgyAya cha kaushalAya utsukA.

Sacrifices her own interests for the benefit children.

\section{Unskilled in contributing opinion}
Perhaps because of negative reinforcement from pitA, where her opinions were wrongly ignored without proper explanation, she has formed the habit of repeatedly stating her opinions in different ways. This causes irritation and disturbs concentration of others.

\subsection{sva-yOjanAnusaraNAya bhEda-prayOgaH}
shrutyA cha mAtarA saha vAdaH ka∼nchi-prayANasya viShayE january 2011 varShE kRRitaH. mAtrA bhEda-upayOgaH cha ati-pAtaH kRRitaH, tat mayA dRRiDhaM ninditaH.

\section{pitRRi-pakSha-saMbandhaH}
pitRRi-pakSha-saMbandhinaH sarvE .

\part{mAtA-parivAraH}
\chapter{mAtAnuja}
Respects tradition deeply.

Likes gambling. Has made bad financial decisions: put 3 lakhs in a failing bank.

Used to eat meat, smoke, but gave it up when in contact with me in Dec 2009. Now does sandhyAvandanaM.

Cares about his family. Likes his dominant role.

Has rather low opinion of pitA's qualities.

\chapter{mAtAmaha}
Has high moral integrity: was viciously against bribes while in government service.

Went through great difficulty in early years, in obtaining a good education.

Rigidly set in his habits. Imposes these habits on others.

Lacks creativity, does not experiment, unwilling to take risks, fears negative consequences too much.

Easily angered. When angered, he freely uses lewd vocabulary from his rather boorish village upbringing.

Devout. champakA-purasya shrInivAsasya bhaktaH.

\part{vidyA-parivAraH}
\chapter{ravi-mAtA kamalA}
Liked village life. Moved to be\~NgaLUru to be with ravi.

For a silly worldly reason, she declared that chandru is an adapted child of her uncle. This affected chandru badly, and perhaps drove his schitzophrenia to surface with a strong instinct against his parents.

Very fashionable and social.

Does not approve of vidyA's abandonment of her traditions - getting a boy's cut.

Wants children to move back to bhArata. Dreams of taking care of her grandchildren.

Started her career in her mid 40's, after her parents in law had died. She then became a teacher with Government schools, and then she did her MA in kannaDa meticulously and well.

She studied saMskRRita after mAtA pegged it to her.

\chapter{ravi-pitA malliArjunArya}
yOgI who did yOga regularly. In his old age, he seems to have reduced dainika-abhyAsa. Influenced by and participates in the theosophical society.

hindI-bhAShAM pAThayat.

kadAchit anArOgyaM agachChat. tat-anantaraM udyOgAt nivartat.

\chapter{ravi-bhrAtA chandraH}
\section{manO-rOgaH}
Schitzophrenic on medications. itihAsaH - Started hearing voices in his head - voices he could not ignore. His verbal and physical aggressiveness heightened. ravi had to return from america to tend to the situation. After much effort and false starts, he accepted regular medications.

\section{parivAra-sambandhaH}
Was very disaffected by his parents declaring on paper for bureaucratic benefits that he has been adapted by a close relative.

mama kEchana guNAH tasmai rOchatE. Admires ravi and vidyA to some extant.

\section{vidyAbhyAsaH}
Had dropped out of BE and out of agricultural college.

About to complete BCA, but must first pass a few remaining exams.

ravi and vidyA try to increase his discipline and structure in his life so that he can settle into a good professional routine.

\section{abhiruchayaH}
Catches a fad once in a while - entrepruneurship (ipad menus in hotels), 'Law of attraction' etc..

Rides bike scarily but with skill.

Has a good friend-circle from his college.

\chapter{ravi}
\section{dharma-tiraskAra}
Has disregard for classical dharma, though he appreciates some scientific accomplishments. His friend committed suicide in auroville.

He supports jAti-based reservation. In be\~NgaLUru, he once argued with me about its usefulness, became visibly angry and left.

\subsection{saMskAra-tiraskAra}
pitA yadA shAstra-anusAraM kavachaM udghATayat '(yajJNOpavItasya) pradarshaNEna alaM' ityavadat. mama AchArANAM viShayE api tathaiva avadat -\tbc

\section{vidyAyAM avishvAsaH}
tena ruddhaM vidyayA puMsnEhita-sAmipyaM - udAharaNAya tasya anastitvE puMsnEhitaiH saha mElanasya rOdhaH.

\section{Management skills and desire}
wants to run a company, earn money. Highly skilled as a manager. Has a 'voice'/ tone and verbiage used to great effect for management tasks.

\section{Behavior on visiting brAhmaNa homes}
Does not realize that he is expected to wash plates.

\section{mayA saha saMbandhaH}
\subsection{Distancing}
Consciously or unconsciously trying to put a distance between me and him, he has used his professional tone of voice and addressing on email and on phone. snEha-rahitaM uttaraM adadAt yadA ahaM prajJNA susaMskRRitA bhavatu iti hita-AShAM prAkaTayaM (janmAt laghu-kAla-pUrvaM).

\subsection{avamAnaH}
\subsubsection{dambI-ArOpaH}
June 2011 prathamE saptAhE mama Redmond-yAtrAyAH anantaraM, tasya kalpanAH cha AprOpAni shrutAH. saH avadat - vishvAsaH sva-vEshEna adhikaM pradarshaNaM karOti, ataH Eva UT prAdhyApakaiH saha kAryaM kartuM na ashaknOt, ataH Eva kAryaM prAptuM na shaknOti iti. tadanantaraM ahaM E-sandEshaiH spaShTIkaraNaM dattavA, su-tarkaM prArabhituM prAyatam. tadA avamAna-dharat tuchCha-bhAShA-likhitaM sandEshaM 'dambI asi' iti ArOpEna saha prEShayat. 'EtasmAt bhavataH gRRihE mama AchArEbhyaH svAgataH nAsti' iti mayA uktasya nirAkaraNaM na akarOt. EtasmAt saH mayi agauravaM rakShati iti jJNAtvA vidyAyai api vyavRRiNam.

\subsubsection{anukriyA}
tadA pUrva-yOjitAM avamAna-pratikriyAM chAlayan amazon-parIkShA-avasare tat-gRRihaM na agachChaM. saH Eva mama parivArEna saha mama prAkOShTaM Agatya agachChat, suvyavahAraM alpaM prAdarshayat.

\section{tarka-akaushalaM}
jAti/dharma-viShayakEShu tarkEShu kOpaM agachChat dvivAraM - yathA anyatra vivRRitaM. dvitIyE tarkE vaiktika-nindAH ArOpayat.

EvaM tarkE anyasya satya-niShThAyAM sandEhaM gachChan, anyasya tarka-uddEshaH kalmashaH iti ArOpayati. udAharaNAya Aug 2010 yAtrAyAM mama gauDa-vaishNavEna saha tarkE.

\part{shruti-parivAraH}
\chapter{shruti-mAtA jAnakI}
\section{itihAsa}
Brought up in be\~NgaLUru. Daughter of shrInivAsa rAghavan ayya\~NgAr from kumbhakONaM who had moved there. Studied abroad for 2 years, apatya-jananAnantaraM, vishvanAthasya uttamEna sahakArENa.

\section{samasyA-pariharaNE kaushalaM}
Very organized.

\section{gRRihE prabhAvaH}
Assertive and dominant in the house. Close to shruti.

vishvanAthasya atithi-satkArAdiShu vartanaM parishOdhayati parivartayati cha.

\section{maitri-rakShaNE kaushalaM}
bandhubhiH saha dIrghakAlInaM maitriM rakShituM prayatati, madhuvAchaH upayojayati.

\section{sImitA saMskAra-bhaktiH}
gRRihanirmANE shvasharA labdhasya gRRiha-dEvasya gaNapatEH vishESha-bhaktA.

\section{abhiruchayaH}
Likes singing and listening to old hindI film songs.

shruti-prabhAvAt self-help pustakAn paThati, yogaM adhyayati.

Like her husband, interested in having a good house, with creepers etc..

sahakAribhiH vaidyaiH saha maitriH.

\section{laukikatA}
Was concerned with her 'image' in society : seemed to want a fashionable marriage.

\section{pratiShThA}
Asserted her importance in their daughter's life by insisting needlessly on interacting with me in person to judge my yogyatA.

asmAkaM vishEShaH sampat iti viparyayaH.

\section{Other traits}
Very emotional some times.

svI-kUrvantI asti shruti-svAtantryaM idAnIM.

\chapter{shruti-pitA vishvanAthaH}
\section{samasyA-pariharaNE kaushalaM}
Logical. pratidinaM svi-trayaH sudoku, kuroku krIDAH krIDati.

\section{abhiruchayaH snEha-vRRindaH cha}
kiTTy cha puTTu shvAnau. sahakAribhiH vaidyaiH saha vyAvahArikaH snEhaH.

pArshva-gRRihiNaH Apta-mitraH, tEna saha aTati, tat-sahayyaM karOti kaShTa-kAlE.

Repairing/ tinkering stuff around the house: a natural engineer.

Likes to keep a 'good' house.

\section{maitri-pradarshaNE akaushalaM}
Reserved. bandhubhyaH atithibhyaH cha maitri-pradarshaNE akushalaH. shruti-mAtuH prabhAvAt atra kadAchit adhikaM prayatatE.

\section{hindu-sampradAyE sImitaH gauravaH}
\subsection{mUla-tattvEShu bhaktiH}
Was a keen member of RSS. Has read bhairappa's books. tamAt musalmAna-kraistEbhyaH saMskAra-prabhAvAbhyAM alpa-virOdhI asti.

sharAdApIThasya chikitsAlayE pramukhaH vaidyaH san shri\~NgEri-shankarAchAryE gauravaM rakShati. asmat-vivAhE sha\~NkarAchAryasya upahAraM dRRiShTvA arOdat.

\subsection{AchArEShu abhaktiH}
dainikAn mantrAn anEkAn na smarati. purOhitEna shikShitAn mantrAn api yajJNAdi-kAryEShu samyak na anuvadati.

shAstrIyaM vEshaM shRRi\~NgEri-yAtrAyAM dharataH mama viShayE asantOShaM putryai prAkaTayat. vivAhAnantaraM sva-gRRihAya dhautasya dharaNaM asamIchInaM iti avadat.

\section{mitaM ArOgya-rakShaNaM}
dhUmaM vA madirAM na pAti.

vyAyAM na karOti.

Skips meals often. Overworks.

\section{laukikatA}
shAstrIya-vEsha-bhUShaNEShu agauravaH tasya vEsha-vartanAdiShu laukikatAM pradarshayati.

\section{agauravaH mayi}
mama shAstrIya-vEsha-pravRRittyAM asantuShTaH tasya AchArEShu abhaktEH.

mama yOgyatAyAH svAbhimAnasya viShayE api sandigdhaH. shRRi\~NgEri-yAtrAyAM  apahAsyaM adarshayat (karATe-abhyAsaM kUrvatA saha saMbhAShaNE jAgarUkatA bhUyAt, asti-pinjaraH iti.).

sva-pratiShThAyAH rakShaNasya api a\~NgaH syAt agauravaH. dUravANi-saMbhAShaNE kadAchit GRE adhyAyE shruti-mAtuH 'GRE small book likhEt' iti hAsyasya uttarE 'shrIrAmAya na' iti spaShTIkaraNaM adadAt.

kAryEShu sva-adhikAraM sImAyAH adhikaM Eva rakShayan anyAn laghUn karOti. shru\~NgEri-kShEtrasya paryaTanE nEtRRitvaM pragRRihan saha anya-sammatiM na aprApnOt yAtrA-yOjanE.

\chapter{shruti-bhrAtA shrIrAmaH}
\section{laukikatA}
Social, fashionable. Likes good clothes. Very friendly. Has huge social circle.

\section{abhiruchayaH}
Becoming serious at studies. Was a state level TT player.

\chapter{shruti-mAtulau}
AnandaH jayalakShmI-patiH cha rAmaH. tEbhyaH mama shAstrIyA pravRRittiH na rOchatE.

\chapter{badarI-nArAyaNa-shAstrI}
vishvanAtha-purOhiEna badarI-nArAyaNa-shAstriNA kAritaM varapUjA, kanyA-dAnaM, kanyA-samarpaNaM, gRRihapravEshaH, uttara-kriyA anEkAni api saMkShiptAni Asan. badarI-nArAyaNEna na anusRRitaM asmat-upavAsaM, arundhati-nakShatra-darshaNaM AdayaH saMskArAH.

\part{pitRRi-pakShaH}

\end{document}
