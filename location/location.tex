\documentclass[oneside, article]{memoir}
% \input{../../../work/packages}
\input{../../../work/packagesMemoir}
\usepackage{fontspec, xunicode}
%\setmainfont[Script=Devanagari]{Chandas}
\setmainfont[Script=Devanagari]{Kalimati}

\input{../../../work/packagesMemoir}
\usepackage{fontspec, xunicode}
%\setmainfont[Script=Devanagari]{Chandas}
\setmainfont[Script=Devanagari]{Kalimati}

\input{../../../work/packagesMemoir}
\usepackage{fontspec, xunicode}
%\setmainfont[Script=Devanagari]{Chandas}
\setmainfont[Script=Devanagari]{Kalimati}

\input{../../../work/packages}
\input{../../../work/packagesMemoir}
\usepackage{fontspec, xunicode}
%\setmainfont[Script=Devanagari]{Chandas}
\setmainfont[Script=Devanagari]{Kalimati}

\input{../../../work/packagesMemoir}
\usepackage{fontspec, xunicode}
%\setmainfont[Script=Devanagari]{Chandas}
\setmainfont[Script=Devanagari]{Kalimati}

\input{../../../work/packagesMemoir}
\usepackage{fontspec, xunicode}
%\setmainfont[Script=Devanagari]{Chandas}
\setmainfont[Script=Devanagari]{Kalimati}

\input{../../../work/packagesMemoir}

\title{Location}
\author{}

\begin{document}
\maketitle

\part{Location strategy}
\section{parichaya}
Make location strategy (environmental threats, flood prone cities, rising sea levels, global warming, radical religion, population pressure, xenophobe populations, university quality, funding situation, environmental damage, government oppression, social collapse, economic collapse, resource abundance, intellectual culture, education quality, scientific capability, spaceflight, crime, medical care: Europe universal healthcare system: height increase: US , disease, immigration restrictions).

Buying power. How many eggs can you buy with the salary that you get?

Health care costs.

Diversity in the population and culture.

\section{bhArata}
\subsection{Getting good jobs}
Better access to bright students. See career strategy.

\subsection{Standard of living}
Middle class can lead a comfortable life, with the unpleasantness and the slums walled out in a way.

\subsection{Health problems}
Allergic rhinitis affects vishvAsa in be\~NgaLUru - this makes strong ang clear breathing difficult; it can possibly be treated using rush immunotherapy at KIMS.

Access to organic fruits is harder.

\section{USA}
\subsection{Advantages}
High number of places with significant concentrations of orthodox South Indian population.

\subsection{Disadvantages}

Medical care crisis:

    * Medicine and medical care is very expensive.

    * Insurance is costly for older people, such as parents residing with you.

    * If you suddenly become unemployed for some time, medical costs can render you pauper.

Immigration difficulty:

    * Parents are not considered dependents. They may be sponsored only when you become a citizen.

\subsection{Strategy}
1. F1 6Y study.

2a. 1Y industrial training. O-1 or H1B work. EB1 or EB2 GC.

2b. IR 2Y. Unconditional IR GC 1Y.

3. naturalization. PIO. IRs.

\section{UK}
\subsection{Advantages}

Good academic job opportunities.
\subsection{Disadvantages}

South Indians and orthoprax brAhmaNa are few.

\section{Continental Europe}
\subsection{Advantages}

Good academic job opportunities.
\subsection{Disadvantages}

Language barrier.

Low concentration of orthodox south Indians.

Long naturalization process.

    * Includes language tests.
    * 12 years needed in Switzerland. 5 years in a job required in Luxembourg.

\subsection{Germany}
machine learning research excellence.

\section{Canada}
\subsection{Advantages}

Easy to acquire citizenship:

    * Even years as a student count.
    * 2 or 3 years to get permanent residence.
    * 1 or 2 years after that to get citizenship.

Highly welcoming and cosmopolitan.

Easy to work in USA with a special (TN?) visa tied to a particular company. When you change your employer, you must visit the border and get a new visa.

High number of places with significant concentrations of orthodox South Indian population.

Low cost medical care.
\subsection{Disadvantages}

Tough industrial job market. Economy only as large as Texas. See notcanada.com .

High tax. In fact, government is said to attract immigrants as a way of earning tax dollars.

\section{Australia and New Zealand}
\subsection{Advantages}

Racist violence:

    * In early 2009, there were many reports of hate crimes, robberies and assaults targetting Indian students in Australia; drawing much ire and protests.
    * New Zeland saw hate-crimes against Chinese students and lost students from there, as reported in early 2009.

\subsection{Disadvantages}

\part{USA Visa details}
-------------------------------------
\section{B Business/Visitor Visas}
Acquire health insurance if needed.

\begin{verbatim}
----------
B-1 visa
    Business travelers who wish to visit the U.S. for a short period of time to conduct meetings, attend conferences, promote business development, or engage in other similar activities, should apply for a B-1 visa. Generally speaking, it is not permitted to work or seek employment while in the U.S. on a B-1 visa.
    Most business visitors are usually given B1/B2 visas, allowing them to enter the U.S. for either business or recreational purposes. B1/B2 visas may be issued with a validity of any length of time up to ten years.
    Typically, travelers entering on B1/B2 visas are permitted to remain in the U.S. for between three and six months. This decision is made by immigration inspectors at the port of entry.
    ----------
    Required Documents :
        Common Required Documents / Photos
        # Invitation letter from the US counterpart / Company with which you are going to conduct business.
        # Letter from your employer or company that is sending you to the U.S.
        # Exchanges of correspondence with U.S. purchasers, suppliers or contacts
        # Recent contracts, bills of lading or other documentary evidence of recent imports and exports of purchases
        # Any evidence of arrangements for lodging and appointments already made in the U.S. in preparation for your trip
        # Incorporation documents and business license(s)
        # Tax statements for the last 2 years
        # Bank transaction statements detailing all deposits and withdrawals for the last six months. Include your personal and company accounts, along with evidence that you have authority to draw on the company accounts
        # Any copies of recent advertising brochures or publications.
        Business and Tourist visa applicants are required to demonstrate their ties to India.
----------
B-2 visa
    The B-2 visa is for foreign nationals seeking to enter the U.S. for a short time for the purposes of tourism, social visits, or other recreational activities.
    ----------
    Required Documents :
        Common Required Documents / Photos
        Business and Tourist visa applicants are required to demonstrate their ties to India.
            property documents
            capability certificate
        # If the applicant is employed, the employment and salary details
        # Income tax statements and documents to show the assets of the applicant, if any
        # A No Objection Certificate (NOC) or Letter and leave sanctioned letter if the applicant is in Government service or with any of the Armed Forces.
        6 months bank statement
       
        If you have a sponsor or are visiting family members in the U.S.:
            # An Affidavit of Support, Form I-134 from the sponsor, and also their bank statements and employment letter
            # Passport copy of the relative in the United States. Preferably, a copy of the relative's Indian passport or any other document that show proof of relationship, if possible
            # Documents to show the sponsor's legal status in the United States

        When invited by an organization:
            letter of invitation
            hotel booking vouchers or letter
           
   
    Manju's List:
        [Take two copies, take originals]
        From Office:
            No objection certificate
            Employment certificate
            Letter from chairman to Visa Officer
            Photocopy of the ID card
            Web Page of the organization (with contact address)
        Travel:
            Affidavit of support
            bank statements
            tax statements
            Itenary
        Destination:
            Invitation letter
            Hotel details
            Program schedule
        Visa office:
            Forms 156 and 157
            Appointment letter
            HDFC bank statement
            passport
            covering letter
        Personal things:
            BE certificate
            10th and 12th certificates
   
----------
Eligibility:
A temporary visitor for business or pleasure must establish that he or she:
    # Has a residence abroad which he or she does not plan to abandon
    # Is coming to the United States for a definite temporary period
    # Will depart upon the conclusion of the visit
    # Has access to sufficient funds to cover the cost of the visit and return passage

When planning to apply for a business/tourist visa to the U.S., ask yourself these questions:
    # What is the purpose of my trip?
    # What evidence can I show to demonstrate that I will return to India?
    # What do I need to show that I can afford this trip?

----------
\end{verbatim}

\section{F1}
\begin{verbatim}
----------
I-20 certificate:
This certificate is issued to students who have accepted the university's offer, have met English proficiency requirements and have shown proof of having funds to cover expenses of 1 year in the university.
----------
Eligibility:
The F1 visa is issued on the basis of I-20 certificate issued by the college. Dual intent is not allowed. no intent to abandon residency in the students home country.
Proof: Family ties
You should be able to present evidence of the existence of
      immediate family members as well as their ages, occupations, residences,
      and standing in the community (an affidavit from a parent should be
      adequate).
Community connections - The students participation in his community such
       as memberships in organizations, religious groups, etc.
Financial ties - Ownership of assets (especially a home) in the home
        country.
Job prospects - present evidence to show likelihood of being offered a
       good position upon returning home.  A letter from a potential employer
       may be helpful.  Also, be prepared to explain why the same education in
       the home country is not ideal.
In rare situations, the consul may request that the student post a
   Maintenance of Status and Departure Bond with the local INS office.

OF-156 Non-Immigrant Visa form
 DOS Form DS-157, Supplemental Nonimmigrant Visa Application, for all male applicants between the ages of 16 and 45
DOS Form DS-158, Contact Information and Work History for Nonimmigrant Visa Applicant
A copy of your passport which is valid for at least six months beyond the period of stay in the U.S. and with at least one blank page
    Two identical color photographs showing full face without head covering against a light background. You may wear a headdress if required by a religious order of which you are a member
student is proficient in English or engaged in English language courses leading to English   proficiency. 
     Transcripts and diplomas from previous institutions attended
Scores from standardized tests required by the educational institution such as the TOEFL, SAT, GRE or GMAT
demonstrate sufficient financial resources to study without having to work: school financial aid, personal and family funds and government assistance. [Anticipated earnings from employment during school may not be used.] if you or your sponsor is a salaried employee, you may include income tax documents and original bank statements.
the student applies at the U.S.   border for admission.
----------
Rights:
You may take a vacation at any point after an academic year is completed,  but two vacation terms may not be taken consecutively and vacation time may not be accumulated.
If you are simply changing majors, but pursuing the same degree,   no notification to INS is required. Otherwise, obtain new I-20 AB within 15 days of attendence. Foreign adviser attests, sends original copy to INS, returns a copy to you.
Any F-1 who is maintaining status may work on the schools campus for up to 20 hours/week during school and full-time during breaks. On-campus employment may not displace US workers, but no proof of this is required.
For graduate students, on-campus employment includes employment by employers off-campus which have an educational affiliation or research contract relationship with your school.
Off campus employment is still available in cases of severe economic hardship.
Employment authorization may be available under the curricular practical training program if the proposed employment is an integral or important part of curriculum. Where employment is required for a graduate degree, employment may begin right away. To apply, the foreign student adviser will mark on the back of the students copy of the I-20AB whether the authorized work is full or part time. The student then applies directly to the    local INS office for an employment authorization document (EAD) by filing    Form I765 with fee and Form I-538. You may apply for post-graduation practical training as early as 90 days prior to completion or 30 days after.  An employment offer is not required.
Pre-graduation and    post-graduation training periods are added together to determine if 12 months are exceeded. The training must be in the students educational field. Pre-graduation practical training is part time.
You may enter the U.S. on US student visa up to 90 days before the designated registration date on the Form I-20AB, and remain in the U.S. for up to 60 days following the completion of the course.

----------
Constraints:
Foreign students are permitted to stay in the U.S. for the entire period of enrollment + any period of authorized practical training + 60-day grace period to depart USA [The whole period is   normally referred to as duration of status and is noted on the I-94 as   D/S.]
Students may be permitted to pursue part-time study if the foreign student advisor recommends this for academic reasons. Graduate level 'full-time' is left to the school to define.
eligibility for post-completion practical training is lost if curricular practical training lasts more than 12 months.
A dependent spouse and children (under age 21) are eligible for an F-2 visa. F-2s may study on a full time or part time basis, but may not receive financial aid or engage in employment
----------
Extension:
if I am unable to complete my program by the time indicated on my I-20AB:
You must apply to your foreign student advisor for an F-1 extension within  30 days preceding the expiration of your I-20AB. You are eligible for an   extension if 1) your application is timely; 2) you have maintained your   status without violation; and 3) you can demonstrate that the need for an   extension is due to compelling medical or academic reasons.
if I am out of status on my F-1 : departure and re-entering using a valid F-1 visa and Form I-20 (student copy) validated by your foreign student adviser if you can convince the INS
----------
Transfer:
notify the present school
obtain the I-20 AB from the new school
complete the Student   Certification portion of the I-20AB
deliver it to the foreign student officer at the new school within 15 days of beginning attendance
 The foreign student officer will endorse the transfer on your I-20 copy and return it to you.
The foreign student officer then sends the original I-20 to the INS and a copy to the old school.
-------------------------------------
Dual intent is a United States immigration law concept. It refers to the concept that certain visitors may be temporarily present in the USA while acting in a manner that later leads to obtaining lawful permanent residence (green cards).
Persons with H-1B visas (for specialty workers), O-1 visas (for workers who have extraordinary ability), L-1 visas (for corporate transferrees), K visas (for fiancees/spouses/minor children of U.S. citizens) and V visas (for spouses/minor children of lawful permanent residents) are generally allowed dual intent.
The rule states that someone who acts in a manner to change status within 30 days is presumed to have had immigrant intent and if they act within 60 days of entry the authorities can be suspicious of the alien's fraud upon their original application.
-------------------------------------
\end{verbatim}

\subsection{OPT}
OPT status holders have to pay Federal and State taxes like H1B Visa holders but unlike them OPT people need not pay Social Security and Medicare taxes.

\section{Employment visas}
\begin{verbatim}

O-1
----------
Eligibility:
The O-1 classification is a type of employment visa under United States immigration law that applies to aliens who have extraordinary ability in the sciences, arts, education, business, or athletics which has been demonstrated by sustained national or international acclaim and who are coming temporarily to the U.S. to continue work in the area of extraordinary ability.
----------
-------------------------------------
H1B
----------
Eligibility:
The H-1B is a non-immigrant visa category provided for in the Immigration & Nationality Act, section 101(a)(15)(H) that allows American companies and universities to temporarily employ foreign workers who have the equivalent to a US Bachelor's Degree.
A specialty occupation is one that requires theoretical and practical application of a body of specialized knowledge along with at least a bachelor�s degree or its equivalent. For example, architecture, engineering, mathematics, physical sciences, social sciences, medicine and health, education, business specialties, accounting, law, theology, and the arts may be considered to be specialty occupations.
For every H-1B petition filed with the USCIS, there must be included a Labor Condition Application (LCA) certified by the U.S. Department of Labor. The LCA is designed to ensure that the wage offered to the non-immigrant worker must meet or exceed the prevailing wage in the area of employment. With PERM, labor certification processing times have been reduced to less than 90 days.
----------
Responsibilities
H-1B workers are legally required to pay the same taxes as any other US resident, including Social Security and Medicare.[2] Any person who spends more than 183 days in the US in a calendar year is a tax resident and is required to pay US taxes on their worldwide income. H-1B workers pay higher taxes than a US citizen because they are not entitled to certain deductions.
----------
Constraints:
The number of new H-1Bs issued each year in the United States is subject to an annual congressionally-mandated quota. Those beneficiaries not subject to the annual quota are those who currently hold H-1B status or have held H-1B status at some point in the past six years and have not been outside the United States for more than 365 consecutive days. In 2000, Congress permanently exempted H-1B visas going to Universities and Government Research Laboratories from the quota. (Nota Bene) The basic quota was left at 65,000 but with an additional 20,000 visas possible for foreign workers with US advanced degrees. A tiny country like Barbados has the same quota as India or China. Dependents of employment-based applicants count against the annual, per-country visa numbers cap.
H-1B aliens may only work for the petitioning U.S. employer and only in the H-1B activities described in the petition.
In the instance that the applicant loses his/her job they have to leave the country within 14 days.
Some recent state regulations prohibit H-4 visa holders from obtaining a driver's license in cases where driver's licenses are no longer being issued on Individual Taxpayer Identification Numbers alone and an SSN is required.
In theory, the maximum duration of the H-1B visa is six years (ten years for exceptional Defense Department project-related work).
There are generally two exceptions to the 6 year duration of the H-1B visa:
    If a visa holder has submitted an I-140 immigrant petition or a labor certification prior to his 5th year anniversary of having the H-1B visa, he is entitled to renew his H-1B visa in 1 year increments until a decision has been rendered on his application for permanent residence.
    If the visa holder has an approved I-140 immigrant petition, but is unable to initiate the final step of the green card process due to his priority date not being current, he may be entitled to a 3 year extension of his H-1B visa.
----------
Rights:
H-1B visa holders are allowed to bring their immediate family members (spouse and children under 21) to the United States under H4 Visa category as dependents. A H4 Visa holder may attend school, obtain a driver's license and open a bank account while in the US.
-------------------------------------
L-1 visa
----------
Eligibility:
It is a nonimmigrant visa, and is valid for a relatively short amount of time - generally two years. L-1 visas are available to employees of an international company with offices in both a home country and the United States, or which intend to open a new office in the United States while maintaining their home country interests.
Application for an L-1 visa begins with the filing of a petition with the U.S. Citizenship & Immigration Services (USCIS) on Form I-129
The L-1 visa has two subcategories: L-1A for executives and managers, and L-1B for workers with specialized knowledge. L-1A status is valid for up to seven years; L-1B status is good for up to five years.
----------
Constraints:
After the expiration of the seven or five years, respectively, the alien must leave the United States for an aggregate of 365 days.
There are two types of L-1 procedures:
    * Regular L-1 visas, which must be applied for and approved for each individual by the U.S. Citizenship and Immigration Services (USCIS); and
    * Blanket L-1 visas which are available to employers who hire large numbers of Intracompany Transferees every year.
For Canadian residents, a special L visa category is available. E-3 visas are issued to citizens of Australia under the Australia free-trade treaty. H-1B1 visas are issued to residents of Chile and Singapore under the amended NAFTA treaty.

----------
Rights:
Spouses of L-1 visa holders are allowed to work, without restriction, in the US, and the L-1 visa may legally be used as a steppingstone to the Green Card under the doctrine of dual intent.
Upon application at the consulate or embassy, the spouse and children of the primary applicant who are under the age of 21 may be issued L-2 visas.
-------------------------------------
K-1 visa
----------
Eligibility:
A K-1 visa is a United States nonimmigrant visa benefitting fianc�s and fianc�es of US citizen petitioners.
The Embassy will contact the fiancee and schedule an interview date. In the meantime, the fiancee has to have a medical examination at an approved clinic, to screen for certain infectious diseases like HIV and TB. The Embassy staff will question the fiancee and ask for additional documents.
The fianc�(e) then has six months to enter the US, and 90 days after that to marry the Petitioner.
----------
Constraints:
The average wait from initial filing to a Consular interview is 181 days.
K1 visa may not be issued to a recent student visitor (to discourage foreigners being students merely for the purpose of meeting US spouses).
\end{verbatim}
-------------------------------------
\section{Permanent Residence (Green Card)}
----------
\begin{verbatim}
Eligibility:
In the first step, USCIS approves the immigrant petition by a qualifying relative, an employer, or in rare cases such as with an investor visa, the applicant. If a sibling is applying, it must be a nuclear relative with the same parents as the applicant. Second, unless the applicant is an "immediate relative", an immigrant visa number through the State Department must be available. This number might not be immediately available even if the USCIS approves the petition because the amount of immigrant visa numbers is limited every year. There are also certain additional limitations by country. Finally, when an immigrant visa number is available, the applicant must apply with USCIS to adjust their current status to permanent resident status.
---
FAMILY SPONSORED
IR Immediate relative (spouses, minor children & parents) of US citizens. [No numerical limit]
Unmarried sons and daughters (21 years of age or older) of US citizens [23,400     6-7 years]
Spouses and minor children (under 21 year old) of lawful permanent residents [87,934 5-6 years]
Unmarried sons and daughters (21 years of age or older) of lawful permanent residents [26,266 9-10 years]
Married sons and daughters of US citizens [23,400 8-9 years]
Brothers and sisters of adult US citizens [65,000 10-11 years]

After marrying, an "Adjustment of Status" (using a USCIS form I-485[2]) must be filed that will convert the K1 fiancee and K2 children status to that of "Conditional Lawful Permanent Resident Status".
The Conditional Permanent Resident Card can be converted just before the two year anniversy of the issuance of the conditional Greencard to unconditional status by making another application and attending a second Interview. If however, the couple has split up and a history of spousal abuse can be documented to the USCIS, the fiancee alone can apply for the uncoditional status if it can be shown the separation or divorce was the fault of the Petitioner.
---
EMPLOYMENT BASED
EB1 Priority workers -- persons with extraordinary ability in sciences, arts, education, business, or athletics, or outstanding professors and researchers [40,000     Currently available].
EB2 Professionals holding advanced degrees (Ph.D., master's degree, or at least 5 years of progressive post-baccalaureate experience) or persons of exceptional ability in sciences, arts, or business[40,000     Currently available] Zhendong Lu used this route to get a green card.
EB3 Skilled workers, professionals, and other workers [40,000     4-5 years]
Certain special immigrants -- ministers, religious workers, current or former US government workers, etc. [10,000     Currently available]
Investors [10,000     Currently available]

The employer must legally 'prove' that it has a need to hire for a specific position and that there are no qualified U.S. citizen or lawful permanent resident available to fill that position. For highly skilled foreign nationals (EB1 and EB2 National Interest Waiver) and "Schedule A" labor[4] (nurses and physical therapists), this step is waived.
Immigrant visa petition. The employer applies on the alien's behalf to obtain a visa number. The application is called the I-140 form and is processed by the USCIS. Currently, this process takes up to 6 months.
After the labor certification, the alien has a choice to finalize the green card via consular processing or adjustment of status.
After the process is complete, the alien is expected to take the certified job offered by the employer to substantiate his or her immigrant status.

\end{verbatim}
---
-------------------------------------
\section{Citizenship}
\begin{verbatim}
----------
Eligibility:
--
Birth within the United States
--
Through birth abroad to two United States citizens:
   1. Both their parents were U.S. citizens at the time of their birth
   2. At least one of their parents lived in the United States prior to their birth.
--
Through birth abroad to one United States citizen
   1. One of his or her parents was a U.S. citizen at the time of the person in question's birth;
   2. The citizen parent lived at least 5 years in the United States before his or her child's birth; and
   3. At least 2 of these 5 years in the United States were after the citizen parent's 14th birthday (see note below).
--
Naturalization
one must be at least eighteen years of age
have had a status of a legal permanent resident in the United States for five years less 90 days before they apply.
this requirement is reduced to three years less 90 days if they :
    (a) acquired legal permanent resident status
    (b) have been married to and living with a citizen for the past three years
They must have been physically present for at least 30 months of 60 months prior to the date of filing their application.
if the legal permanent resident was outside of the U.S. for a continuous period of 6 months or more they are disqualified from naturalizing (certain exceptions apply )
They must be a "person of good moral character"
must pass a test on United States history and government.
Most applicants must also have a working knowledge of the English language (there are exceptions for long-resident older applicants and those with mental or physical disabilities)
Also, you have to be a   resident for 3 months in the state or INS district where you are filing the application.

----------
Responsibilities:
Citizens have the duty to serve in a jury, if selected.
Citizens are also required to pay taxes on worldwide income, including income earned while residing abroad but only beyond the first \$82,400.
U.S. citizens travel into and out of the United States on a U.S. passport, regardless of any other nationality they may possess.
Male U.S. citizens (including those living permanently abroad and/or with dual U.S./other citizenship) are required to register with the Selective Service System at age 18 for possible conscription into the armed forces.
----------
Constraints:
Naturalized U.S. citizens are not eligible to become President of the United States.
it is possible to be a U.S. national without being a U.S. citizen. A person whose only connection to the U.S. is through birth in an outlying possession. Nationals who are not citizens cannot vote or hold elected office. However, they may reside and work in the United States without restrictions and apply for citizenship under the same rules as other resident aliens.
----------
Rights:
Loss of citizenship occurs only in those cases in which an individual engaged in conduct with an intention of abandoning their citizenship. it is now virtually impossible to lose one's citizenship without expressly renouncing it before a U.S. consular officer.
\end{verbatim}
-------------------------------------
\part{Status in India}
\section{Person of Indian Origin Card}
----------
\begin{verbatim}
Eligibility:
   1. person at any time held an Indian passport or;
   2. person's parents or grand parents or great grand parents where permanently resident in India, but not moved to Pakistan and Bangladesh or;
   3. person is spouse of a citizen of India or a person of Indian origin as per above.
----------
Rights:
    * enter India without a visa.
    * join educational institutions
    * enjoy benefits similar to NRIs except the acquisition of agricultural/plantation properties.
    * (but) unable to vote.
\end{verbatim}

\end{document}
