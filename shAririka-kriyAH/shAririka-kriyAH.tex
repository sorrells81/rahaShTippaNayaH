\documentclass[oneside, article]{memoir}
\input{../../../work/packages}
\input{../../../work/packagesMemoir}
\usepackage{fontspec, xunicode}
%\setmainfont[Script=Devanagari]{Chandas}
\setmainfont[Script=Devanagari]{Kalimati}

\input{../../../work/packagesMemoir}
\usepackage{fontspec, xunicode}
%\setmainfont[Script=Devanagari]{Chandas}
\setmainfont[Script=Devanagari]{Kalimati}

\input{../../../work/packagesMemoir}
\usepackage{fontspec, xunicode}
%\setmainfont[Script=Devanagari]{Chandas}
\setmainfont[Script=Devanagari]{Kalimati}

\title{॥शारीरिकाः क्रियाः॥}
\author{विश्वासः॥}

\begin{document}
\maketitle

\tableofcontents

\part{स्वास्थ्य-रक्षणं॥}
\chapter{योगासनानि॥}
\section{परिचयं॥}
    * Contortions and breath control should not be lazy or sedate or mindless; but alert and sharp. Progress towards deep consciousness during the contortions or breath control is important to achieve.
    *
          o So, it is important to observe in great detail, with thoughts of high verbosity, the effects of each contortion on various parts of the body, during the contortions.
    * Form a sequence (Vinyasa) of contortions (Aasana), rather than isolated contortions; include Anjali hand-formation (mudraa) during transitions.
    *
          o Repeat the name of the contortion to come in the mind, during transitions. This ensures throroughness and removal of distraction in the exploration of the body.
    * Include obstinacy (Hatha) in doing difficult Aasanas. In order to stretch the limits, hold the postures for a few breaths.
    * No violent twists. Keep it gentle.
    * Heavy concentration on one (balancing) Aasana for a month is desirable in order to master it.

\section{Current contortion goals}

Increase flexibility, stretch the hamstrings:

    * Work towards Hanumanthasana.

Increase upper arm strength:

    * Work towards Adho Mukha Vrukshasana.

\section{आयोगाभ्यासं॥}

सम्यक् आसनस्थापनं।

सङ्कल्पं॥ इदानीं योगासनं करिष्यामि।
पतञ्जलिप्रार्थनं॥ योगेन चित्तस्य पदेन वाचां मलं शरीरस्य च वैदिकेन योपाकरोति तं प्रवरं मुनीनां पतञ्जलिं प्राञ्जलीरानतोस्मि॥

\section{क्रिया-श्रेणिः॥}
\subsection{Clear the nose}

    * Anuloma Viloma PrAnAyAma.

\subsection{Loosen the backbone and the muscles attached to it}

    * jaTara parivartanAsana. eka pAda Jatara ParivartanAsana. Utthita Supta pAdAngustAsana. pAshAsana. paripurna Matsyendrasana. MatsyAsana. SimhAsana 2 with eye exercises.

\subsection{Loosen the feet and the arm muscles}

    * Akarna dhanurAsana. MuulabandhAsana. KandAsana. योगदण्डासनं।

\subsection{Loosen the hamstring muscles and the lower back}

    * Krounchasana. MariichyAsana 1: forward bend. JAnu ShirsAsana. parivrutta JAnu ShirsAsana. paschimottAnAsana, parivRRitta paschimottAnAsana.

\subsection{Work on the hamstring muscles}

    * AnjaneyAsana. hanumAnAsana. upaviShTa konAsana. SamakonAsana. kurmAsana. Eka pAda ShirsAsana. स्कन्दासनं। UttAnAsana.

\subsection{Clear the respiratory tract}

    * Uddiyana Bandha and Jalandhara bandha simultaneously. Anuloma Viloma prAnAyAma.

\subsection{Work on the eyes}

Retinal muscle Asana: Track your thumb as you move it around. Ciliary muscle Asana: Focus on your thumb and then focus at a far away point. rapid blinking of eyelids. staring at a point to the point of tears.

\subsection{Work on balance, backbone and the arms}

    * VrukshAsana. GarudAsana. NatarAjAsana. Utthita TrikonAsana. Ardha ChandrAsana. parivrutta Ardha ChandrAsana. Virabhadrasana 3.

\subsection{Work on the backbone and the muscles attached to it}

    * eka pAda RAjakapotAsana 2. Supta ViirAsana; kapotAsana. 
    * ShirsAsana, parivrutta ShirsAsana; parivruttaikapAda ShirsAsana; eka pAda ShirsAsana, pArshvaikapAda shirsAsana; Urdhva padma shirsAsana; pArshva Urdhva padma shirsAsana; pindAsana in shirsAsana.

\subsection{Strengthen the arms, work on balance}

    * पिञ्चमयुरासनं. Adho Mukha Vrukshasana.

\subsection{recover from physical strain of the Asanas}

    * shavAsana.


\part{क्रिया-विशेषाः॥}
\chapter{मल-विसर्जनं॥}
प्रयत्नेन मल-विसर्जन-स्नायि-क्रिया चालयेत्। पायुना रेचकं साधयन्निव उदर-स्नायूनां‌ प्रयोगः भूयात्। क्षेत्र-अपान शैली श्रेष्ठा।

Excretion = relaxation of inner sphincter + peristalysis in the large intestine. Use the breathing rythm to stimulate peristalysis.

Even after the first pellet is excreted, check to see if there is more excreta. Sensors in the intestine and rectum indicate if there is further material to be excreted.

Drink a  glass of water to enable efficient excretion.

See forceful defecation section in health strategy.

\chapter{मूत्र-विसर्जनं॥}
यथा धेय-स्थलस्य बाहिः न क्षिप्यते विसृजितव्यं। पात-वेग-दमनाय लिङ्गाग्रचर्मस्य उपयोगः च जानु-नति भूयात्।

\chapter{स्त्री-रतिः॥}
एतत् भागं पत्न्याः विशेष-गुणान् परिगणति।

\section{अभिनयः॥}
स्त्रिया च पुरुषेण सन्तोष-अनुभवस्य अभिनयः रत्यां कौशलं च अन्ये उत्साहः वर्धयति।

\section{पूर्व-क्रिया॥}
दीर्घ-चुंबनं। सकाम-मृदु-स्पर्शः (aka sensate touch/ caress)। Spooning: dvau chamasau iva Ekasya pRRiShTataH Ekasya shayanaM, agra-sthitasya jaTarasya upari hasta-sthApanaM, anantaraM, dIrga-shvAsa-samEtaM shavAsanAt iva vishrAntiH.

vA sulabhaM yOnyAH a\~Nguli-pravEshaH.

\section{प्रवेशः॥}
निरोध-परीक्षा।

आवश्यकं चेत् प्रवेश-पूर्वं निरोध-धरणं अग्र-भाग-गृहन् वायु-रोधनं निवारयितुं।

\section{लिङ्ग-रतिः॥}
अग्रे योनि-प्रवेशः - वेग-मन्दः च मृदुः।

यथा-लंबं साध्यं लिङ्ग-रतिः, स-चुंबनं, sa-writhing।

\section{निर्गमनं॥}
योगेः बाहिरागमनं निरोधं गृहन्। शुद्धि-पत्र-आवृतस्य निरोधस्य विसर्जनं। लिङ्ग-परिसर-शोधनं।

\chapter{शरीरस्य विश्रान्तिः॥}
स्हरीरस्य विश्रान्त्यै आवश्यकः मनसि शान्तिः - अतः सा प्रथमं प्राप्तव्या। तदनन्तरं अनावश्यकान् स्नायुकर्षणान् निवारयेत्।

\chapter{उपवेशनं॥}
आसनस्य उपरि समरेख-उपवेशने जटर-स्नायूनां काय-भार-शरणाय उपयोगः भूयात्।

मानव-शरीरः दीर्घ-उपवेशनाय अनुचित-रूपः - अतः मध्ये मध्ये विश्रान्तिकरानि अटनानि कुर्यात्।

अधोपवेषणे सुखाय उपयुजेत् पद्मासनं। तथापि अवलम्बनाय भित्तिः लाभाय, तथा‌ मध्ये मध्ये आसन-परिवर्तनं अपि।

\chapter{शिर-केशः॥}
\section{लाम्ब्यं॥}
अधिक-लाम्ब्येन ग्रन्थि-जातिः अधिकः, शुद्धि-शुश्कीकरण-श्रमः अपि। परन्तु, स्हिखायै किञ्चित् लाम्ब्यं आवश्यकं। अतः शिखायाः लाम्ब्यं नियन्त्रयेत्। तत् करतलमितं वा तत्-सार्धं वा भूयात्।

ल्आम्ब्यनियन्त्रनाय कर्तयेत् शिखा-बन्धनानतरं।

\section{शिराग्र-मुण्डनं॥}
शिखा-धरणाय केश-कर्तनं। ब्राह्मण-केश-शैलि-अनुसरणं प्राथमिकम् ध्येयम्।

\subsection{धेयानि॥}

शिखा-धरणे कस्यापि केशस्य अति-कर्षणं न भूतिः, अग्रे अच पार्श्वयोः त्रिभागेषु सुदर्शणं, पार्श्वयोः मनुSःय-दर्शणे समानत्वं (असाम्ये सति अपि वस्तुतः)। 

शिरसः अग्र-भागस्य मुण्डने आकारः शिखास्थानस्य परितः चक्रवत् भवेत्। चक्र-वैशाल्यं स्वरुचिं अनुसरेत्।

\subsection{विधिः॥}
सीमा स्पष्टीकरणाय मुण्ड्य-केश-परितः प्रथमं रज्जु बन्धनं। तदन्तरं यथा-सामान्यं केश-कर्तनं खड्गेन।
अग्र-केशमुण्डनम् द्वि-अन्गुष्ट-वर्धन-अनतरम् सुलभं।

\section{शिखाधरणं।}


The lengthy hair can be tied with the help of left thumb and index fingers. You roll up the lock of hair over the left thumb and index fingers put together by your right hand till you reach the tail end. Then hold the tail end of hair by the left thumb and index fingers and pull out the fingers with the tail end of the hair. You get the knot. After some little practice you will get a tight and neat knot. [Thanks to Vembu.]

\chapter{Ergonomics while using the computer}
\section{Sitting vs standing}
Arrange your workstation so that you can switch easily between standing and sitting. Standing should be preferred when possible.

\section{Posture}
Posture is very important - you should not slouch forward. A good posture for the backbone and neck often entails proper use of wrists and shoulders; and the nerves which control the arms and wrists pass through these muscles.

Also see properties of good chairs described elsewhere to ensure comfort of the back and the feet.

\section{Shoulder comfort}
While typing or using the mouse, rest a large part of your elbow on the table. You shouldn't have to lift or twist the shoulder.

\subsection{Chair/ desk height}
Adjust the chair/ desk height, perhaps using a pillow, so that you don't have to lift the shoulder.

\subsection{Mouse position and use}
When you place the mouse to the side of the keyboard, the shoulder needs to be strained while operating it - the keyboard typing position is more relaxed. So, Comfortably position the mouse in front of the keyboard. For this, you need a wireless mouse, making an acute angle with the keyboard.

Also, become ambidextrous with mouse use: often it is better to use the mouse with your left hand, and reserve the right hand for writing and typing.

\section{Neck comfort}
The monitor should be at the level of the vision. If working with a laptop, use a laptop riser to raise the monitor to the appropriate height.

\section{Eye comfort}
Adjust font size and window size appropriately. While reading, the eye can comfortably read text if it is contained within a certain width - so don't maximize windows out of habit.

So, it makes sense to get monitors which are much taller than they are wide - this is possible by rotating some commercially available LCD displays.

\section{Wrist comfort}
Use a wireless mouse and keyboard which are very slim, so that you are not required to raise your palm at your wrist. Eg: Kensington mouse, apple keyboard. A good mouse should also be light, so that you can avoid unnecessary energy expenditure.

A good keyboard is especially slim/ low forward Eg: apple alumninum keyboard. It is not sufficient to be thin on average Eg: one Kensington wireless keyboard.

Using the touchpad on the laptop often does not give enough wrist support.

Become ambidextrous with mouse use.

\section{Breaks}
Take breaks once in a while to rest muscles involved in typing and supporting the posture.

\chapter{चलनं॥}
There should be no unnecessary tension in muscles.

\chapter{धावनं॥}
Strike the ground using the forward-ball of the foot; the heel should usually never touch the ground. Breath automatically synchronizes with movement; there should be no unnecessary tension in muscles.

मध्ये-मध्ये विश्रान्तिः भूयात् - एतत् विशेषतः पर्वत-आरोहणे आवश्यकं।

\section{पर्वत-अवरोहणं॥}
In speedy descent down a mountain, be like water flowing down - use gravity to attain controlled fall. Run down the trail, with the step growing shorter with steepness of the incline to increase grip. Keep a careful eye to put down the foot against rocks and other protrusions to maintain secure grip on footfall. Be able to decelerate gradually - or suddenly by catching branches or rocks.

\chapter{वृक्ष-पर्वतयोः आरोहणं, नदि तरणं॥}
The main idea is to be able to maintain static equilibrium at all times; but sometimes one must sacrifice static equilibrium to make a predetermined move (eg: walking across a branch). Find suitable foot-holds and hand holds - the more you find the better.

\end{document}
