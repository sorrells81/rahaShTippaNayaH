\documentclass[oneside, article]{memoir}
% \input{../../../work/packages}
\input{../../../work/packagesMemoir}
\usepackage{fontspec, xunicode}
%\setmainfont[Script=Devanagari]{Chandas}
\setmainfont[Script=Devanagari]{Kalimati}

\input{../../../work/packagesMemoir}
\usepackage{fontspec, xunicode}
%\setmainfont[Script=Devanagari]{Chandas}
\setmainfont[Script=Devanagari]{Kalimati}

\input{../../../work/packagesMemoir}
\usepackage{fontspec, xunicode}
%\setmainfont[Script=Devanagari]{Chandas}
\setmainfont[Script=Devanagari]{Kalimati}

\input{../../../work/packages}
\input{../../../work/packagesMemoir}
\usepackage{fontspec, xunicode}
%\setmainfont[Script=Devanagari]{Chandas}
\setmainfont[Script=Devanagari]{Kalimati}

\input{../../../work/packagesMemoir}
\usepackage{fontspec, xunicode}
%\setmainfont[Script=Devanagari]{Chandas}
\setmainfont[Script=Devanagari]{Kalimati}

\input{../../../work/packagesMemoir}
\usepackage{fontspec, xunicode}
%\setmainfont[Script=Devanagari]{Chandas}
\setmainfont[Script=Devanagari]{Kalimati}

\input{../../../work/packagesMemoir}
\input{../../../work/macros}

\title{Communication}
\author{}

\begin{document}
\maketitle

\part{parichaya}
For ideas about sambandha, see social strategy. For oral presentations, see career skills strategies.

\chapter{Nature and purpose}
Fundamentally, all communication is about making another person understand something which you have to say, which could be a request/ demand, an explanation etc.. .

So, in designing the delivery mechanism, one should not only consider the message to be delivered, but also the other person’s ability to receive that message.

\section{dharmAya saMvAdaH}
dharme viniyogAya cha anyeShAM sahayogAya cha saMvAdaH AvashyakaH. eShaH dheyaH samyak dharitavyaH.

dharme samyak kArya-karaNAya eva sAmAjikAH AchArAH.

Even by silently observing or by being absent, you are participating.

\chapter{Branding: enabling precise appraisal}
\section{Aim: honest appraisal}
Don't worry about impressing people too much. Just be honest, don't feel that you have to project a false image of yourself, while simultaneously avoiding erroneous perception of you by others.

Sincerity and self confidence involved in doing so are valued.

\section{Informing others of your worth}
Discreetly sell your work and worth - It is insane to expect others to do it entirely for you. "If I tell you I'm good, you would probably think I'm boasting. If I tell you I'm no good, you know I'm lying." - Bruce Lee

\section{Branding}
A powerful brand - built not on a product but on a personal message - is a competitive advantage.

Aim to truthfully project both competence and warmth.

Gravity and immersion/ mastery over an art are appreciated deeply - display this. Even though he was not especially kind or considerate, Ramanujan was very likable. He wore his heart on his sleeve. He was humble in attitude. His sincerity resulted in the magnetic personality of a secular sanyasi.

\subsection{svarUpa-saMskAraH}
Often questions are raised about various aspects of the brAhmaNa saMpradAya - especially shrIcharaNaM, yajJNOpavItaM and shikhA. It is good to have well-prepared responses.

'These are the customs of my people. They are a very important part of my cultural identity, so I value them very much. They are a critical piece of who I am. If I were to cut any of them out, it would be akin to loosing a piece of my soul.'

\subsection{dEvatA-astitvE}
anyatra-vivRRitaM.

\chapter{Biological mechanism}
Important hormones such as oxytocin and endorphins are considered elsewhere.

\subsection{hAsyaH}
Laughter releases endorphins.
 

\chapter{Areas in need of improvment}
Using names while greeting.

Accurately projecting warmth along with competence.

\part{Special situations}
\chapter{Incidental meetings}
What to do when you unexpectedly meet someone?

\section{Greeting}
If you are greeted, whether by an acquaintance or by a stranger, return the greeting.

\section{Acquaintances}
\subsection{Goal: smooth functioning}
Behaviour here should ensure smooth functioning of both you and of other animals. Example: greeting acquaintances is important for your own smooth functioning.

\subsection{Greeting protocol}
If you encounter an acquaintance whom you have not encountered many times during that day, if significant eye contact is made, greet with a smile and pronunciation of his name. Note that significant eye contact is not possible if the other animal is involved in another conversation.

If you encounter an acquaintance whom you have encountered many times during the past few hours, merely acknowledge with a nod, and perhaps a smile.

\subsection{Serendipitious conversation}
Adapt the confidential voice.

\chapter{Debate}
\section{Discipline}
\subsection{Need}
A certain discipline is essential in picking points, supporting arguments and criticizing them in a systematic manner - otherwise people simply resort making irrelevant statements and arguments (eg: erecting and taking down strawmen), making unsound arguments, starting from false premises, finally resorting to to venting/ rhetoric and no one comes closer to the truth.

One can use discipline to take control of such a debate and make it clear. Eg: bAlaga\~NgAdhara tackling sOmasuShma while debating ``The heathen in his blindness".

\subsection{Expected structure}
The discussion should have clear stages.

\subsubsection{Specifying claims}
First one specifies as precisely as possible the claims which are contested. (Often the claims being discussed are not clearly specified: it could be that both parties actually agree to the claims, but confusedly discuss and contest minutiae for a long time before realizing that they both agree.)

\subsubsection{Filtering and picking claim to contest}
Both parties should state their positions on the claims. If they both agree on a certain claim, it need not be discussed further, but may be taken as an axiom.

\subsubsection{Argument and refutation}
A party may take up the burden of proof and provide an argument supporting or refuting the claim. The opposite party then has the burden of refutation - it must find a flaw in the argument.

This process is continued until either all arguments are refuted or unrefuted arguments in favor of only one party remain.

\subsubsection{Closure and continuation}
Based on the prior step, both parties reach a consensus on the status of the claim for the purpose of the argument, which could be one of : true, false, unknown.

One then continues to the 'Filtering and picking claim to contest' stage.

\subsection{Ground rules}
Jumping from claim to claim should be curbed.

'Red herrings', presenting provocative unrelated topics in order to detract the conversation, should be avoided.

\subsubsection{External references}
Pasting entire articles is bad etiquette - it makes scrolling a pain, and it will not force anyone to read it. One should provide links if necessary with a brief description as to why it is relevant and why one would want to go read it.

\subsubsection{Role of ridicule}
I myself am happy to receive and give ridicule as a way of making a point clear.

But others may not like it. "There is no reason for anyone to be ridiculed." It is true that arguments involving males are often conducted according to (testosterone driven?) aggressive rules which females are uncomfortable with; I have seen it in many places and it is probably insensitive of us to do so. Bullying is bad.

\subsection{Fallacies in arguments}
Starting from a false premise.

\subsubsection{Addressing the wrong claim}
It is especially common to pick an easy-to-refute claim which is different from the contested claim but is superficially similar for the purpose of refutation: aka strawman.

\subsubsection{Logical error in the argument}
Common fallacies include presenting a false-dichotomy in an implication.

\subsection{Enforcement}
Acknowledge that they may be making good points, then point out errors.

\subsubsection{Insufficient empathy correction}
I don't think that you are making sufficient effort in understanding the perspective of those you are speaking to properly - otherwise you would have known that many of the arguments / references you make will have no effect on their positions at all. Maybe a good idea to empathize, talk to them rather than at them - perhaps you will learn something in the process yourself.

\subsubsection{Insufficient homework}
First, one needs to do some basic homework required to understand the basics about. 

\subsubsection{Encouragement for continuation} Systematic and clear thinking is important, everyone gets better with deliberate practice; It takes some effort, but you must believe that can do it right.

\section{bhAvanA-niyantraNaM}
tarkaH bAlEShu krIDA iva bhUyAt.

tarka-parAjaya-Ata\~NkaH bhavati. Etat-bhayaH niVarayEt tat-bhayasya vachanaiH vadanAt cha svIkArAt. antE ``satyaM Eva jayatE, nAnRRitaM'' iti vishvAsaH bhUyAt.

The truth will win in the end, and it should ideally be clear to all participants at the end of the debate.

tarka-vidhiH cha mantrAH anyatra vivRRitAH.

\section{Allowing saving of face}
The ground rules established should be such that focus is sharply on the topics being discussed, and the ground rules - this makes discussion a relatively inoffensive impersonal mechanical process.

Even when you win a debate clearly, allow for face saving by conceding that the argument was good and that it is natural to be misled.

\chapter{Negotiations}
\subitem Use the Statue strategy to project forcefulness when necessary.

\subitem It is always better to take time to think and then produce a response, than to produce a bad response.

\subitem Don't be fooled by smooth-talkers. They work mostly by making you take instantaneous decisions. Take as much time as you can while taking major decisions - such as those that involve a lot of money.

\subitem Determine the other person's motivation in interacting with you.

\subitem Negotiation is best done early: before everyone is too estranged to be rational.

\subitem Even if you are getting a free service, you have right to every courtesy due to a customer, and to the promised service in full. Be forceful but polite in demanding it. Remember that you can ask to see the boss.

\subitem Get things in writing.

\chapter{Establishing good memories in others}
People choose between memories of experiences, rather than experiences themselves. So, the ending is extremely important in any interaction. Giving gifts at the end of formal occasions is a good idea.

\chapter{Email and phone}
\section{Email protocols}
In sending keep-alive messages, try to mail individually.

While sending messages which might offend the other person, be careful and use tact.

\subsection{Addressing others}
In America, suppose that you are writing to X. The letter might begin with the string 'X,' unlike in other countries where one begins with the string 'Dear X,' or 'Hi X,'.

This is used to indicate a professional attitude stripped of compassion/ friendliness. It is used to focus the conversation on work and challenge the recipient into performing his role, with an ongoing appraisal being the incentive.

By itself, this form of addressing should not be assumed to be a sign of excessive rudeness, but as a sign of deliberate distance, used to make reasonable demands.

\subsection{Role of email}
Know when to prefer face-to-face communication: "When you communicate with a group you only know through electronic channels, it's like having functional Asperger's Syndrome`` you are very logical and rational, but emotionally brittle." 

\subsection{Awaiting Response}
If someone does not respond to an email in 48 hours, bug them.


\section{Phone calls}
\subsection{Work-related}
Turn phone calls into emails - so that you have a record of things agreed upon.

Wear formal dress - this greatly helps foster a professional attitude in the phone conversation.

Smile if you would have smiled in an in-person meeting - the smile helps connect to the other person, even across the telephone line.

\chapter{Apologizing techniques}
\subitem Focus on your direct actions. Do not make non-apologies like "I'm sorry if you were offended by my remarks".
\subitem Acknowledge the consequences of your actions.
\subitem Focus on the other party, not on yourself.
\subitem Take responsibility.
\subitem Try to rectify the error.
\subitem Offer material compensation. Rolando offered me 1/2 month's rent reduction after vexing me during the move-in.

\chapter{Technical discussions, lectures, interviews}
See 'Career skills' strategy.

\chapter{Conning and lying}
Control of body language during lying, and its detection are considered elsewhere.

Conning requires generation of trust on the part of the victim. One clever way of doing this is by stimulating oxytocin generation in the victim by appearing to place trust and regard in him, thus triggering the reciprocation mechanism. Eg: Pegion drop con.

Even simple lying exploits to a certain extant the willingness of the victim to be lied to, his eagerness to believe in something false.

\part{Real-time interaction protocols}
\chapter{Composing oneself before planned meetings}
Prepare content in advance if necessary.

\section{Calming turbulence}
\subsection{Importance}
This is especially important while meeting strangers.

Most people do not want to be burdened by an unsettled, harried individual - they are wary of getting tangled in unsavoury business.

\subsection{General technique}
Concentrate on the high order bits: the important matter. Solemnly recognize that exchange of information/ expectations is the most important point of the meeting and determinedly focus the intellect on it.

\subsection{Suppressing inopportune reactions}
It may sometimes be inopportune to feel or display laughter or anger or nervousness.

Use breath control. (PraaNaayaama)


\section{Atma-parichayaH}
pAshchAtyasha vA bhAratIyasya vA itarANAM saMskRRItInAM suguNAnAM cha durguNAnAM brAhmaNa-dharma-bhakytAH cha parichayaH bhavet vyavasthitaH, tat adhaH likhitaH.

saMskRRite cha susaMskRRityAM mayi abhimAnaH vartate. vedeSu udAttAH vichArAH dRRishyante. vaidikaH saMpradAyaH me sArthakaH, tat mama asti iti mama saubhAgyaH. tIkShNabuddhiH, Atma-niyantraNaM, dharmaH etat sarvaM tat-pAlana-phalAH. mama prajAyAH api eShaH saubhAgyaH bhavatu iti AShA.

\section{Dress strategy}
\subitem Dressing in clothing in light/bright colors will make you stand out more in a crowd and look more approachable.

\subitem Wear something comfortable, but which will project the impression of one serious in his work.

\section{Timing}
When you are overtly trying to sell something, for example your qualifications, do it only when you are adequately prepared, otherwise, you may create an unfavorable impression leading to the failure of the mission.

Do not interrupt a person who is known to dislike interruptions.


\chapter{Content}
In conversation, it is better to talk about others' work and learn, than to talk about oneself.

Ask questions that require more than a simple "yes" or "no" answer to subtly encourage a shy or introverted person to keep talking.

Conduct interviews. Make them talk about themselves and their work.

Be prepared - try not to ask questions which could have been answered by other means. Special questions:
\subitem How did you get your job?
\subitem How did you decide you wanted to pursue this specific career?
\subitem Did you have a mentor who helped you? Did you have role models?
\subitem What satisfactions do you get from your position?
\subitem Did the organizational culture turn out to be what you expected?
\subitem How were the most recently employed people hired?
\subitem How would you suggest someone with my background to look for a job?
\subitem Professors: Do you have any suggestions which may help my research or graduate studies?

Prepare answers for questions.

Compliment them - Everyone has an ego. But, don't overdo it.

If they're close, inquire about their family.

\section{Exchange stories}
Eg, in case of friends: Ask for a personal story of childhood or home, and give one in return.

\chapter{Beginning and ending}
\section{Beginning meetings}
This is especially important if you are meeting a stranger.

Greet confidently. Introduce yourself. Greet with hand, eyes, face and words. Be specific: say their name, thence give them pleasure.

\section{Ending meetings}
Do this gracefully, not awkwardly. The ending is specially important.

\chapter{Courtesy and formality}
Be courteous. Discourtesy implies not wanting friendship. Formality in politics and diplomacy was not invented without reason.

Courtesy is required for smooth operation, without misunderstandings, in the society of strangers. This is particularly useful, if help or information is later required from those strangers.

Common courtesys: Hold heavy doors open for people who are just behind you. Follow the queue system. Shake hands. Stand when others are standing.

Say please. But, don't overdo it.

Do not take anything for granted.

Include name with high frequency.

\chapter{Language, accent, voice}
Use an accent understandable by them.

Use the language closest to their native language.

\section{Voices}
\subsection{Confidential voice}
Drop defenses, smile. Adapt a calm, soft/ confidential voice.

\subsection{Commanding voice}
Adapt a deep tone, clear and easily heard voice, act as if it is your right to be obeyed.


\chapter{Body language}
\section{parichayaM}
The body language is indeed a language with words and phrases. The body can communicate even when there are no spoken words. The primary function of body language is to express emotions and status among surrounding animals; and its vocabulary is specialized for that. Using body language, one can speak multiple words simultaneously.

\section{Importance of mastering it}
Body language is widely understood, often unconsciously. Bad skill at it is often results in miscommunication. So, its vocabulary should be mastered systematically.

Just as one can speak words to oneself, one can use the body language to speak to oneself to great effect!

Realize that a significant minority does not speak it well.

\section{Vocabulary}
\subsection{Confidence}
Prolonged eye contact, relaxed but alert posture.

\subsection{Friendliness, welcoming interaction}
Smile. Nod.

\subsubsection{Depth, truthfulness}
The basic idea in understanding a smile is imitate it. Observe if the eyes are wrinkled.

\subsection{Trustworthiness, appeasement}
Mirror others' body language.

\subsection{Acknowledgement of superiority}
a\~njali-mudrA. Back slightly bent forward. Hands folded submissively.

\subsection{Doubt or mental processing}
One of the eyes narrowed at the outer corner. Raised eyebrows. Pursed lip.

Lowered eyebrows and squinted eyes illustrate an attempt at understanding what is being said or going on. It's usually skeptical.

\subsection{Authority, firmness}
Act like a statue: Your secure position does not require you to care for making others comfortable by mirroring. (Observed in bosses of corporations, mafia and emperors.)

\subsection{Boredom and distraction}
People who look away while you are talking to them are thinking about something else.

\subsection{Solitude}
People with crossed arms are closing themselves to social influence.

\subsection{Anger}
Bulged eye-balls. Flared nose. Eyes narrowed at the inner corner.

\subsection{Fear}
Arms fending unseen things away: Adjusting cufflinks, purses etc..
Casual face touching. The hand shrug: palm upward, fingers slightly curled. Squirming.

\subsection{Anxiety}
\subsubsection{Detecting lies}
Outward calm and inner turbulance. A trained person can detect lies with great accuracy. While the other is speaking, look for fear.

\subsubsection{Uncomfortable truthfulness}
When the other person is silent, these could also indicate uncomfortable truthfulness.

\section{Expert lying}
Sincerity is very important. But, in some situations, lying is inevitable. Avoid non-verbal leakage.

\subsection{Statue strategy}
\subitem Deadpan face.
\subitem Eyes concealed by glasses (Pupil dilation is involuntary, and can reveal interest level).
\subitem Rigid body, Minimal hand movement.
\subitem Seen in dominant males of many species, emperors, corporate heads. Indicates a lack of need to pander to companions.

\subsection{Flase clown strategy}
\subitem Mock-aggression or mock-friendliness.
\subitem Throw in much body language noise.
\subitem Could even be used as a permanent mask in the Orient (and Iran, I hear).

\subsection{Erratic signaling}
\subitem Deliberately misleading signalling.
\subitem Very tough to execute correctly, but very effective.

\end{document}
