\documentclass[oneside, article]{memoir}
\input{../../../work/packages}
\input{../../../work/packagesMemoir}
\usepackage{fontspec, xunicode}
%\setmainfont[Script=Devanagari]{Chandas}
\setmainfont[Script=Devanagari]{Kalimati}

\input{../../../work/packagesMemoir}
\usepackage{fontspec, xunicode}
%\setmainfont[Script=Devanagari]{Chandas}
\setmainfont[Script=Devanagari]{Kalimati}

\input{../../../work/packagesMemoir}
\usepackage{fontspec, xunicode}
%\setmainfont[Script=Devanagari]{Chandas}
\setmainfont[Script=Devanagari]{Kalimati}

\input{../../../work/packagesMemoir}
\input{../../../work/macros}
\title{॥योगः चित्त-वृत्ति-निरोध:॥}
\author{विश्वासः वासुकेयः॥}

\begin{document}
\maketitle

[वैज्ञानिक-युगोचितः उपनिशद्यत्नः।]

\tableofcontents

\part{पूर्व-वाचः॥}
\chapter{॥प्रार्थनं॥}
गुरुभ्यो नमः हरिः ॐ॥

सहनाववतु। सहनौ भुनक्तु। सहवीर्यं करवावहै। तेजस्विनावधितमस्तु। मा विद्विशावहै। ॐ शान्ति शान्ति शान्तिः॥

असतो मा सद्गमया। तमसो मा ज्योतिर्गमया। म्रुत्युर्मा अम्रुतङ्गमया। ॐ शान्तिः शान्तिः शान्तिः॥

\chapter{॥सत्यं, अनुमानं॥}
\section{Bounded rationality}
अहो! सत्यं किं इति केनापि पूर्णतया न ज्ञातुं शक्यते। परन्तु अनुमान-आगम-साक्षीणां उपयोगेन सत्यस्य समीपं गन्तुं उत्तम-प्रयत्नः शक्यते - एषः एव rationality।

\section{॥अनुमान-परिशोधनौ कुर्यात्॥}
कश्चित्सु आसन्न सूक्ष्म-विशयेशु, एतज्जीवन-नीतयः दोश-युक्ताः वा \\
वर्तमान-काले अनुचिताः। बुद्धिमन्तः अस्मिन् सत्य-ज्ञानान्-अविसृजवन्तः यथोचितं अनुमान-समीकरणं क्रवेयुः। अनुमान-आलस्यं न भवेत् कदाचिदपि।

\part{अन्तः-स्थितिः॥}
यथा कठोपनिषत्-उक्तं - 'आत्मानं रथिनं विद्धि शरीरं रथमेव तु। बुद्धिं तु सारथिं विद्धि मनः प्रग्रहमेव च ॥३॥ इन्द्रियाणि हयान् आहुर्विषयास्तेषु गोचरान्। आत्मेन्द्रिय-मनो-युक्तं भोक्तेति आहुर्मनीषिणः।'

\chapter{इन्द्रियाणि॥}
Sensory input. Described in detail in congnition survey.
इन्द्रियाणि ज्ञानेन्द्रियाणि वा कर्मेन्द्रियाणि। ज्ञानेन्द्रियैः भौतिकाः विषयाः बुद्ध्यै सूचिताः। jJNAnEndriyas include both the external organs and specialized regions of the brain which process signals from these organs.

\section{इन्द्रिय-विनोदः॥}
इन्द्रियाणि स्व-प्रीतिपात्रैः सह संपर्के विनोदं अनुभवन्ति - उदाहरणाय जनन-इर्न्द्रियः रति-स्पर्शे, जटरः पूर्णतायां, जिह्वा सुस्वादे, चर्मः सुस्पर्शे, अक्षिः दृश्य-सौन्दर्ये।

\subsection{इन्द्रिय-पीडा॥}
इन्द्रियाणि अप्रिय-सन्दर्भेषु पीडां अनुभवन्ति।


\chapter{बुद्धिः॥}
\section{निश्चयात्मिका॥}
Non-attention part of executive functions.

बुद्धिः (कल्पना-युक्ता अपि) निश्चयात्मिका चित्त-वृत्तिः। सा विषय-अवगामिनी - विशेषतः कार्य-क्रम-योजिनी, स्मृतिः च।

विवराणि animal intelligence सूत्रे।

\section{बुद्धि-वैविध्य-गुणानि॥}
\subsection{विषयाणा कालः॥}
विमर्श-विषयाणां गुणेषु मुख्यः तेषां काल-संबन्धः।

\subsubsection{भूतः॥}
भूत-काल-बुद्धिः पूर्वानुभवान् स्मरति, पूर्वान् विकारान् पुनरेव अनुभवति। स्मृतिस्थापने अनुभूताः विकाराः प्रभावशालिनः - काञ्चन घोर-विषयान् विस्मारयन्ति, काञ्चन स्मृतीन् दृढतया रक्षन्ति।

(peak, end) नियमानुसारं सङ्क्षेपेन विषयाः स्मृताः।

\subsubsection{(अ)भविष्यत्॥}
भविष्य-काल-बुद्धिः भविष्यतः च अभविष्यतः विषयान् (तस्मिन्नपि कार्यान्) योजयति, अनेकदा आतङ्कं अनुभवति।

\subsubsection{वर्तमानः च अकालः॥}
वर्तमान-काल-बुद्धिः वर्तमान-क्षणस्य समीपे स्थितान् वा अकालान् विषयान् विमर्शयति।

\subsection{विमर्श-स्थूलता॥}
विषयाः विविधासु स्थूलतासु विमर्शिताः बुद्ध्या।

\subsection{विकार-वितर्क-मात्रा॥}
एषा अन्यत्र विवृता।

\subsection{स्वकेन्द्रता।}
बुद्धि-विषयाः स्वकेन्द्राः वा परकेन्द्राः।

विशेषतः भूत-भविष्यत्-बुद्धिसु दीर्घ-स्वकेन्द्रता दृश्यते।

\section{विकाराः॥}
\subsection{भावाः॥}
बुद्ध्याः भाव-विकारेषु शान्तिः, लज्जा, भत्सरः, भयः, सन्तोषः, भक्तिः, विस्मयः, रोमहर्षः इत्यादयः। Art-सूत्रं अपि वीक्षणि।

\subsection{वितर्क-जाताः निश्चयाः॥}
संपूर्णतया विकार-जाताः निश्चयाः अनुमान-राहित्यात् 'axiom' सदृशाः भवन्ति। भावना इति विकारस्य नाम-अन्तरं।

\subsubsection{सरलता, वेगः, उपयोग-सीमा॥}
मर्शितः विषयः क्लिष्टः दीर्घः चेदपि, विकार-निश्चयाः सरलाः - एकेन एव वाक्येन कथितुं योग्याः। अतः अनेक-भाग-युक्तायाः योजनायाः स्थापने वितर्कस्य उपयोगः आवश्यकः।

विकार-प्रेरिताः निश्चयाः वितर्क-प्रेरितैः अधिक-वेगेन जाताः।


\subsection{प्रभावौ॥}
\subsubsection{वितर्क-प्रभावः॥}
विकार-जनकानां वितर्केन विमर्शनात् विकाराः प्रभाविताः। उदाहरणाय श्रिङ्गार-अभिज्ञाने वितर्क-प्रभावः अन्यत्र विवृतः।

\subsubsection{परिसर-प्रभावः॥}
परिसरस्य विकार-बुद्धीनां उपरि सूक्ष्मः प्रभावः वर्तते।

स्त्रीणां‌ वैरल्ये पुरुषाः riskier निश्चयान् कुर्वन्ति।

\section{वितर्कः॥}
\subsection{परिचयः॥}
न्यायं उपयोजयन्ती बुद्धिः‌ वितर्कः इति। वितर्कः प्रमुखेन भाषा-उपयोगेन रूपेभ्यः नाम-दानं, तथा नामानां सञ्ज्ञा-गणित-वत् (उदाहरणाय 2nd order logic) योजना स्वकार्याय। अन्यं वाक्-रहितः अतिमन्दं च क्षीणः (विकारेभ्यः एषः भिन्नः)।

तीक्ष्ण-वितर्कः क्लिष्टान् समस्यान् अपि मुख्य-विषयान् गृहित्वा, सरलीकृत्य वीक्षितुं शक्नोति।

वितर्के न्यायदोषाः communications सूत्रे तर्क-अध्याये विवृता। साधारणतः अशान्ताः विमर्शाः न्याय-दोष-युक्ताः।

\subsection{Remembering conclusions}
Biological mechanism being learning common to वितर्क and विकार (including the dopamine system) is described elsewhere. This is discussed in the संस्कार section.

\subsection{क्रमाणि च तत्-गुणानि॥}
अनेकेषु सन्दर्भेषु निश्चयने अनेकानि वितर्कक्रमाणि उपयोग अर्हाणि। ते निर्दोषत्वे च उपयोग-वेगे असमानानि।

\section{विनोद-बुद्धिः॥}
वैफल्यस्य वा दोषस्य अवगमनात् वा क्रीडायाः जातः हास्यः।

\subsection{आत्म-विनोद-बुद्धिः॥}
स्वाभिमानः स्वस्य स्थित्यां वा प्रयत्ने वा कौशले तृत्पिः/ अभिमानः।

\subsection{आत्म-पीडा-बुद्धिः॥}
लज्जा दुर्वर्तन-साक्ष्या अहङ्कारे/आत्मकथायां क्षतेः जायते।

\subsection{वर्तमत्सु नूतनत्व-प्रीतिः॥}
बुद्धिः नूतनत्व-प्रिया - brain's dopamine अङ्गं तत्-कारणः।

सा सुलभेन परिवर्तमत्सु विषयेषु -तस्मिन्नपि दैनिकेषु सामाजिकेषु सन्दर्भेषु -  सुलभेन रता भवति। एतत् एव दूरदर्शण-नाटक-प्रीत्याः कारणः।

व्यर्थेषु Random विषयेषु अपि अर्त्थं अन्वेषयति। Patients prescribed dopamines tend to fall to gambling addiction.

\section{संस्कारः॥}
बुद्धि-कौशलं संस्कारात् जातः। शिक्षणेन संस्कार-परिवर्तनेन बुद्धिः परिवर्त्यते। जीवशास्त्रीयं संस्कारक्रियाविवरणं अन्यत्र दत्तं।

\subsection{अन्यत्र-विवृताणि॥}
वितर्केन ज्ञानानुमानं अन्यत्र विवृतं (ज्ञानार्जन-सूत्रं अपि वीक्षतु); तत्-स्मरणं च reinforcement learning, habit formation अत्र विवर्णयन्ते।

\subsection{Slow learning}
When one works the brain/ body, one must remember that one is operating a powerful reinforcement learning machine, not an instantaneously mutable computer.

\chapter{मनः॥}
Aka attention. मनः सङ्कल्प-विकल्पात्मिका चित्त-वृत्तिः। सङ्कल्प-विकल्प-अनुसारं सा इन्द्रियाणि च बुद्धिं‌ योजयति।

\section{प्रेरणा॥}
मनः इन्द्रियाणां स्व-विनोद-अन्वेषणेन च विनोद-बुद्धि-अन्वेषणेन च बुद्धि-निश्चयैः (तत्-अङ्गत्वात् अहं-कारेन अपि) प्रेरितः।

मनः सन्तोष-अन्वेषणे बहु-अल्प-कालीन-दृष्टि-युक्तः। अतः बुद्धि-प्रभावेषु विकाराः बलयुक्ताः।

\subsection{परिसर-प्रभावः॥}
यथा बुद्धिः च इन्द्रियाणि परिसरेन प्रभाविताः, तथा मनः अपि। अतः एव 'out of sight is out of mind' इति ख्यातं।

\subsection{वितर्क-प्रभाव-अवसराणि॥}
मनसः वितर्केण प्रभावाय अवसरं सर्वदा नास्ति। एषः‌ अवकाशः पुनर्पुनः आगत्य गच्छति - चतुरङ्गवत् पार्यायिक-क्रीडासु चेष्टा-पर्याय-वत्।

\section{मग्नता॥}
यदा मनः एकं एव सङ्कल्पं बहु-कालं वहन् अस्ति, सा स्थितिः मग्नता इति, सः मनः एकाग्रः। मग्नता कार्येषु कौशलाय आवश्यकं, परन्तु addiction/ काम-प्रमादे अपि भागं वहति।

केषुचित् समयेषु मनः अनेकाग्रः एकाग्रता-न्यूनः - उदाहरणाय निद्रापत्त्यौ। उदाहरणाय गंभीरेषु विमर्शेषु साधारणतः एकाग्रता दृश्यते।

\subsection{समाधिः॥}
मग्नतायाः उच्छतम-स्थितिः समाधिः, यस्यां केवलं कार्य-विषयस्य प्रज्ञा अस्ति, परन्तु मग्नतायाः वा प्रयत्नस्य प्रज्ञा नास्ति। यथा योगसूत्र-उक्तं - तत्-एव-अर्थ-मात्र-निर्भासं स्वरूप-शून्यं इव समाधिः॥

\chapter{अहं-कारः॥}
\section{कथा॥}
एषः एव 'identity' बुद्ध्याः कल्पना विशेषः, पश्चात्ताप-तृप्ति-अनुभूत्यै च ध्येयवरणाय उपयुज्यमाना कथा। एषः कार्याणां कर्तुः च विकारानां अनुभोक्तुः च वितर्कान् कुर्वतः कथायां बुद्धिः। यथा बुद्ध्याः विकारः च वितर्कः इति द्वौ अङ्गौ स्तः, तथा एव अहङ्कारः अपि द्विधा विभक्तः।

\section{कथा-अङ्गानि॥}
अहङ्कारः विविधेषु क्षणेषु विविधः, तथापि तस्य कश्चित् अङ्गं स्थिरतरः। एषः दीर्घ-कालीनः स्थिरतरः अङ्गं आत्मा इति। स्व-गुण-दोष-रुचि-विरुचीनां अनुमानं self-concept इति।

\subsection{सामाजिक-पात्र-अनुमानं॥}
अहं-कारस्य सामाजिक-संबन्धैः जातः अङ्गं सामाजिक-पात्र-अनुमानं (self-image)। तस्य अङ्गेषु सांस्कृतिक-अङ्गत्वानि (cultural identities) च विविधाः कार्मिक-पात्राः (occupational roles and identities)।

प्राणिनः सामाजिक-पात्रस्य वर्तमान-स्थितिः च अन्येषां स्वविषये अभिप्रायः प्राणिना अनुमेयेते। एतत् तु अन्यैः भेद-जनकान् वेश-संस्कारान् ईषत् अवलम्ब्यते।

\section{बुद्ध्यां‌ प्रभावः॥}
बुद्ध्याः प्रामाणिकता अहं-कारस्य सत्त्वं अवलम्ब्यते। मर्शिताः विचाराः अपि अहं-कारेन ईषत् प्रभाविताः।

\subsection{वर्तन-सम्मति अपेक्षा॥}
अहङ्कारस्य कथायाः आन्तरिक-विरोध-दोष-अभावः स्पष्टता काङ्क्ष्यते। वर्तने आत्म-कथायाः समर्थनं अपेक्ष्यते। परन्तु यदा‌ वर्तनं आत्मकथायाः विरोधे (cognitive dissonance) भवति, तदा अहङ्कारस्य वर्तनेन सह सम्मतिं पुनर्स्थापितुं प्रयत्नः (justification) क्रीयते।

\subsubsection{उदाहरणानि॥}
सहाय्यप्राप्तेः मित्रप्राप्तिः अन्यत्र विवृता।

तथा एव फलानुसरणे कष्ट-अनुभवस्य अनन्तरं फल-प्रीतिः पूर्वात् अपि अधिकं वर्धते।

\section{परिसर-प्रभावः॥}
\subsection{दूर-विचार-प्रभावः॥}
एकास्मिन् विचारे बुद्धिः भिन्नैः दूर-विचारैः प्रभावितः। अहङ्कारः बुद्धि-विशेषः, इत्यतः अहङ्कारः अपि समाजपात्र-धर्मनिष्ठा-वाचकैः वेश-संस्कारादिभिः प्रभावितः।

\section{परिवर्तनं॥}
यथा अन्तः-स्थितिः परिपाल्यते, यथा कार्याणि क्रीयन्ते, तथा आत्म-कथा परिवर्तते। बुद्धि-परिवर्तनं संस्कारैः अन्यत्र विवृतं।

\part{॥पुरुषार्थाः॥}
\chapter{॥परिचयः॥}
\section{पुरुषार्थ-विषयः॥}
श्रेष्ठायै तृप्त्यै कथं जीवितव्यं जीवनं? कदा किं करणीयं? एतत् एव पुरुषार्थ-विचारः।

\subsection{॥सन्तोष-वर्गौ॥}
सन्तोषः मूलतः बुद्धि-उपयोगानुसारं द्विविधं - इन्द्रिय-विनोदः च बुद्धि-विनोदः - एतौ अन्यत्र विवृतौ। अनेकदा सन्तोषः द्वयोः वर्गयोः संमिश्रितः।

\subsection{असन्तोषः अनिवार्यः॥}
प्रति-क्षणं संतोषः वा तत्-साधनाय प्रयत्नः असाध्यः। यावत् शरीरः अस्ति च तस्मिन् इन्द्रियाणि सन्ति, तावत् असन्तोषः अनिवार्यः। संपूर्णतया दोष-रहित-धर्म-पालनं अपि असाध्यं।

\section{ज्ञान-मूलः॥}
पुरुषार्थ-प्रमाणं प्राप्तं आधुनिक-विज्ञान-दर्शणेन प्राणि-प्रेरणानुमानेन च।

\section{अङ्गानि॥}
पुरुषार्थस्य अधो-विवृताः त्रिवर्गाः धर्मः च संग्रह्य-अर्थः च कामः। एतानि विविधेषु गात्रेषु भवन्ति।

\subsection{सन्तोष-प्राप्तिः॥}
कामात् तात्क्षणिकः इन्द्रिय-विनोदः तु प्राप्यते। बुद्ध्या परिस्थित्यां तृप्तिः धर्म-अर्थ-सिद्ध्यां साध्या प्रतीक्षानन्तरं (deferred-gratification) - दीर्घ-धर्म-पालनं एव सार्थकं मनुष्याय।

\subsubsection{॥अधर्मे अपि॥}
परम-अर्थः स्वयं-धृतः। आत्मज्ञान-युक्तेन कुशलिना अधर्मिना अपि स्व-कौशले तृप्तिः साध्या अर्थ-साधनेन।

\subsection{प्रामुख्य-श्रेणिः॥}
धर्म-साधनं प्राणीनां मूल-धेयः प्राकृतिकः, अर्थ-कामौ तु परिस्थिति-स्वभाव-गणन्तौ आवश्यकौ उपधेयौ। धर्म-साधनाय च धर्म-साधन-स्क्ति-वर्धनाय अनधार्मिकं अर्थ-सङ्ग्रहणं भूयात्। कामः तु आवश्यकः दोष-सहितात् प्राणी-कार्य-विधानात् - प्राणी स्वभावतः सन्तोषं इच्छाति, पीडायाः बिभेति। अतः, यथा साध्यं, सुमात्रायां अनधार्मिका काम-प्राप्तिः भूयात्।

\subsubsection{मोक्ष-निष्प्रयोजकत्वं॥}
शास्त्र-स्तुतः धेयः मोक्षः तु जीवने जुगुप्सा-ग्रस्थेभ्यः पुनर्जन्म-भीतेभ्यः जरा-रोग-त्रस्तेभ्यः प्रियः, पुनर्जन्म-अविश्वसते मे अनावश्यकः।

\chapter{॥धर्मः॥}
\section{मनुकुल-कल्याणं॥}
मनुकुलस्य च तत्-समाजस्य कौशलाय कृताः कार्याः धर्म-वर्गीयाः। धार्मिकाणां कार्याणां लक्षणं मनुकुलाय सुमूल्यस्य अस्तित्वं।

\section{॥बीजोपदेशाः॥}
कश्चित्-रासायनेषु स्थितानि बीजोपदेशानि। बीजोपदेशेभ्यः देहसृष्टिः। \\देहस्य परमार्थः बीजोपदेशसमृद्धिः। आनक्शत्रं विस्फोटेम! नाशः न भवेत् अस्माकं बीजानां। एतद्परमार्थं धर्मं। अज्ञाने अपि प्राणिनः बहुदा एतदर्थ-साधनं इच्छन्ति। पुरुषार्थ-ज्ञानात् तु सिद्धिबलमधिकं।

यथा शास्त्रोक्तं - 'प्रजातन्तुं मा व्यवच्छेत्सीः'।

\chapter{॥संग्रह-अर्थः॥}
स्थिति-धन-विद्या-आरोग्य-आदीनां सङ्ग्रह-योग्यानां सङ्ग्रहः अर्थ-वर्गीयाः ध्येयाः।

\chapter{॥इतरार्थः कामः॥}
धर्मार्थौ विहाय अन्याः ध्येयाः काम-वर्गीयाः। ध्येय-वरणे च कार्यसाधने अनुभवव्यक्तिपरिचये Randomness-प्राप्त्यै, लीलया धर्मार्थेषु उत्साहपुनरुद्भवाय एषः आवश्यकः।

\section{सीमित-प्रामुख्यम्॥}
स्वाभाविकः इत्यतः सन्तोषाय लघुमात्रायां अनुसरितव्यः कामः। परन्तु स्वप्रयत्नाभिमानाय वा कौशलाभिमानाय काम-अनुसरणं व्यर्थं।

\section{अत्याकर्षकः॥}
बहुषु सन्दर्भेषु कामः च तत्-जातः सन्तोषः धर्म-अर्थाभ्यां अधिकः सुलभः, धर्म-अर्थ-निष्कास-प्रेरकः।

\subsection{काम-मग्नता (addiction)॥}
काम-ग्रस्तः पुनर्पुनः काम-अनुसरणात् सन्तोषं प्राप्नोति, परन्तु धर्मार्थौ, तस्मिन्नपि आरोग्यमेव निष्कासन् स्वाभिमानं (स्वस्य प्रयत्नाभिमानं कौशलाभिमानं) त्यजति।

\subsection{कारणं पूर्व-परिस्थितिः॥}
पूर्वजानां निषाद-जीवने कामिताः मिष्ठान्न-रोमाञ्चककथा-आदयः दुर्लभाः अपि धर्म-अर्थ-साधने लाभाय आसन्, अतः तत्-सन्दर्भे जातः काम-स्वभावः।

अधुना तु परिवृते परिस्थित्यां कामिताः अनेकाः सुलभाः च धर्म-अर्थ-विरोधिनः।



\end{document}
