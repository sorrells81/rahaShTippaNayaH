\documentclass[oneside, article]{memoir}
\input{../../../work/packages}
\input{../../../work/packagesMemoir}
\usepackage{fontspec, xunicode}
%\setmainfont[Script=Devanagari]{Chandas}
\setmainfont[Script=Devanagari]{Kalimati}

\input{../../../work/packagesMemoir}
\usepackage{fontspec, xunicode}
%\setmainfont[Script=Devanagari]{Chandas}
\setmainfont[Script=Devanagari]{Kalimati}

\input{../../../work/packagesMemoir}
\usepackage{fontspec, xunicode}
%\setmainfont[Script=Devanagari]{Chandas}
\setmainfont[Script=Devanagari]{Kalimati}

\input{../../../work/macros}
\title{Social and relationship strategy}
\author{vishvAsaH vAsukeyaH}

\begin{document}
\maketitle
\part{parichayaH}
\chapter{viShayaH}
This is about the correct management of sAmAjika-saMbandhas. For ideas about individual yoga in the presence of others, or for team work in the pursuit of research, or for specific communication techniques, see communication strategy.

\chapter{saMbandhAH}
\subsection{aprakaTAH dheyAH}
ApatkAle sahAyasya yAchanaM cha dAnaM tasmin.  \\
karuNA, protsAhakara-ninda api avashyakau angau saMbandha-dharme. ataH kAle kAle saMbhAShaNaM avashyakaM. 

\section{Biological Mechanism}
The biological mechanism (including various hormones) which are involved in critical social functions such as stress, well-being, trust, reciprocation are considered elsewhere. Yet, referring to them helps elucidate much.

\chapter{mUla-vyavahAra-tantraH}
\section{Understanding other animals}
What do they want? How do they react to you? See karuNA file.

\section{Communication}
For details, see communication strategy.

\subsection{Social conformity vs authenticity}
A nerd is someone who is not "cool"; they who mainly draw delight in propitiation of the intellect. "cool" people are those who derive pleasure mainly out of being liked by others.

Sincerity is more important than drama. People recognize and respect sincerity, even if the accompanying drama and other protestations are missing.

\section{Using social network websites}
\subsection{By intention}
Classified by (non mutually exclusive) intentions, use of social networking websites varies. In both cases, this serves as a  complement to (occasionally even a replacement for) physical interaction.

\subsubsection{Gaining familiarity, respect}
Use for propaganda.

This also facilitates connecting to people with whom there is an overlap.

\subsubsection{Getting information}
Classify people you are subscribed to into various categories/ circles depending on the matters on which they post. Consume news from these streams, employ appropriate filters to remove items you dislike from the stream.

Use in getting questions answered from the crowd.

\subsection{Posting}
Use something like tweetdeck to update twitter, facebook, buzz.

Use of buttons on websites like google reader etc.. facilitate easy sharing.

\part{udyoga-saMbandhAH}
\chapter{Collaboration, mentoring, command}
\section{Mentoring}
Praise effort more than achievement, teach delayed gratification, limit reprimands and use praise to stimulate curiosity.

\subsection{Reward good behavior}
Reinforce and reward behavior you want repeated. (This is used in training of both human and non human animals.) Praise is valued more than money in such situations.

\subsubsection{Praise effort, not ability}
One group was praised for their intelligence ("You must be smart at this"), while the others were praised for their effort ("You must have worked really hard"). This simple difference had a startling effect. Children who were praised for their effort were more likely to choose a harder test when given a choice, were less likely to become disheartened when given a test they were guaranteed to fail, and when finally given the original tests again, their marks improved. 

\section{sahakAra-prApti}
mitra-prApti-tantraM vIkShEta.

\chapter{Social Movements}
\section{Leadership of movements}
\subsection{Starting movements}
A leader must be willing to stand out, publicly, and risk ridicule.

\subsubsection{Use of social networking websites}
Social networks may be used in starting a movement in the spirit of the talk on "changing the world, tribes and the internet".

Rational Hindu, drew two bold members apart from myself. It later grew to over 100 people.

\subsection{The first followers}
The first followers are very important. They show others how to follow.  They transform the status of the potential leader from a crazy nut to a leader. They show others how to be followers of the movement.

The leader must treat the first followers as equals: the movement is more important than himself.

\subsection{Later followers: the 'hipness' factor}
After a tipping point, joining the movement becomes a 'hip'/ fashionable thing.

\section{Processing information and ideas}
Democratically curated news and comments are discussed in the jJAna-prApti-sUtra. Collaborative efforts at improving skills and knowledge, and solving problems are considered in the Career survey.

Such participation in a society's information and idea processing endeavor is very satisfying - whether it is done through comments in debates and discussions, or through rating others' comments or contributing to the popularity.

\chapter{Organization and movements}
\section{Dealing with a bureaucracy or an organization}
\subsection{To extract required goods from them}
Exploit the interior divisions and tensions.

In a large organization, sometimes one hand does not know what the other hand is doing.

Escalate the matter to a more powerful person in the organization.

Threaten legal action and bad publicity. Use UT Legal service. A laudromart damaged a student's pant, and upon approach to the UT legal service, a legal letter was sent to the laundromart.

\chapter{spardhA}
\section{dvesha-nirOdhaH}
smRRitiH cha pUrva-anubhava-anusAraM buddhyAH abhinatiH bhUyAt, parantu dveshaH (grudge) na bhavitavyaH.

\subsection{karuNA}
sarvE dOSha-yuktAH, parantu tasmAt shuddhaH bhavaituM avakAshaH kalpitavyaH.

\subsection{chintA-niyantraNaM}
anyEShAM viShayE vimarshEShu vartamAna-kShaNa-saMbandhaH spaShTaH bhUyAt.

\part{saparivAra-udyoga-saMbandhAH}
\chapter{Mate}
Obsolete, after finding shruti. See sahayoga-tantra.

% \section{Mate finding}
% \subsection{Factors affecting mate quality}
% \subsubsection{Marrying kin}
% Because parental work is energetically costly, and kinship generally favors cooperation, one possible explanation for kin preference in breeding in this species is that it offers a benefit by facilitating parental cooperation. \href{http://www.sciencedaily.com/releases/2007/02/070205132431.htm}{[Ref]}
% 
% Women born between 1800 and 1824 who mated with a third cousin had significantly more children and grandchildren (4.04 and 9.17, respectively) than women who hooked up with someone no closer than an eighth cousin (3.34 and 7.31). Those proportions held up among women born more than a century later when couples were, on average, having fewer children. Despite the general pattern for reproductive success favoring close kinship, couples that were second cousins or more closely related did not have as many children. \href{http://www.sciam.com/article.cfm?id=when-incest-is-best-kissi}{[Ref]}
% 
% \subsubsection{Cultural quality}
% \paragraph*{Divorce rates and lack of seriousness}
% \begin{itemize}
% \item US - 50\%, Canada: 48\%, Australia: 40\%, Japan: 27\%. Singapore 10\%. India - 1.1\%. \href{http://www.divorcerate.org/divorce-rate-in-india.html}{[Ref]}
% 
% \item According to enrichment journal on the divorce rate in America:
%             The divorce rate in America for first marriage is 41\%.
%             The divorce rate in America for second marriage is 60\%.
%             The divorce rate in America for third marriage is 73\%.
% 
% \item At least 10\% of Australian and Canadian children are conceived in an affair. Some DNA studies put it as high as 15\%. In 2003 more than 3,000 DNA paternity tests were commissioned by Australian men, and in almost a quarter of those cases, the test revealed that 'their' child had been fathered by someone else. In 30\% of paternity tests by the American Association of Blood Banks the father was not the true biological parent. \href{http://www.peterfox.com.au/fidelity_4.html}{[Ref]}
% \end{itemize}
% 
% 
% \subsection{Target identification}
% \subsubsection{Target pool}
% \begin{itemize}
% \item A summary of the target profile: In line with parenting objectives, find a healthy and fertile female who is either:
% \subitem A Bright brAhmaNa survival machine. Likely location: Pool of PhD students.
% \subitem Someone embued with strong brAhmaNa culture with a bias towards intellectual activities, who is pliable to logic and persuation. Likely location: Children of accomplished Vedic scholars or highly regarded priests/ vaadyaars.
% \item Expectations from a marriage amongst females:
% \subitem A huge fraction (around 95\% or 1 in 20) expects frequent attention, particularly inquiries about welfare and activities like going out and visiting places. This population is frequently vain, highly social, non-individual, and are without a grand focus for their life. Examples related by others: S expected calls inquiring about performance in exams etc..
% \subitem A small fraction, such as a couple of Vidya's friends and like many woman PhD students, are highly individual/ strong willed/ able to opt for deferred gratification rather than instant pleasure, less vain, highly pliable to logic, less suceptible to whims, possess stronger interest in intellectual activities, purer and honest in character, and endowed with sufficient social skills.
% \end{itemize}
% 
% \subsubsection{Checks and probes}
% \begin{itemize}
% \item Gauge health and fertility.
% \item Detect compatibility with private discussion:
% \subitem Be prepared. In the interests of decency, don't eat anything in their place - insist on it.
% \subitem Do not disqualify a candidate based on whether she decorates herself for the occasion. Evidence indicates that propensity to decorate themselves is wired into women.
% \subitem Talk for 2 hours - If you can't determine the suitability of a person in 2 hours of talking, you cannot determine it in 20 months of talking.
% \subitem Set the agenda and time.
% \subitem Talk about yourself being a Bright brAhmaNa mathematician survival machine, by explaining the ideas behind those terms, rather than label itself. Explain your values: that you will often scorn money to explore nature and mathematics; your admiration for the high bias towards intellectual activities that can be naturally induced by strongly inculcating the brAhmaNa culture in children.
% \subitem Explain your tastes: Sanskrit. Disinterest in movies and shopping for style, but without objection to her limited indulgence in these activities.
% \subitem Ask her to similarly introduce herself.
% \subitem Relate your ideas about family goals, family size, parenting goals (including the importance of career relative to family), the requirement for certain discipline (No Salsa lessons, no meat, no liqour, no tobacco).
%           o
% \subsubitem So as to not sound like a jerk, make the point about Salsa tactfully, by contrasting it with Ballet, Bharatanatyam and other dance forms which do not involve embracing strangers.
% \subsubitem Discuss location ideas, career goals, potential uncertainties, backup plans. Discuss work habits, schedule constraints.
% \subitem Solicit her ideas about the those topics.
% \subitem Ask subsidiary questions:
%           o
% \subsubitem Ask about important formative influences.
% \subsubitem To determine if she is a good survival machine, tactfully ask her to relate some challenges she faced and how she overcame them.
% \subsubitem Check if she is Bright: Ask her about her thoughts about Science, skepticism and Mathematics.
% \subsubitem Check if she values being a modern brAhmaNa.
%                 +
%                       \# No Salsa lessons, no meat, no liqour, no tobacco.
%                       \# Ask other questions, including those about her career.
% \subsubitem Solicit any questions she may have.
% \subsubitem Introduce your family.
% \subsubitem Ask her to introduce her family.
% \subsubitem If suitable, make the best pitch. But don't declare the decision immediately, go home and discuss.
%                 +
%                       \# Praise the intelligence of women, as they are more used to being praised for their beauty.
% \end{itemize}
% 
% \section{Winning the mate}
% \begin{itemize}
% \item Factors which affect females' choice:
% \subitem Among pair-bonding species like humans, in which males and females stay together to raise their children, females also prefer to mate with big and tall males because they can provide better physical protection against predators and other males. \href{http://www.sciencedaily.com/releases/2007/02/070205132431.htm}{[Ref]}
% \subitem A recent study has found a strong correlation between a woman's choice of a partner and her relationship with her father.
% \subitem The levels of their testosterone surged to the same extent whether they were talking to an attractive woman or someone they may not fancy at all. They found a testosterone increase after only five minutes of exposure to a woman. With the increase in testosterone levels males tend to display more dominant behaviour. They talk more with their hands, there is more eye contact, their posture is more upright, and they are more likely to tell stories designed to impress the woman. \href{http://timesofindia.indiatimes.com/HealthSci/Men_just_tuned_to_lust_Study/articleshow/3256582.cms}{Ref}
% \end{itemize}
% 
% \section{Keeping the mate}
% \begin{itemize}
% \item Beware of long distance romantic relationships. Consider the cases of Raja V and Chinmoy D.
% \end{itemize}



\chapter{Parenting, descendents}
\section{Adolescents}
An analysis of motivations and behavior of adolescents and children is done elsewhere. Thence we know that their brain is not fully developed, yet it is highly flexible. So, it is predisposed to seek new sensations, accept risks more easily and prefer peer company. Here we consider ways of dealing with them.

'Adolescents want to learn primarily, but not entirely, from their friends. Studies show that when parents engage and guide their teens with a light but steady hand, staying connected but allowing independence, their kids generally do much better in life.'

\section{Abilities of descendents}
\subsection{Arrival of Homo evolutis}
Tissue and genetic engineering, and robotics will change the world. Inevitably, Homo Sapiens will be replaced by Homo Evolutis. With better brains, better bodies and better interface with machines, man will be able to solve harder problems better. There will be inter-species conflicts and resource conflicts. Animals who will adapt quickly to take advantage of the change, yet retain relevant cultural values, will thrive.

\section{Desirable ideas, habits, skills}
\subsection{Meta-cognition}
know the weaknesses of will-power, intuition, logical deliberation; when and how they can be manipulated. This is easily trainable, but the older the kid is, the tougher this is. etat bhAratIya-saMskRRiteH shaktiH.

\subsubsection{Delayed gratification}
Ability to delay gratification, rather than being easily lured by instant gratification. This includes repression or circumlocution of biological urges in favor of lasting and fundamentally important rewards.

\subsection{Metaphysics}
Ideas about the origin of life. These ideas should consonate with the bod of evidence and scientific knowledge.

\subitem They should be educated about scientific theories, their strong evidence, and their superiority over ancient religious beliefs.

\subsection{Epistemology}
\subsubsection{Desire for the truth}
Skepticism, aversion towards nonsense, ability to think critically and rigorously.

\subitem Proper training in correct reasoning, the scientific method and mathematics is essential.

\subsubsection{Research Aptitude and desire}
Desire to advance man's knowledge of and power over nature and mathematics.

\subsection{Shrewdness in society}
Diligent parents accomplish this by furnishing social opportunities and instructing about customs, appearance and dress. Importance of reputation should be communicated by words and by example.

\subsubsection{Understanding of distinction among cultures}
An understanding of the differences in values and support for one's objectives among various communities, and the ability to choose and exploit one's membership in the right communities.
\subitem Parents should point out the distinctive values and resources of such communities, along with the nature and examples of their importance.
\subitem Important values provided in part by a community are inculcated in part due to positive stereotypes associated with it, partly due to the example of community members and partly due to its customs and traditions.
\subitem The parents themselves should try to be members of the right communities.

\subsubsection{Social companionship}
Children should not have to join alien religious communities, with their dangerous and irrational ideas, just for the sake of social support and marital opportunities.

\subitem Instead, knowing their own nominal religion inside out, they should be strongly aware of the falsehood of the ancient religious ideas, and their danger in swaying an animal and its descendents away from its purpose.

\subsection{Good values}
\subsubsection{Hard and clever work}
Parents who provide examples of and who exemplify hard and clever work accomplish this.

\subsubsection{Procreative desire and ability}
Desire to procreate and extremely strong commitment to raise young children in a strong family.
\subitem Besides the parents' indoctrination, values acquired from one's clan and tribe are important.

\subsection{Hygene, chastity and disgust at intoxication}
\subitem Besides the parents' indoctrination, values acquired from one's clan and tribe are important.

\subsection{A template parenting strategy}
\subitem Parents' and community's examples and implicit/ explicit contrast with other strategies is important.
\subitem Provided by brAhmaNa community.

\subsection{Maintenance of fruitful membership in the brAhmaNa community}
The perpetuation of the relevant and noble aspects of the brAhmaNa culture requires some discipline, knowledge and commitments. My personal list is below.

This is implemented by picking and choosing values and customs of the brAhmaNa  culture one finds deeply meaningful and relevant. Thus, one can proudly stick to the ancient symbols of tribal affiliation - the Sricharana, the sacred thread, the shikhA, vegetarianism, saMskRRita, the excellent spirit of inquiry from the Upanishads; while denying the false ideas held as true by the ancestors.

\subsubsection{Discipline}
(AyAma, niyama, anuShtAnam):
\begin{itemize}
\item Acquire mastery (including advanced degrees) over a field of study.
\item Avoid yettchil. Maintain strict hygene rather than ritual cleanliness.
\item Stay vegetarian (perhaps eggitarian), unless you are in places (eg: Near the arctic circle) where nutrition from meat becomes essential for good health.
\item Don't smoke or drink alcohol or imbibe other intoxicants.
\item Look at the customs document for a list of customs followed.
\item Very optional: When possible, perform Upakarmam, Thiruvidyanam/ Shraadhham in order to socialize and to stay in touch with the rituals.
\item Be comfortable with worship of idols/ specific forms of Gods in temples. With proper प्रज्ञा: Remember that you are worshipping the सुगुणानि of the deity.
\item सुसंस्कृति-युक्तायाः सह विवाहं।
\end{itemize}

\subsubsection{Honoring susaMskRRitas}
Especially archakas and purOhitas.

daivakai\~NkaryaiH vedayOgAdinAM rakShaNaM bhavati. athaH gauravaM arhanti archakAH cha purOhitAH.

\subsubsection{Negative aspects}
Misunderstanding can lead to serious flaws. One might wrongly value being brAhmaNa by birth rather than by actions.

Make a strict separation between loyalty and pride towards the noble aspects of the brAhmaNa  culture on one hand and loyalty and pride towards the irrational and ignoble aspects on the other. One does not imply the other. Thus, it becomes suddenly possible (even easy) to give the young the benefit of a noble heritage, while deleting the burden of irrational ideas.

\section{Kindness}
Kids carry into adulthood memories of kindness and excessive unkindness. These then affect their adult and adolescent behavior, especially towards parents.

Unkindness and negative feedback should be used as necessary to mold behavior; but one should rely more on positive feedback and praise.

\section{Happiness of parents}
People with children have on average been shown to 'feel less happiness', but that they have more peak events (happiness and sadness); and it does not include abstract feelings of satisfaction.

Obsessive parenting (described elsewhere) is not only ineffective in yielding many of the expected results, but it also negatively affects the happiness of parents.

\section{Developing the child's skills}
\subsection{Nature vs nurture}
Obsessive parenting (trucking children to museums, forcing sports, compulsory music lessons) has been shown to be mostly futile.

\subsubsection{Nature}
Intelligence, educational achievement, even salary beyond young adulthood seem to be independent of parenting style.

\subsubsection{Nurture}
However, in activities like pursuing social interactions and in the formation of habits (vegetarianism, smoking, drinking) - and thereby in the formation of community/ social identity, parenting are very effective. For example, especially in early childhood and especially for socially disadvantaged families, preschool has been shown to reliably increase social skills required for professional success.

\subsection{Risk measurement}
With any activity (even walking in a sidewalk), One is always taking a risk with one's child (even its life).

\subsection{High expectations?}
Have high expectations from children, but don't have narrow expectations; if you have low expectations, children will easily sink to it. Encourage good things freely. This is unconfirmed - considering the huge role nature plays in professional achievement levels.

\subsection{Questioning method}
Aka Socratic teaching. Ask (possibly leading) questions, try not to provide answers; perhaps propose answers.

This is useful in the presence of rebellious feeling.

\subsection{Activities and projects}
These include: Games, stories and projects. Though these are probably futile as a way of developing the intellect- nature is far more dominant; they may be useful in developing interests, and in the ability to function in society.

\subsection{Community identities}
Importance of cultural membership in forming the values/ default behavior of human animals is described in the societies survey.

\subsubsection{Important memberships}
\begin{itemize}
\item Intelligentsia in the society; those that dream and execute grand dreams and delight in their intellectual life.
\item The brAhmaNa community.
\item The academia and the research community.
\item Indian diaspora.
\item Permanent residence in a developed country with the important communities (desired).
\end{itemize}

\subsubsection{Influential but undesirable Cultures}
Western materialism.

\subsubsection{Need for faithful replication of culture}
The extinction of Charvakas teaches us that sensible ideas and habits should be as zealously propogated as superstitious ideas and habits often are.

Children acquire their values and habits from their parents, parents' religious and secular ideas, clan, tribe and larger society. Young kids have an evolutionarily benefitial propensity to believe in their parents.

Parental nutrition and the values of the culture in which the child is reared plays a greater role than specific parenting in the child's development. They also have access to resources specific to their communities.

Adolescents are the true benefactors of social pressure - they are highly influencible by their surroundings - especially peers.


\subsubsection{Location}
See the location strategy.

\subsection{Occupational identity}
This should depend strongly on the child's interests.

\chapter{Culture/ community participation}
Also see chapter on participation in movements. \tbc

\part{maitra-saMbandhAH}
\chapter{dharmaH cha prAptiH}
\section{dharmaH}
ApatkAle sahAyyaH. jJNAnasya cha sUchanAnAM cha protsAhanasya vinimayaH.

\subsection{Regional variations in the maitra-dharma}
In many countries, friendships involve formality, reciprocity and obligations. They require some time of formality before they are forged.

\subsubsection{Activity partners in USA}
\subitem USA is a mobile culture, and friendship is regarded as a "survival aid". Friends in USA are often merely "activity partners".

\subitem Most friendships in USA do not carry any obligations whatsoever. There is no compulsion, responsibility or reciprocity. Hence, they are easily made and dissolved. Hence, they do not visit/ call at inconvenient times. However, Americans may have an inner core of friends among whom obligations and reciprocity exist.

\section{mitra-prAptiH}
In order to acquire friends, be one. Help make others successful.

Acquisition of friends takes time. This cannot be helped. It is important to be seen and greeted for a significant stretch of time before trust can be acquired.


\subsection{Identification of potential friends}
\subitem Look for earnestness, rather than hipness.

\subsubsection{Conscious people in america}
Despite defects in belief, susceptibility to fallacies, they are commune in places such as SRF, Chinmaya mission etc.. They are yogis, they frequent places like whole foods organic store, REI, Mother's cafe vegetarian resturant etc.. Good way to explore their circle is through websites of these places.


\subsection{Others' criteria}
People mainly look for 2 qualities: warmth and competence. Sometimes, initially, detecting the presence of one, they may think that the other is absent.

\subsection{Meet new people}
\subsubsection{kAryAlaye}
\subitem Do not eat alone.
\subitem Coffee time for socializing in Europe.

\subsubsection{gRRihe}
\subitem Invite people for talks, meals and visits. Ask to lunch with them. Pretend that you will be in their geographical area if they are important.

\subsubsection{Social network websites}
anyatra vivRRitaM.

\subsubsection{Informational interviewing}
Do a lot of informational interviewing.

People like talking about what they do. This is especially true if they know that you're not after a job with their organization. It is called "information interviewing". There will be a significant percentage of successful interviews.

\subsubsection{maitri-saMsthAH satsa\~NghAH}
vamshena cha kulena sadRRiShAH saMsthAH vartante. taiH kAle kAle melanena, cha sadasyANAM kvachana viShayeShu mahamateH vamsha-kula-vat snehaH sAdhyate.

udAharaNAya kraisteShu church, hinduShu RSS/ HSS sa\~NghaH, brAhmaNa-sabhA, saMskRRita-bhAratI.

\subsection{sahAyya-prAptEH maitraM}
Aka Ben Franklin effect

Franklin set out to turn his hater into a fan, but he wanted to do it without “paying any servile respect to him.” Franklin’s reputation as a book collector and library founder gave him a reputation as a man of discerning literary tastes, so Franklin sent a letter to the hater asking if he could borrow a selection from the his library, one which was a “very scarce and curious book.” The rival, flattered, sent it right away. Franklin sent it back a week later with a thank you note. Mission accomplished.

The next time the legislature met, the man approached Franklin and spoke to him in person for the first time. Franklin said the hater “ever after manifested a readiness to serve me on all occasions, so that we became great friends, and our friendship continued to his death.”


\chapter{vyUhAH}
\section{Decay of saMbandhas}
rakta-saMbandhas do not decay over time. Unlike maitra-saMbandhas, they can withstand abuse, are very resilient.

\subsection{sambandha-rakShaNAya kAle kAle saMbhAShaNaM}
Others do decay, with the lack of frequent face to face communication. Social networking websites reduce the rate of this decay.

tasmin bhavatu infromational interviewing cha kaushala-prashnAH.

Like sindhu, keep in contact with people by targeting/ arranging to meet (perhaps for lunch), 1 person each week.

\subsection{Limitations on size of saMbandhis}
This investment of time and energy limits the size of saMbandhas of various sorts. 150 seems to be the natural limit on number of people a human tracks.

\section{maithunaM}
vishvAsaH, shrutiH.

\section{nikaTa-parivAraH}
mAtA, vidyA, pitA, rAghavaH, anaghA.

\section{dvitIya-vyUhaH}
mAtAmahaH. mAtAnujA, mAtAnujaH, teShAM parivAraH.

mahimA-parivAraH. prItiH.

\subsection{dainika-vyavahAra-mitrAH}
indrajita-gurukulaM.

parAgaH.

\section{tRRitIya-vyUhaH}
They with whom you share a history, who can be relied upon. Maintain contact with emails, phone calls.

saMjayaH, shrutiH jAgadIshA. pitRRi-sahodarAH.

vvshs: manjunAthaH.

ssec: ajaya-sharmA. guruprasAdaH, tasya parivAraH. shrI-harShaH.

infosys: manoj.

microsoft: rohiNI.

iiSc: manjunAthaH, AnandaH kude, vijayaH, bAlAji.

utexas: hRRiShikeshaH.

\section{chaturtaH-vyUhaH}
Use social networking websites.

UT miscellaneous.

saMskRRita-bhAratI-mitrAH. 

anyAH rakta-saMbandhinaH.

\section{Developing and maintaining sambandha}
Use social network websites.

Greet them on festivals and birthdays.



\chapter{paristhitayaH}
\section{Disrespect}
\subsection{Cumulative vs selective}
Animals have good or bad regard for various behavioral attributes (and theorized mental attitudes) in other animals. For each animal, a weighted summation of these reveals itself as a respect function towards a specific other animal. This function results in positive or negative emotions/ behavior being communicated to that animal. This section mainly concerns itself with this cumulative, rather than selective, regard of one animal towards another.

\subsection{Detection}
Disrespect could be detected through body language, a verbal or written response or the lack of it, lack of reciprocation or other action/ inaction.

\subsubsection{Errors}
One may misread an action as a sign of disrespect when none was intended. This implies that the mappings of actions to regard in the two animals does not match.

\subsection{Identify cause, detection errors}
The cause could be one's prior offense, or it could simply be the other animal's social/ inter-personal respect-ranking algorithm.

\subsubsection{Simulation method}
Consider the root disrespectful action. Simulate the animal's thinking.

\subsubsection{Questioning}
In case the situation is ambiguous, obtain either a clarification by direct contact.

\subsection{Response}
Response depends on the cause of the observation, presence of detection errors and on whether you wish to modify the other animal's social/ personal respect algorithm (perhaps acquire good regard for oneself or one's ilk in the future). If the answer to the latter is negative, no response is required. 

\subsubsection{Detection error case}
Future match in the action-regard mappings may be desired. Even so, modifying one's own action-regard mapping may suffice.

If the other animal's action-regard mapping is to be modified: Make it perfectly clear that you are offended - through direct messages and hints. Recognition of disrespect is made obvious, in hopes of triggering a self-correction mechanism in the other party - which may include either an explicit or an implicit apology.

\subsubsection{Prior offense case}
One's actions which caused prior offense to the other animal may have been misread - in that case explain the mismatch in action-regard mappings and resolve to remove this mismatch in the future. This could possibly include an apology for the lack of prior match in the action-regard mappings. Direct, clear communication is best in this case.

Otherwise, if you recognize the offensive action as a product of erroneous habits or models, apologize and resolve to fix it in the future.

\subsubsection{Algorithmic disrespect case}
At times, you may not be able to abide the knowledge of  disrespect towards you ingrained in another's algorithm. 

\subitem Contact is minimized.
\subitem Politeness when contacted is continued.
\subitem Do not go out of your way to commit a favor.
\subitem Non-begging is enforced.

\paragraph{Reversal}
The policy is reversed only when consistent evidence of respect is observed again. Opportunistic, temporary, false show of respect does not count.

This usually involves waiting until one's circumstances change to elicit positive output from the offensive algorithm; or until the other animal's respect-algorithm mutates for better.

\section{Spurning}
\subsection{Not disrespect}
This is not the same as disrespect, though it is often accompanied by disrespect. It indicates a desire for the imposition of a limit in certain type(s) of social interaction. Thence the extant of availability/ involvement of an animal may be determined.

\subsection{Symptoms}
It is observed mainly through non reciprocation, or by creation of unpleasentness to dissuade future approaches.

\subsection{Response}
ati-parichayAt avaJNA. So spurn spurners in equal measure.


\section{Visits/ overnight stays}
\subsection{Short duration visits}
See social dining strategy.

\subsection{Purpose}
Visits should be made, in order to maintain familiarity with kin and friends despite geographic distance.

It also provides exposure to unfamiliar professions, values and ways of life.

\section{Entertainment of visitors}
\subsection{Places to visit}
\subsubsection{San Antonio}
river-walk, sea world, ripley's museum, devasthAna.

\subsubsection{Austin}
Simply use yelp on iphone.

\paragraph*{Scenic places}
Mozart's cafe, Mt Bonnel, Oasis cafe, digambara-beach on lake travis. stream in UT, stream in golf course.

zilker park/ lady bird lake, zilker park botanical gardens, the science and natural history museum nearby.

pedernales falls state park. mc kinney state park.

\paragraph*{Intellectual places}
dAkShiNAtya devasthAna. Capitol, UT shows, UT. austin zoo.

\paragraph*{Museums near UT}
Dinosaur museum. LBJ museum: about president LBJ, successor of kennedy. Center for American History - Briscoe center usually has exhibits. Blanton museum of art.

\paragraph*{sattvika-AhAra-shAlAH}
Good organic/ vegetarian places to eat out in ausTin: Mr. Natural, Mother's Cafe, Kerby Lane, beats cafe.

\subsection{Other areas around austin}
Temple devasthAnaM.

\subsubsection{Houston}
San Padre sea shore.


\section{Social dining/ short visits}
\subsection{Wait for everyone's food}
Do not start eating until everyone's food has arrived. If others' food has arrived, but your's has not, let them begin eating, lest their food go cold.

\subsection{Avoidance of display of greed}
This is not merely a matter of satisfying hunger. There is the constraint of having to exhibit social manners, avoiding any exhibition of greed in eating: not doing so signals lower social status.

Must not fully fill the plate to the brim; instead come back for refills. Must not keep others waiting in the queue.

\subsection{Gifting}
In USA, it is customary to carry a gift (flowers, wines and candy) when visiting a family for dinner.

\subsection{Leaving}
Leave before you are asked to leave.

\chapter{Dogs}
\section{Relationship with humans}
\subsection{Human perspective}
\subsubsection{Use}
Companionship, guarding, various specialized roles (guiding blind people, herding sheep).

\subsubsection{Cuteness}
Cute faces are liked because they trigger the release of oxytocin.

\subsection{Dog's perspective}
Dog's relationship to their owner is ideally one of a pack-member to the pack-leader, combined with some playfulness.

\section{Communication}
Dogs may be trained to recognize many words, nouns and gestures. Dog barks and gestures can be deciphered by humans to some extant: wanting food etc..

Yet, people wrongly impose human emotions on dogs by mistake - like ascribing regret (deep cogitation in general) in to submissive gestures (simple emotions in general).

Eye-contact or baring-teeth by strangers is viewed as a challenge. Growling and baring-teeth is a sign of a challenge from the dog's side. A wagging tail is a sign of playfulness. A tucked-in or a lowered tail is a sign of submission.

\section{History}
Dogs joined early human packs - their sense of smell was useful, and they were deemed useful for hunting, guarding. First, natural selection as part of this symbiosis selected animals which were more friendly towards humans. This selection was replicated artificially in case of foxes in Siberia, where cute friendly foxes emerging within 30 years as a result: the development of these foxes seems to be arrested in playful adolescence. 

Then, artificial selection led to creation of breeds with various behavioral (Eg: hunting, herding sheep, spotting prey) and physical characteristics.

\end{document}
