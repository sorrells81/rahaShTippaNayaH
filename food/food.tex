\documentclass[oneside, article]{memoir}
\input{../../../work/packages}
\input{../../../work/packagesMemoir}
\usepackage{fontspec, xunicode}
%\setmainfont[Script=Devanagari]{Chandas}
\setmainfont[Script=Devanagari]{Kalimati}

\input{../../../work/packagesMemoir}
\usepackage{fontspec, xunicode}
%\setmainfont[Script=Devanagari]{Chandas}
\setmainfont[Script=Devanagari]{Kalimati}

\input{../../../work/packagesMemoir}
\usepackage{fontspec, xunicode}
%\setmainfont[Script=Devanagari]{Chandas}
\setmainfont[Script=Devanagari]{Kalimati}

\input{../../../work/packagesMemoir}
\title{Food and nutrition}
\author{vishvAs}

\begin{document}
\maketitle

\part{parichayaM}
\chapter{Nutrients and sources}
\section{Nutrients}
\subsection{Fiber}
Fiber can be soluble and insoluble. Insoluble fibers aid bowel movement.

Soluble fiber helps reduce cholesterol, reduces sugar response after eating, normalizes blood lipid levels. Soluble fiber is not broken down until it reaches the large intestine where digestion causes gas (flatulence). In excessive amounts, soluble fiber causes high flatulance, cramps.


Fibers in legumes, such as peanuts, beans is mostly soluble fibers.

Tree Nuts have insoluble fibers.

\subsection{Omega 3 fatty acid}
Vegetarian sources of omega 3 fats: flaxseeds and flaxseed oil, walnuts, hemp seeds, leafy green vegetables. Unfortunately, however, most naturally occurring vegetarian sources of omega 3 fatty acids are not very rich in DHA (a very important type of omega 3 fat). How much DHA is recommended per day? 100-300mg/day [Ref]

\subsection{Phytochemicals}
Recent studies indicate that high intake of antioxidant vitamins (vitamins C, E and beta-carotene) does not appear to provide significant protection against Alzheimer's disease. It is Phytochemicals in fruits or vegetables that provide the real benefit. The skins of fruits and vegetables are particularly rich in these phytochemicals. Recent studies have shown that polyphenols (like resveratrol in wine) extend maximum lifespan by 59 percent and delay age-dependent decay of cognitive performance in animal models.

\section{Malnutrients}
\subsection{Sugar}
One must avoid rapid rises and fall in sugar levels in most circumstances - except after exercise, when it is used up by muscles. Reasons: According to  \href{http://nutritiondiva.quickanddirtytips.com/how-sugar-affects-your-body.aspx}{link}: suppression of immune system and growth hormone secretion, promotion of inflammation, glycation, and insulin-resistence (by promoting fat livers even in lean people according to \href{http://www.nytimes.com/2011/04/17/magazine/mag-17Sugar-t.html?_r=1}{link} ).

Further complications include: higher blood pressure, higher trglyceride levels, lower HDL cholesterol levels, higher LDL cholesterol levels (aka metabolic syndrome), thence leading to both type 2 diabetes and heart disease (surprise!). 

There are also observations leading to speculation about its role in the increased incidence of cancer with western diet/ lifestyle: insulin promotes tumor growth!

\subsection{Trans-fats}
\subsubsection{Structure}
Unrefined fats include double bonds in the carbon chain, which lead to a molecular structure which is not a straight chain. Upon hydrogenation, single bonds are formed in place of double bonds, with a strong preference for the formation of trans-hydrocarbon backbone rather than cis-bonds. Such molecules have a more linear structure.

\subsubsection{Effects}
Trans facts tend to accumulate and eventually harden in blood-vessels, particularly in the heart and the brain, leading to heart disease and higher blood pressure.

\subsection{Lead}
Affects cognition.

After several reports of lead poisoning in Indian children in the Boston area were linked to consumption of Indian spices. \href {http://www.time.com/time/health/article/0,8599,1971906,00.html}{Reference}

\section{Calories}
Around 2240.7 calories per day required to maintain 125 lb weight in 68 inch body.

\chapter{Food components and sources}
\section{General preferences}
\subsection{Organic food}
Organic fruits are usually grown without the use of pesticides. In the case of apples, the difference in taste was apparent, and the white deposits near the stalk was absent.

Apples/ peaches etc, berries, grapes are especially vulnerable to high pesticide use. Beware of organic produce which claim to use organic pesticides - they tend to be deadlier than inordganic pesticides!

\subsection{Ideal variety}
Based on a 2,000 calorie diet, you should consume fats sparingly (less than 3 servings of high fat foods per day) and at least:
    
    * 5 servings of fruits and vegetables
    * 4 or more servings of whole-grain foods
    * 3 (2-3 ounce) servings of high-quality protein (eggs, beans, nuts, seeds, or tofu)
    * 3 servings of low fat milk or dairy products

\subsection{Low glycemic index food}
Avoid high  glycemic index food in order to stay satiated for longer. 

Prefer intake of food with high portions of soluble fiber. Note that this is different from food high in insoluble fibers.  [\href{http://www.medicinenet.com/script/main/art.asp?articlekey=56527&page=2}{Ref}] .

Fats tend to ensure longer satiety too.

\section{Grains}
Need staple food source of carbohydrates. Brown rice. Oats (lowers bad cholesterol). Whole wheat (eg: Couscous, bread).

\subsection{Pseudograins}
These are not from the grass family. The following are high in proteins and nutritious: Quinoa, Buckwheat groats (not related to wheats). 

\section{Fruits}
\subsection{Fruit juices}
Avoid prune, plum, apple juices as they have caused diarrhoea in the past.

Odowalla superfood, carrot juice, orange juice are good.

Avoid fruit juices in general, as they provide many calories, without other good things like pulp and fiber.

\subsection{Long lasting fruits}
Cucumber. Capsicum.

\subsection{Pesticides, ripeners, preservatives}
Mango and sapoTa in bhArata are heavily sprayed with artificial ripeners and insecticides.

In USA, Mango is usually clean.

\section{Oil and fat}
\subsection{Olive oil}
Consider olive oil, to avoid hydrogenated oil. Buy high quality (extra virgin) olive oil. Beware that there is much mislabeling. Do not buy extra-light olive oil (it is refined, clear and bland).

\section{Vegetables and fungi}
Buy frozen, pre-cut, pre-cleaned vegetables; or dried vegetables (available in chinese stores) for easy cooking without compromising on nutrition.

\section{Milk products}
\subsection{Artificial Hormone avoidance}
Avoid yogurt from cows injected with growth harmones, or with gelatin in them.

\subsection{Whey}
Isolate is lactose free, while concentrate is not; but the latter has more immune-system boosters.

\section{Spices}
Vertain Indian spice powders (even in USA) are contaminated with lead (see elsewhere). Prefer seeds and leaves which you can easily grind instead.

\chapter{Meal choices}
Choice/ basic content of food is considered elsewhere. Yet, prepared food contains various food components and (mal)nutrients due to their ingredients and cooking procedures.

Be conscious of the nutrition and calories in what you eat. 

\section{Be wary of taste: Stomach over brain over tongue}
Taste was evolved to identify nutrient rich food; but in the modern world, the tongue is easily fooled by cleverly mixed chemicals. In this high calorie environment, it is not wise to trust whims of the tongue.

\section{Ingredient awareness}
\subsection{Vegetarianism}
In accordance with culture and current knowledge of healthy nutrition, stay ovo-lacto vegetarian. Avoid food cooked with same cutting boards and knives as fresh meat.

\subsubsection{Caution while eating out}
Thai food almost universally uses fish sauce.

Double check in restaurants of mAmsAhAri cultures: ajJNAnAt mAmsa-bhakShaNaM abhavat brazilian resturant madhye. default mAmsaH AsIt.

\subsection{Avoid poisons}
Malnutrients considered elsewhere should be avoided.

\subsection{Adverse reaction causants}
Beware of food which cause flatulance or diarrhoea or gastro-intestinal irritation (see health info for a list of known malefactors).

\section{Cooking awareness}
Avoid fried food. Do not heat plastic containers in microwave.

\section{Dish awareness}
corelle vitrelle glass, glass are fine. corelle stoneware/ china are not acceptable: possible lead content.

\section{Eating pattern in USA}
\subsection{Breakfast}
Relatively high calorie food at 8:30 or 9. Couscous is easy to prepare.

\subsection{Lunch}
At 12:30 or 12.

Eat out with friends, for social benefits.

Or Oats with Yoghurt, with peeled carrots. Around 607 + 80 = 687 calories. This is chosen because of ease of storage in office.

Or oats with whey protein concentrate and water.

\subsection{Snacks}
Generally at around 1600, or whenever in need. 120 to 220 calories.

Fruits. But don't eat too much insoluble fiber.

Oats with yogurt pr whey.

\subsection{Dinner}
Low calorie food. 2 eggs with vegetables and/ or frozen fruits. Observe ekaadashiphalaahaaraM. 74*2=148 calories.

\section{Quantity awareness}
ati-bhakShaNAt prAptA nidrAvasthA.

Never aim to say 'I am full' after eating; instead aim to say: 'I won't be hungry for 4 hours.'. Do not eat to relieve boredom or to take a break. Let it not be the dominant source of Ananda - use buddhi instead to gain Ananda from dhArmika-kAryas!

Example: Okinawans stop eating before they are 80\% full: they say this to themselves before eating.

\subsection{adhikAnnasya visarjanaM}
adhikaM kretaM vA sRRijitaM api visarjanIyaM annaM bhaviShye khAditaM rakShaNaM kaShTAya chet.

viralaM anna-visarjanaM nAsti annasya avamAnaM.

\subsection{Eating out}
Most restaurants and food venues give more than enough food for one meal. Don't eat it all at once. When eating out, take home leftovers and have 2 meals for the price of 1.

\section{Good resturaunts}
Visit supermarkets with the specific intent of eating fruits.

\subsection{Social dining}
See elsewhere.

\part{Cooking}
\chapter{Procedures}
\section{Freezing}
Store all foods at 0 F or lower to retain vitamin content, color, flavor and texture. Freezing preserves cooked food for long durations.

\subsection{Defrosting}
Never defrost foods in a garage, basement, car, dishwasher or plastic garbage bag; out on the kitchen counter, outdoors or on the porch. There are three safe ways to defrost food: in the refrigerator, in cold water, or in the microwave. [Ref]
Refreezing:

It is safe to refreeze, provided you did not introduce or allow the multiplication of bacteria during thawing. So, during thawing, it shouldn't have sat between40°F–140°F (4°C–60°C) for more than two hours. [ Ref] It is however safer to cook the food and then freeze it.

\section{Signs of cooking}

There is almost always a change in color. Cooked vegetables are easier to chew and cut. Cooked vegetables sometime change color slightly. White matter may turn brownish. Cooked onion turns somewhat transparent but brownish. Peas loose water - their skin becomes wrinkled.

Boiling generally means production of good sized bubbles even in the middle of the liquid, and not just on the borders.

\section{Cooking X in pressure cooker with 1:N water with M whistles}
\subsection{Recipe}
Clean lid and gasket of the pressure cooker. In a container pour X and water in 1:N ratio. Close lid with gasket. Turn on the flame. Wait for steady steam to appear. Place the weight on the steam egress. Wait for the cooker to whistle M times. If cooker does not whistle M times in a period of 10 minutes after placing the weight, something might be wrong - Turn off the flame at once.Turn off the flame, and let the cooker cool down for 15 minutes. Remove weight.

\section{Tadka}
This is essentially about preparing spices before cooking. Take a vessel. Put 3 spoons of oil. Add 1/2 spoon cumin seeds (which are small, brownish and elongated). Add some tamarind powder. Add 0.5 palmfuls of mustard. Add a pinch of asafoetida. Wait for the cumin seeds to sputter. Now, the tadka is ready.

\section{Fermentation}
Leave item (like suitably prepared batter or milk) outside overnight or until signs of fermentation are seen - porousness in case of batter, thickness in case of milk.

\section{Rehydration of dried East Asian food}
Add water to dried food and boil.

\chapter{Yogurt}
Heat milk well, until it is slightly more than luke-warm. Mix a few spoons of old yogurt to supply necessary bacteria. Ferment. Refrigerate for preservation.

\chapter{Grains}
\section{Multi grain, whole grain Bread}
Can be eaten with yogurt. Maybe toasted with a toaster.

\section{Kali (Tamil) or Mudde (Kannada)}
Flours of these grains have been tried so far: Ragi (Millet flour), wheat, soya, red rice.
Recipe:

1 tumbler water. 1 pinch salt. 2 spoons oil. boil. rapidly add enough flour to produce a non viscous paste. stir vigorously to ensure that all parts of the flour have had contact with the boiled water. Wash preparation vessel at once. Eat.

Err on the side of adding less flour/ more viscous paste: raw flour is unedible.

\section{Rice}
Clean rice thrice. Cook rice in pressure cooker with a suitable ratio (1:1.5 in USA and 1:3 in India) of water with 2 whistles. Eat.

\section{Instant idli/ pancake}
\subsection{Batter preparation}
Mix flour (possibly of various kinds, maybe whole-grain) with water, some salt, a spoon of oil, then add powdered pulses/ dAl or egg for protein and connectivity. Batter should be very viscous - add water gradually while stirring to ensure that too much is not added.

If egg is not used, optionally leave it out and let yeast ferment it: It will become porous and increase in volume. Possibly add vegetables and goji berry to it.

\subsection{Cooking}
There are two ways of cooking it:

1] Spread it over an oiled tavA over a stove and wait for solidification. (untried)

2] Put it in a small microwave bowl and let microwave until no more bubbles are formed due to evaporating and escaping water (around 10 minutes). Using a plate instead of a bowl is not advisable because the volume in the center of the plate is heated much more than the peripheri.

\subsection{Pseudograins}
Quinoa is boiled in 1:3 ratio till it changes appearance to a softer larger sprouted-like form.

Buckwheat is cooked similarly.

\chapter{Egg}
\section{Egg in microwave}
\subsection{Recipe}
Pour contents of two eggs in a microwave; add available vegetables; cover tightly with plastic sheet; microwave for 1 or 2 minutes, depending on power of microwave.

Remember to ensure that the egg is solid, and fully cooked.
Oats and cereals:
Can be eaten with yogurt or milk, with or without heating.

\chapter{Soups and vegetables}
\section{Soups}
\subsection{Dal}
Wash Dal once. (In USA, washing Dal is usually not necessary.) Cook Dal in pressure cooker with a suitable ratio (1:3 in USA) of water with 3 whistles. Eat.

\subsection{Dal, the liquid}
Prepare Dal (2.25 cups for 4 people) in a cooker. Cut a couple of onions into fine pieces. Cook the onions. Prepare the tadka. Add the cooked onions. Add 4.5 spoons of tomato puree. Add .5 spoons of chilly powder. Add enough water to cover the contents of the vessel. You now have a paste. Add the cooked dal. Boil. Eat.

\subsection{Rasam or Chatamd or Soup}
Prepare Dal. Put it in vessel. Pour 4 or 5 tumblers of water. Add 3 spoons of salt. Add 4.5 spoons of rasam powder (which is composed of chilly powder, tamarind powder, dal powder, coriander seeds, pepper, jeera, mustard, asafoetida, bengal gram). Add tomato if it is available. Boil. Drink.

If quick use powder is available:  Pour 1 tumbler of water. Add a spoon of soup powder and some salt. Boil.

\subsection{Nalagri or Sambar}
Bake vegetables. Follow the same procedure as that for the preparation of Rasam, but add extra salt.

\subsection{Chocolate drink}
Heat water, add some honey and some chocolate powder; possibly also some lime juice, stir. Very comforting in cold weather.

\section{Vegetables}
This includes mushrooms.

If using frozen vegetables, use knife to good effect to ensure that it fits fully within the vessel; No additional water required.

If using fresh vegetables, add quarter cup of water. Add an assortment of spices and olive oil, mix. Microwave for a suitable period of time. (10 minutes in India for most vegetables. 5 minutes for onion and capsicum in India.) Big pieces require more time. Err on the side of overcooking, in the beginning.

\subsection{Curry}
Bake vegetables. Take a vessel. If vegetables are too dry, add quarter tumbler of water. Add equal proportions of salt and rasam powder based on quantity. (Less for spinachs.) Boil. Eat.

\subsection{Curry - fried}
Bake vegetables. Prepare the tadka in a vessel. Add a palmful of urad dal in that vessel. Watch it change color. Add the cooked vegetables. Spread around salt (1 spoon for 1.2 packets of frozen vegetables) and mix very well. Spread around chilly powder (.5 spoon for 1.2 packets of frozen vegetables). Eat.

\end{document}
