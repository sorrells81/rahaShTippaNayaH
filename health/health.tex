\documentclass[oneside, article]{memoir}
\input{../../../work/packages}
\input{../../../work/packagesMemoir}
\usepackage{fontspec, xunicode}
%\setmainfont[Script=Devanagari]{Chandas}
\setmainfont[Script=Devanagari]{Kalimati}

\input{../../../work/packagesMemoir}
\usepackage{fontspec, xunicode}
%\setmainfont[Script=Devanagari]{Chandas}
\setmainfont[Script=Devanagari]{Kalimati}

\input{../../../work/packagesMemoir}
\usepackage{fontspec, xunicode}
%\setmainfont[Script=Devanagari]{Chandas}
\setmainfont[Script=Devanagari]{Kalimati}

\input{../../../work/packagesMemoir}
\input{../../../work/macros}
\title{Health strategy}
\author{vishvAs}

\begin{document}
\maketitle

\part{Strategies}
\chapter{Health strategy}
\section{Longevity}
Avg life capacity: 90 yrs.  Sardinian Highlands, Okinawa, 7th day adventists enjoy high life expectancy.
 
 Okinawans, Sardinian highlanders, adventists enjoy low-intensity workouts due to their lifestyle - they don't specially exercise.
 
 They enjoy the benefits of close knit social circles of family and loved ones first, then about 4 to 6 friends throughout the life, then the tribe with which common easily identifiable values. The avg american used to have 3 good friends, now that number is down to 1.5.

\section{Problem response}
Always be prepared for common illnesses. Have important medicine. Have phone number of the nurse-line handy.

\subsection{Strong focus on recovery}
When illness attacks, attempt to stop all work, take rest; otherwise the stressed body will be open to other illnesses. Don't even venture out for amusement.

\subsection{Attitude}
When disease attacks, an effective fight must be orchestrated. Information and a fighting-attitude is essential.

During disease, one may feel: 'Life has no purpose. It is not worth living in this pain.' But, this feeling about the nullity of life is a result of a body too sick to feel its natural urges, and of a mind which is feeling helpless. So, Never feel helpless and remember that mental capacity/ mode is affected by disease.

\subsection{Techniques}
Modern medicine must be fully used. Learn lessons about causes and cures of illness.

Also use ancient and effective techniques to alleviate symptoms. For example, deep breathing relieves pain and sedates.

\subsection{Important resources}
bcbstx.com.

UHS 24-Hour Nurse Advice Line (512) 475-NURS (6877). Or 1-866-412-8795 opt2. Condition management: opt3. Lifestyle mgmt: opt1.

Blue Care advisor: Registered nurse Janet Mwinamo. M1-866-412-8795 - ext 29070 M-F 0830-1700CST.

\subsection{Unnecessary/ wrong treatments}
The wrong belief among patients that more care implies lower sickness, cognitive defects in doctors who don't rely on research, the fear of law-suits and prospects of more income (in countries like USA) also lead to these useless treatments.


\part{Information}
\chapter{Physical data}
\subsection{Places to buy or test}
Walmart Vision center: (512) 491-9707, 45\$. UT Health insurance does not cover polycarbonate lens.

Titan eye + or baLEpETE, be\~NgaLuru.

Online: [\href{http://www.allaboutvision.com/buysmart/eyeglasses.htm}{Ref}] [\href{http://www.43folders.com/2007/11/29/adventures-40-eyeglasses}{Ref}] [\href{http://glassyeyes.blogspot.com/}{Ref}]
[\href{http://www.bettervisionbetterprices.com/giveaway.html}{Ref}]

googles4u: Can get frame for 10, polycarbonate for 33\$, photocromatic glass for 59 - only for glass lenses, but in current specs (as of 2010), they seem to have missed it despite my order. You get a discount if you go through glassyeyes.

29dollarglasses: polycarbonate transition glasses: 40 + 50.

\subsection{Properties of good eyeglasses}
For sunglasses details, see clothing strategy.

\subsubsection{Lens}
Polycarbonate lenses are thinner and lighter than traditional plastic eyeglass lenses. They also offer 100 percent ultraviolet (UV) protection and are up to 10 times more impact-resistant than regular plastic lenses. Fewer scratches too. no UV treatment needed, no scratch resistant coating needed.

No need to prefer thinner and lighter high-index lens materials for current needs: 1.5 refractive index enough. Edges that are thicker than their centers.

Antireflective coating: No halos around lights. Polarization: eliminate glare from horizontal surfaces: not needed usually.

Photochromatic lens: Darken to sunglass shade when exposed to sunlight. Transitions lenses: variable-tint technology, rapid darkening, come in high-index polycarbonate, expensive.

\subsubsection{Frame}

You can also get a headache if you're not looking through the optical centers of the lenses. Frames that are too large or that don't fit the bridge of your nose properly can slip.

Full rim frame. Spring hinges.

If playing sports, use sports frame to withstand impact.

Material: Plastic is easier to break. Titanium, beryllium and stainless steel are lightweight, strong and corrosion-resistant.

Note disadvantage of frames of small height, when having to look at both blackboard and book.

\chapter{Brain health}
\section{Genetics}
Some special genes can predispose you or increase resistance to diseases, including cognitive ones like alzheimers. As explained in the human society survey, genes play a limited role in general in the heritability of intellect.

\section{Blood supply to the brain}
Fatty build up in blood vessels is linked with brain impairment, and deadly strokes and heart attacks. The brain relies on a good blood supply to keep its functions and processes in top order.

\subsection{Hydration}
Hierarchical regression models demonstrated that lower hydration status was related to slowed psychomotor processing speed and poorer attention/memory performance, after controlling for demographic variables and blood pressure. 

\subsubsection{Negative agents}
Drinking heavy amounts of alcohol shrinks your brain.

\subsection{HDL levels}
One important determinant of blood vessel health is cholesterol levels, with the LDL subtype of cholesterol being bad for your arteries and HDL being good.

Many dietary items which increase HDL naturally include.

\subsubsection{Negative agents}
Smoking reduces HDL.

\paragraph*{Trans fats}
Trans fatty acids (commonly termed trans fats) are a type of unsaturated fat. Most trans fats consumed today, however, are industrially created as a side effect of partial hydrogenation of plant oils. Partial hydrogenation changes a fat's molecular structure (raising its melting point and reducing rancidity), but this process also results in a portion of the changed fat becoming trans fat. In trans fat molecules, the double bonds between carbon atoms (characteristic of all unsaturated fats) are in the trans rather than the cis configuration, resulting in a straighter, rather than a kinked shape. As a result, trans fats are less fluid and have a higher melting point than the corresponding cis fats. The primary health risk identified for trans fat consumption is an elevated risk of coronary heart disease (CHD).  [Ref] In other words, it blocks blood vessels, including those which plumb the brain.

\subsection{Capillary damage}
\subsubsection{Oxidative damage}
Since the brain uses about 25\% of the body's oxygen, the brain is highly susceptible to oxidative damage.

\subsubsection{Diabetes}
Diabetes damages the small blood vessels in the brain, and eventually rots these vessels to the point where they entirely close off. When this happens, the brain tissue fed by the blood vessel dies (i.e. a stroke). The diabetic brain therefore frequently looks like Swiss cheese, with lots of little holes scattered all over the place.

\subsubsection{High blood pressure}
Also damages the plumbing.

\section{Nutrition}
See food strategy.

\section{Activities: Stimulation and rest}
\subsection{Physical activity}
Physical activity may be beneficial to cognition during early and middle periods of the human lifespan and may continue to protect against age-related loss of cognitive function during older adulthood. The tasks, which measured subjects' reaction time and response accuracy when presented with congruent and incongruent visual patterns, involve cognitive processes known as executive control function (ECF).

Exercise causes the frontal lobes to increase in size. But other regions benefit from exercise in many secondary ways. "Wherever you have the birth of new brain cells, you have the birth of new capillaries."

\subsection{Cognitive exercise}
The Religious Orders Study, which began in 1993 and includes more than 1,000 nuns, priests and brothers across the country, has found that those who engage more often in reading, puzzles and processing information have a 47 percent lower risk of Alzheimer's disease than those who do little or none.

As Begley points out, many scientists now pooh-pooh the "use-it-or-lose-it" theory of mental functioning. Instead, they argue that it is "cognitive reserve" built up largely before the age of 30, not ongoing mental training, that benefits aging adults.

\subsection{Negative agents}
Games such as Mahjjong epilipsy.

\subsubsection{Prolonged stress}
The more hours you put in at work, the more likely you are to have high blood pressure. High Blood Pressure ravages the small blood vessels that feed the brain, and over time leads to many little holes in your gray and white matter that are quite obvious on MRI brain scans. [Ref]

In medical students studying for exams, the medial prefrontal cortex shrinks during cram sessions but grows back after a month off. [Ref]

\subsubsection{Sleep deprivation}
Some aspects of memory consolidation only happen with more than six hours of sleep. [Ref]

Teenagers who stay up late on school nights and make up for it by sleeping late on weekends are more likely to perform poorly in the classroom. This is because, on weekends, they are waking up at a time that is later than their internal body clock expects. [Ref] Sleep debt can be repaid, though it won't happen in one extended snooze marathon. Tacking on an extra hour or two of sleep a night is the way to catch up. [Ref]

By depriving rats of sleep for 72 hours, the researchers found that those animals consequently had increased amounts of the stress hormone corticosterone, and produced significantly fewer new brain cells in the hippocampus. [Ref]



\section{Physical damages}
About an American football player: He committed suicide November 2006 at the age of 44. The results of his brain autopsy have just been announced, and the pathologist from the University of Pittsburgh concluded that his brain cells had the appearance of an 85-year-old man with Alzheimer's disease.

They found brain damage in virtually every Everest climber but also in many climbers of lesser peaks who returned unaware that they had injured their brain.

On shockwave injuries: If the skull doesn't break, sometimes this can lead to the energy of the impact being more fully absorbed by the brain, often leading to shearing and tearing of the white matter pathways as the brain 'bounces around' inside.

\section{Aging effects}
In general, memory declines with age. But, for a few people, it remains clear and functional. [National Geographic, Nov 2007]

\subsection{High spread}
Claude E Shannon died of Alzheimer's disease. It's estimated that 5-10\% of the population aged 65 years or older has dementia.

\subsection{White matter decline}
White matter naturally degrades as we age—causing disrupted communication between brain regions and memory deficits—after conducting a battery of cognitive tests and brain scans on 93 healthy volunteers, ages 18 to 93. [Ref] The observed weakening of these brain network interactions (associated with default state and executive functions) was significantly correlated with a measured decline of cognitive functions as a result of aging. [Ref] 

\part{Health problems}

\chapter{The head}
\section{Heavy head}
\subsection{Symptoms}
Head feels heavy, stuffy. Attention is impaired.

\subsection{Cause}
Exposure of head to cold air or cold water.

\subsection{Solution}
Go to sleep at once. A nap suffices sometimes. But may last entire day. Even then, should be gone in 2 nights sleep.

Use paracetamol (known as acetaminophen in the USA) if necessary.

\section{Migraine}
(*possibly* a type of vascular headache)
Symptoms:
blur patches in the vision field (Fully reversible, one sided, "visual aura", a mostly neurological event), inability to read,
developing to a blur ring around vision field (tunnel vision),
lasting for around ten minutes, ending in nausea and slight headache

\subsection{Cause}

    * stress
    * bright light
    * skipped/ irritating meals
    * unnatural use of the brain (noticed on two occasions) in trying to induce an artificial, memorized "state of the mind" (eg: simulation of quick thoughts by bringing up quick, moving images in the mind)

\subsection{Solution}
Go to sleep at once.
Use paracetamol (known as acetaminophen in the USA) or vasograin (includes paracetamol and caffeine) if necessary.


\section{Dry eyes}
Symptoms: Dryness of the eye

\subsection{Cause}
Strain to the eyes, generally because of reading too much.

\subsection{Solution}
rest eyes.

\section{Eye strain}
\subsection{Symptoms}
No burning sensation; just a feeling of lack of strength in muscles required to rotate the eyeball.

Greater social irritability, lack of concentration.

\subsection{Causes}
\paragraph*{Strained eye muscles}
Straining of some eye muscle; especially the ciliary muscle involved in focusing. Switching between distances rapidly can hasten the strain as well.

Staring at a monitor for a long time also causes it.

\paragraph*{Environmental factors}
A flickering wireless mouse cursor is suspected to have once played a part.

High contrast also causes eye strain. (Was once using low contrast in linux installations for this reason.)

\paragraph*{Stress and fatigue}
Stress, in part: so can be psycho-somatic. 

\subsection{Solution}
\paragraph*{Immediate relief}
Deep breathing. A nap. A walk.

\paragraph*{Use of eye as a physical activity}
Research or reading or programming should rightly be considered a physical activity as it involves heavy use of the eye muscles, wrist, shoulder, fingers - there is often over-emphasis on the mental aspect of the job. 

\paragraph*{Fixing the environment}
Use a low contrast background; make the working area evenly lit. Remove glare.

\paragraph*{Developing and maintaining strength of eye muscles}
Use eye exercises: eye yoga.

\paragraph*{Giving breaks to the eyes}
Every 20 minutes, stare at a point 20 feet away for 20 seconds, giving 20 seconds for the transition in focus.




\section{Stiff neck}
\subsection{Cause}
Stress or response to cold or cold drafts of air.

\subsection{Solution}
Take a walk.

Deliberately learn to relax your muscles when cold.

Take care of stress.

\subsection{History}


\chapter{Respiratory system}
\section{Nasal congestion, rhinorhoea}
\subsection{Causes}
This could be a symptom of allergy or infection (common cold or flu).

\subsection{Solution}
Ingest lot of fluids, warm food. Also treat cause.

Topical decongestants should only be used by patients for a maximum of 3 days in a row, because rebound congestion may occur in the form of rhinitis medicamentosa. [Reference]

Use prANAyAMa and mantras with many anunAsikas.

\subsubsection{Decongestants: vasoconstrictor}
Use vaso-constrictors, like Pseudoephedrine, oxymetazoline (afrin spray).

\subsubsection{Decongestants: Steroids}
Nasal steroids reduce inflammation in the nasal passages and are better than oral antihistamines at relieving most nasal symptoms, including a blocked nose. Nasal steroids have to be used regularly to be effective. They are best started a couple of weeks before the pollen season begins.

Use fluticasone (eg Flixonase allergy nasal spray) : a spray for each nostril twice daily.

\subsubsection{Decongestants: water based}
Saline spray: seems surprisingly effective.

\subsection{Complication: acidity}
Swollowed mucus stimulates the stomach. If food is not ingested, acidity follows.


\section{Dry cough/ itchy throat}
\subsection{Symptoms}
Dry cough has no purpose and the irritation on the chest is more nuisance than benefit. Can lead to disturbed sleep.

\subsection{Cause}
Post-nasal drip. Daily, one swollows much mucus. In case of post nasal drip, mucus production is excessive.

\subsection{Solution}
Drink much water to wash down mucus. Nasal irrigation. Sleep with many pillows. Breathing slowly and deeply reduces dry cough.

Honey.

Benadryl.

\subsubsection{Local anasthetics}
Menthol tablets: Halls, Vicks etc.. were very effective during a flu attack in Feb 2011, especially when they were held in the mouth for long while sleeping.

Chloraseptic spray.

\section{Common Cold}
Symptoms: Yellow mucus indicating infected sinuses.

\subsection{Cause}
Your nose runs during a cold or the flu to get rid of unfriendly bacterial or viral invaders.

\subsection{Solution}
Rest. Relieve congestion.

vitamin C, zinc (esp before going to bed) will reduce duration of symptoms. The latter promotes the immune system.

Zinc gluconate and zinc acetate have shown the greatest anti-viral effectiveness as lozenges.

\section{Flu/ Viral fever}
Symptoms: Rhinitis, possible bronchitis, Cough which is initially dry, Fevers which come and go, chills.

\subsection{Cause}
Viral infection.

\subsection{Solution}
Rest.

Get symptomatic relief for fever, dry cough, congestion etc..

\subsubsection{Prevention: Vaccination}
Get flu-shots: even though they are ineffective if the strains in the vaccine do not match the ones in circulation, one cannot tell in advance if this is the case. When they are well matched, they are very effective.

\subsubsection{Prevent contamination}
While caring for the sick, take care not to exchange bodily fluids, wash hands often, have good air circulation/ maybe wear mask, sleep separately etc..

\subsection{History}


\chapter{Wide spread infection}
\section{Fever}
Symptoms:
\subsection{Cause}
Viral or bacterial infection. People with lower immunity are especially suceptible: HB fell ill after walking in the  rain.

\subsection{Solution}
Rest

Ingest Temperature reducers and painkillers like paracetemol.

If bacterial, get antibiotics, ingest complete dosage to avoid creation of superbugs

\section{Chicken pox}
\subsection{History}


\chapter{Gastro-intestinal tract}
\section{Stomach ache / cramps}
\subsection{Symptoms}
(Possibly) bubbling sounds in stomach, flatulence, burping, stomach pain / burning, vomit of digestive acids and food.

\subsection{Cause}
    * Stomach releases digestive acid - only to find that there is nothing to digest. Trigger could be Skipping food, abnormal stimulation of stomach.
    
    * possibly Indigestion/ over-eating, eating outside routine times, spicy food
    
    * stress
    
    * food poisoning or infection.

Also, may be connected to excessive soluble fiber intake: see flatulance section.

\subsection{Solution}
First vomit the acid out. Pain killers. Antacids (salts to neutralize acid) or H2 blockers (Zinetac/ ranitidine, ), or proton pump inhibitors (Prilosec).

Rest.

Take a walk.

Avoid tea, caffeine, milk or anything acidic. Eat (oat meal) broth as possible.


\subsection{History}

\section{Forceful defecation}
\subsection{Cause}
Irritants like excessive spice.

\subsection{Solution}
\subsubsection{Safe defecation}
Ejection of feces with a fart often occurs - especially at the beginning of defecation. Splashing often occurs.

Avoid splashing in western toilets by inserting toilet paper on water surface to smoothly decelerate ejected feces.

Avoid infecting others and to stay safe: clean the toilet surroundings/ components, take a bath.


\section{Gut bacterial damage}
\tbc Fecal transplants.

\section{Diarrhoea}
\subsection{Cause}
Infection in the gastro-intestinal tract.

Osmotic diarrhoea: too much water is drawn into the bowels. Usually due to overconsumption of incompletely digestible substances. 

\subsubsection{Irritants}
Or apple or prune juice having high fructose: glucose ratio (undigested fructose due to excessive intake) or large quantities of artificial sweeteners (sorbitol is hard to absorb). So gastro-intestinal tract tries to get rid of irritants.

\subsection{Solution}
Rest.

Replenish bodily fluids and electrolytes faithfully to avoid dehydration. Symptomatic relief: Loperamide/ Eldoper: Ensures that food stays in the intestine longer, allowing for more water to be absorbed. Dosage: In adults and children 12 years of age and older, the usual dose is 4 mg (2 capsules) as a first dose, followed by 2 mg (1 capsule) after each unformed stool. The maximum dose is 16 mg/day.

Await diarrhoea cessation for a day or two. Visit a doctor of medicine if it does not.

Loperamide does not cure the infection which may have caused diarrhea in the first place. So, may need to ingest antibiotics.

Also see forceful defecation section.



\section{Flatulance}
\subsection{Cause}
Also see diarrhoea subsection.

\paragraph*{Excessive fiber intake}
Cramping, diarrhea, and intestinal gas are some of the problems associated with a sudden increase in fiber intake. Gradually increasing your fiber intake over a period of six to eight weeks can minimize undesirable effects. Some individuals may experience discomfort even with a gradual intake.

Soluble fiber is not broken down until it reaches the large intestine where digestion/ fermentation by microorganisms causes gas (flatulence).

\subsubsection{Gas producing food}
Excessive fiber intake: see Cramps Subsection.  Food high in complex oligosaccharides (which are acted on only by bacteria).

Undercooked beans (and possibly corn). 

Certain brands of whole wheat bread, low fat Blue bell ice cream (likely irritant: large quantity of sorbitol).

Large amounts of unripe banana or apple.

Excessive nut or dry fruit (eg: fig, date) intake.

Old Dove Promises Dark chocolate caused severe very smelly flatulance on 26th august 2010; on the previous day, it had caused severe (almost insatiable) hunger. Caffeine in it probably irritated the gut. [Theory: An intolerance to chocolate is an adverse reaction (not an immune response) by the body to chocolate. The adverse reaction results from the body's inability to metabolize the food.]

\paragraph*{Especially smelly flatulence}
Dates and garlic have been confirmed to yield very smelly gas (Oct/ Nov 2010: date walnut cake and garlic soup on separate occasions).

\paragraph*{Lactose tolerance level}
Drinking milk products beyond a certain level produces fart - the body can produce only a certain amount of lactase; the rest gets broken down by bacteria, producing flatus in the process.

Visited Seattle during the last weekend of August 2010, did not imbibe many milk products. Only farted when excretion was imminent. When I returned to Austin, I drank 2.25 tea cups of milk. After some hours, I produced a series of smelly farts; but there was no bloating or cramps.



\subsection{Solution}
Defecate processed food. Drink lot of water to ensure that this happens: it is normal to defecate 2-3 times a day.

Walking stimulates digestion, peristalysis, allows gas to escape smoothly.

To maintain general digestive health: Use food ingredients like coriander. Mix a teaspoon of lime juice with ginger (tablet?) and a glass of water and drink.

When eating food with a lot of insoluble fiber, eat the beano tablet. Restore the microbial ecosystem of the stomach by eating acidophilus probiotics tablet (as a substitute for yogurt, which still has some fermentable lactose).

In very bad cases, use activated charcoal (as an underwear or a diaper or as a dietary supplement; the last option causes it to interfere with nutrient/ drug absorption and is undesirable.)

\section{Inefficient/ high frequency excretion}
\subsection{Symptoms}
A short while after a spell of excretion, it turns out that the body is ready for excretion again. Excretion occurs in two bouts.

\subsection{Cause}
It is a common flaw to assume that the excretion is done after the first pellet has been excreted.

Insufficient water causes the feces to become hard, and the movement tough.

Excessive insoluble fiber/ roughage intake causes frequent bowen movement.

\subsection{Solution}
See shArIrika-kriyA strategy.


\section{Over-eating}
\subsection{Cause}
udvegaH, vA Ata\~NkaH. There is an autonomous gangion in the stomach which is active in such situations, eating is an attempt to calm it down.

\subsection{Solution}

\subsection{History}


\chapter{Sleep}
\section{Delay in falling asleep}
\subsection{Causes}
Indiscipline.

Noise. Exceessive heat or cold.

Eagerness to implement plans, to think and understand: ambition.

\subsection{Solution}
\subsubsection{Abandon plans}
Abandon ALL ambition, except the one to attempt to sleep. Set/ control ambition level to 0. Sleep is not a time for it. Make no plans, solve no problem, attempt no understanding. Don't solve mathematical problems in the middle of the night: that leads to alertness.

\subsubsection{Smooth gradient to sleep}
Avoid stimulation.

Try a light snack: banana or yogurt.

Stay in bed and do something boring. Focus on relaxing the body, restoring sleep.

\subsubsection{Other ideas}
svapna-darshaNa-abhinayena nidrAgamanaM sAdhyaM.

\subsubsection{Temperature}
Reduce body temperature slightly.

\subsubsection{Setting body clock}
Wake up naturally, to sunlight. Listen to your body - wake up when you feel ready.

Physical activity done earlier during the day shown to improve sleep. Activity level helps set the body clock.

Limit midday naps to afternoons.

Have a regular bed time: try to go to sleep 8 hours before sunrise. \exclaim{Don't try to squeeze in one more task before sleep.} Must get up only at the end of a full sleep cycle. Beware: Wake-up time is much more stable than sleep-time.

Maybe use melatonin supplements.

\section{Sleep disturuption}
\subsection{Cause}
\subsubsection{Temperature}
shavAsanaH upayojitavyaH.

Upon waking up in the middle of cold nights, it is difficult to return to sleep without being warm. Or, if temperature inside the sleeping bag increases too much, sleep is disrupted.

\subsubsection{Urine}
Waking up in the middle of the night to pass urine.

\subsubsection{Noise}
Excessive noise in the middle of the night causes sleep disturbance.

\subsubsection{Oxygen deprivation}
Lower oxygen levels lead to nightmares and waking.

\subsubsection{Stress}
Excessive worries lead to bad dreams and sleep disruption.



\subsection{Avoid disruption}
Reduce water intake during night to avoid urge to urinate.

Control temperature: Have warm clothing handy, if it is mildly cold. If it is hot, leave the fan on. Beware of cold drafts blowing from the window.

Close the door, if necessary, to avoid noise.

\subsection{Restore sleep}
Fix the problem which disrupted sleep.

See 'delay in falling asleep' section.

\section{Unusual need for sleep}
\subsection{Symptoms}
Sleepiness during the day/ work-time.

\subsection{Causes}
Delay in sleeping, or sleep disruption followed by inability to return to sleep.

Bed bugs, mosquitoes.

\subsection{Consequences}
Cognition, memory is impaired. Work is more careless: you forget and leave things where they are not supposed to have been left.

Hunger is often increased, atleast with mild sleep disruption.


\subsection{Solution}
Solve the problem which is causing the lack of sleep: delay in sleeping, or sleep disruption. Repay the sleep-debt.

Waking up after having overslept can result in grogginess. So you can wake up with a relaxing shavAsana.


\section{Snoring}
\subsection{Cause}
A loose palate?

\subsection{Solution}
Find a posture where you will not snore- In case of father (and probably I), sleeping on the side reduced snoring.


\chapter{Fatigue, body pain}
\section{Fatigue and strain}
\subsection{Cause}
\subsubsection{Forehead muscles}
Long hair falling on forehead causes strain to muscles in forehead, which in turn causes fatigue.

Lack of sleep causes desire to sleep during the day, when attention may be required. This causes strain in muscles in the forehead.

\subsection{Solution}
Put away long hair securely with a band.

Maybe go to sleep.

\section{Delayed onset muscle soreness (DOMS)}
\subsection{Cause}
Sudden change in exercise/ physical activity pattern causes microscopic tears in the muscles.

\subsection{Solution}
The first two days are painful. But rest and wait for the muscles to heal and grow (they undergo hypertropy); and it will go away.

Avoid it when possible - progress gradually.

\subsection{History}

\section{Severe muscle pain}
\subsection{Solution}
Ingest pain relief - eg: Ibuprofen (pick non sleep inducing version).

\chapter{Skin}
\section{Dry skin (Xerosis)}
Symptoms: Itching skin. Broken skin.
\subsection{Cause}
Oil coating over skin removed. Maybe because of dry cold winter weather or long shower.
\subsection{Solution} Restore oil layer.

\section{Buttock boil}
\subsection{Symptoms}
Difficulty in sitting.

\subsection{Cause}
Infection in a hair follicle. Or clogged hair follicle.

\subsection{Solution}
If possible, drain it: that brings immediate relief. Or apply heat.

Otherwise, a day's not sitting on a chair will cure it!

Avoid it by shifting position often.

\section{Prickly heat aka Miliaria}
\subsection{Cause}
Sweat gland ducts get plugged due to dead skin cells or bacteria or bad clothing. The trapped sweat leads to irritation (prickling), itching and to a rash of very small blisters, usually in a localized area of the skin.

\subsection{Treatment}
The main aim of treatment of miliaria is to prevent excessive sweating, many person use talcum powder which have some cooling effect.

Keep the crotch dry- maybe place tissue there.

Lidocain and Hydrocortizone cream.

Don't scratch.


\section{Unknown small rashes at multiple patches}
Candidates: Milaria rubra or hives.

\subsection{Symptoms}
A skin rash with pale red itchy bumps, at various parts of the body. Itchiness (pins and needles which is not numbness) may disrupt sleep.

\subsection{Cause}
If hives: In many cases, this is due to an allergic reaction; in many others the cause is unknown.

In case of milaria: sweat pooling + excessive skin bacteria.

\subsection{Solution}
If hives: Anti-histamine medication. Avoidance of allergen.

Pricking increases the pain - don't do that.

Sleeping in a way as to apply pressure on the itchy side enables sleep.

\subsubsection{Detecting allergen}
Keep a log of items eaten and pulse rates around the time of each meal (5min before, 30, 60, 90 minutes after), and while waking and sleeping. Consistent increases in pulse rate after eating a certain type of food indicates allergy.


\section{Abrasion and cuts}
\subsection{Solution}
Wash wound. Apply antiseptics like turmeric powder. Apply breathable, water proof, invisible, flexible liquid bandage twice to protect the wound from further dirt. For large wounds, use sprays.

\section{Bed bugs}
\subsection{Symptoms}
Multiple bites in a straight line. Drowsiness in the morning.

\subsection{Solution}
Dispose possibly contaminated bankets and clothes.

Ensure you get good sleep: isolate the cot from bedbugs - place cot legsin bowls of water.


\end{document}
