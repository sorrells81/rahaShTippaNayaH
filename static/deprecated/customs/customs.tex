\documentclass[oneside, article]{memoir}
\input{../../../work/packages}
\input{../../../work/packagesMemoir}
\usepackage{fontspec, xunicode}
%\setmainfont[Script=Devanagari]{Chandas}
\setmainfont[Script=Devanagari]{Kalimati}

\input{../../../work/packagesMemoir}
\usepackage{fontspec, xunicode}
%\setmainfont[Script=Devanagari]{Chandas}
\setmainfont[Script=Devanagari]{Kalimati}

\input{../../../work/packagesMemoir}
\usepackage{fontspec, xunicode}
%\setmainfont[Script=Devanagari]{Chandas}
\setmainfont[Script=Devanagari]{Kalimati}

\input{../../../work/macros}

\title{॥आचाराणि॥}
\author{विश्वासः वासुकेयः॥}

\begin{document}
\maketitle

\tableofcontents

\chapter{।कार्यक्रमः।}
\part{दीर्घ-कर्म-विधयः॥}
\chapter{सन्ध्यावन्दनं॥}
\section{समयः॥}
सन्ध्यावन्दनं सन्ध्यायां एव करणीयं। रक्त-अंशु-युक्तः भू-आकाश-सीमायां स्थितः सूर्यः उदये च अस्ते, एतौ समयौ प्रकृति-आराधनाय विशिष्टौ।

\section{विधिः॥}
दीर्घं आचमनं। प्राणायामः। सङ्कल्पः। सात्त्विकं त्यागं।

प्रोक्षणं। अर्घ्यप्रदानं।

केशवादि-तर्पणं। आचमनं। प्राणायामः।

आसनं। गायत्री-आवहनं। गायत्री-मन्त्र-जपं। गायत्री-उपस्थानं। आचमनं।

अभिवादनं। अर्घ्य-प्रदानं।

\subsection{गायत्री-आवाहनम्॥}
सर्वाङ्गन्यासः 'ॐ भूः' इत्यादि सप्त-व्याहृतयः उपयुज्य। 'ॐ आपो ज्योतिरसो..' इति मन्त्रस्य न्यासः शरीरे। (हृदयादि-न्यासः कथं?) 'बूर्भुवस्सुवः' इति दिक्-बन्धः।

न्यासं - वामदेव ऋषिः अनुष्टुप्‌ छन्दः गायत्री देवता।

"आयातु वरदा देवी अक्षरं ब्रह्म-सम्मितम्। गायत्रीं छन्दसां मातेदं ब्रह्म जुषस्व नः।" 

'ऒजोऽसि सहोऽसि बलमसि भ्राजोऽसि देवानां धामनां असि विश्वमसि विश्वायुः सर्वमसि सर्वायुः अभिभूरों'। इति वदन् करन्यासः।

'ॐ भूः, ॐ भुवः- ' इति-आरभतं सप्त-व्याहृतीं उपयुज्य अङ्ग-न्यासः। 'ॐ आपो ज्योतिरसो . . ' इति-प्रारभतं मन्त्रं उपयुजन् सर्वाङ्ग-न्यासः। 'वौषट्' इत्यादीन् उपयुज्य अङ्ग-न्यासः। 'भूर्भुवःस्वरों' इति दिग्बन्धः।

आवहन-अभिनयं कूर्वन् \\
'ग्आयत्रीं आवाहयामि। सावित्रीं आवाहयामि। सरस्वतीं आवाहयामि।' इति वक्तव्यं।

\subsection{गायत्री-उपस्थानं॥}
न्यासं गायत्री-आवहन-वत्। उद्वासन-मुद्रा-समेतं उपस्थान-अभिनयं। 'उत्तमे शिखरे देवी भूम्यां पर्वत मूर्धनी
ब्राह्मणेभ्यो ह्यनुज्ञानं गच्छ देवी यथा सुखं'।

\chapter{श्रीवैष्णवी इज्या॥}
\section{सामग्रयः॥}
६ पात्राणि - अर्ध्यं पाद्यं आचमनीयं स्नानीयं/पानीयं शुद्धोदकं प्रतिग्रह-पात्रं च।

दैवार्पणाय तोयं, भोजनं, आरत्यै धूपं वा इन्धनं, घण्टा।

दैव विग्रहः - विष्णोः मूर्तिः/चित्रं वा सालग्रामः।

\section{पूर्व-कार्यं॥}
कृत-सन्ध्या-वन्दनेन, धृत-पुण्डरेन, धृत-पवित्रेण शुद्ध्-भूतेन एव क्रीयं।

\section{कार्य-क्रमः॥}
आचमनं च प्राणायामः। सङ्कल्पः। पूर्व-सात्त्विक-त्यागः।

दीप-ज्वालनं, उचितं दुर्गासूक्त-अग्निमीळे-सदृशं अग्निमन्त्रं उपयुजन्।

चमसे तीर्थं गृहित्वा विविध-अङ्गुल-स्पर्श-समेतं दैव-मूलमन्त्रं उच्चरन् आत्मनं च पात्राणां प्रोक्षणं। पूजा-पात्र-परिकल्पनं।

शिल्प-मूर्तिः उपयुज्यते चेत् दैवस्य आवहनं।

साधारणतः योगनिद्रायां स्थितं दैवं मन्त्र-स्नान-अलङ्कार-भोज्य-पुनर्मन्त्र-पर्यङ्क-नाम-युक्तैः ६ आसनैः प्रयाणं कारयेत्। स्नानासन-अनन्तरं पुनः पूजा-पात्र-परिकल्पनं भवेत्।

अन्ते अन्तिम-सात्त्विक-त्यागः। शेष-आपः प्रतिग्रह-पात्रे सङ्ग्रहः। पूर्व-आचार्य-स्मृतिः। तदनन्तरं उपस्थितेभ्यः आरती-ज्वाला-दर्शणं च तीर्थ-प्रसाद-विस्तरणं।

\subsection{आसन-सेवा-विवराणि॥}
मन्त्रासने उचित-मन्त्रोच्छारणं भवितव्यं - उदाहरणाय प्रथमे मन्त्रासने, विघ्नविनाश-प्रार्थनं, मेधासूक्तादिभिः श्रुत-गोपन-प्रार्थनं, सुप्रभात-श्लोक-उच्चारणं।

स्नानासने उचित-श्लोक-उच्चरेन सह विशेषायां स्थालिकायां स्नानीयेन तोयेन स्नानं समर्पयन्, स्नाने-उपयुक्तं तीर्थं प्रतिग्रह-पात्रे सङ्ग्रह्णातु। क्षीर-स्नानं समर्पितं चेत् तदनन्तरं जल-स्नानेन प्रक्षालनं भूयात्। अन्ते विशेषेन वस्त्रेन अर्च-मूर्तिनः शुश्कीकरणं।

अलङ्कारासने पुष्प-तुलसि-कुङ्कुम-गन्ध-आदिभिः अलङ्कारं। तदनन्तरं घण्ट-घोषेन सह धूप-ज्वाला-अदिभिः सह प्रदक्षिण-पूर्वकं आरतिः/ दर्शणं, मन्त्रपुष्प-उच्चारणं च। 'नानाविध-परिमल-पुष्प-पत्रान् समर्पयामि' इति घोष-समेतं नामावलि-समेतं समर्पणं। 

भोज्यासने नैवेद्यं भोजनं पुरतः स्थापयित्वा, स्वभोजन-सदृशं तत्-परिसिञ्चनं कारयेत्, तदनन्तरं बीज-मन्त्रं उच्चरन् भगवता भोजन-स्वीकारः साभिनयं कल्पयेत्। अन्ते मुख-प्रक्षलनं, हस्त-प्रक्षालनं, पाद-प्रक्षालनं च आचमनं समर्पयेत्।

द्वितीये मन्त्रासने कर्पूर-ताम्बूलं अल्प-प्रोक्षणानन्तरं समर्पयेत्, तदनन्तरं मन्त्रोच्चारणं। ज्वाला-आरतिं अपि किञ्चित् समर्पयेत्।

पर्यङ्कं आसने श्लोकोच्चरः।

\subsection{आसन-प्रवेश-विधिः॥}
प्रत्येके आसनस्य प्रवेशे अर्घ्य-पाद्य-आचमनीय-पानीय-समर्पणं क्रीयं।

\subsubsection{विविधार्थ-तोय-समर्पण-विधिः॥}
दैवस्य बीज-मन्त्रं उक्त्वा 'x समर्पयामि' इति उच्चरन् चमसे स्थितं‌ तोयं दैवं दर्शयित्वा प्रतिग्रह-पात्रे निष्कासयेत्, तदनन्तरं विविधार्थ-आपः शुद्ध्तां रक्षितुं चमसस्य औपचारिकं प्रक्षालनं शुद्धोदके निमज्जनेन।

\subsection{तीर्थ-विस्तरण-विधिः।}
त्रिवरं आचमन-सदृषं दीयते। 

\end{document}
