\documentclass[oneside, article]{memoir}
\input{../../../work/packages}
\input{../../../work/packagesMemoir}
\usepackage{fontspec, xunicode}
%\setmainfont[Script=Devanagari]{Chandas}
\setmainfont[Script=Devanagari]{Kalimati}

\input{../../../work/packagesMemoir}
\usepackage{fontspec, xunicode}
%\setmainfont[Script=Devanagari]{Chandas}
\setmainfont[Script=Devanagari]{Kalimati}

\input{../../../work/packagesMemoir}
\usepackage{fontspec, xunicode}
%\setmainfont[Script=Devanagari]{Chandas}
\setmainfont[Script=Devanagari]{Kalimati}

\input{../../../work/packagesMemoir}
\input{../../../work/macros}
\title{Career strategy}
\author{Vishvas}

\begin{document}
\maketitle


\part{Objective and strategies}
\chapter{dhEyAH cha mArgAH vibhinna-ghanatAsu}
\section{vimarsha-tantrAH}
\subsection{udyOga-ruchayaH}
\subsubsection{svAbhAvikaH-ruchiH}
kliShTAnAM viShayANAM avagamanE svavEgE (vijJNAna-vardhana-AShayA cha vidyArthi-shikShaNAya cha iti dhEyaH prAyaH amukhyaH).

\subsubsection{Avashyaka-guNAni}
1. pUrvajJNAta-kAryEShu vaividhyaM vA nUtanE viShayE anubhavaH: in the end products/clients, data and techniques used to solve problems.

2. prabhAvaH: through my ideas, on the company/society. In startups, impact is immediate; in big companies, impact is there because millions of people use the product; in research impact is long-term.

3. kaushala-vardhanaM: Developing creativity, interpersonal skills  and theoretical skills (if research job)

4. susaMbandhAni sahakAribhiH saha.

\subsubsection{kAmya-guNAni}
1. Not very high pace, so that lifestyle is balanced.

2. Bridge to PhD: add value to the application.

\subsection{dhyEya-kalpanAyAM}
\subsubsection{vimarsha-bIja-prashnAH}
kathaM dhArmikA sEvA samarpaNIyA? mama shaktiM cha ruchiM cha paristhitiM parigaNya kiM shrEShThaM sAdhituM shaknOmi?

parantu dhEyAnAM nirdhAraH cha sAdhana-mArgasya yOjanA adhikatayA bhinnau viShayau. dhEyAnAM nirdhArE kEvalaM 'nyUnatayA sAdhyaM vA?' iti praShTavyaM bhavati, tadanantaraM sAdhana-mArgasya yOjanA bhUyAt.

\subsubsection{Having great impact}
Consider some positive outcome o which would have taken x years to happen without you: if you make it happen in y<<x years, you are said to have had a positive impact. The case where y is much smaller compared to x and the case where y is significantly smaller than x but o is of great value are called “great impact”.

You can have great impact in many endeavors - but socio-cultural movements, art, science and industry in particular have some great examples.

\subsubsection{dhEya-pAvitryaM}
I have very ambitious professional, social and personal dreams. I try to not contaminate them with other more common competitive ideas of "success" - which I find silly.To quote someone: Often, people work long hard hours in jobs they hate to earn money to buy things they don't need, to impress people they don't like.

Rat-race, money beyond what is necessary for a comfortable and fruitful family life is not an objective.

\subsection{mArga-yOjanAyAM}
\subsubsection{Deliberate practice}
Choose the career path such that you are always working close to the edge of your technical abilities.

\subsubsection{suparisarasya sthAnaM}
EkAkinA mahat-sAdhanaM kaShTaM. anyEShAM siddhayaH upayujya Eva anyEShAM sahakArEna cha utkRRiShTaM sAdhyatE.

\section{dIrgha-kAlInAH}
\subsection{dhEyAH kalpitAH}
\subsubsection{anusandhAnaM abhiyAntrikaM}
Maybe you can apply the computer science and mathematics you know in disciplines which use that knowledge more peripherally: eg: computational biology. Maybe you can apply deep technical understanding in trade and engineering enterprises.

Or may you can be part of a team which builds very useful tools with simple, known technology.

Industry research role models: Junling Hu. Tricia Hoffman.

\paragraph{Examples of impact}
Making business more efficient. Click fraud detection. Predicting which patients will be hospitalized. Consider the many million-dollar prizes awarded for advances.

\subsubsection{saMshOdhanaM}
mUla-saMshOdhanaM bhavitavyaM.

\subsubsection{shikShaNaM}
anya-saMshOdhakAnAM shikShaNaM, sukRRitAnAM prOtsAhanaM.

\subsubsection{saMskRRiti-rakShaNaM}
Subversion of the western subversion of the hindu dharma (the western subversion has gone too far).

prathamaM sva-parivAraH susaMskRRitaH cha mahat-sAdhanE shaktaH bhUyAt.

\subsubsection{vitta-upArjanaM}
kushalAn api na pramaditavyaM. nivRRIttyai cha ArOgyA-rakShaNAya cha apatya-shikShaNAya dhanaH sa~NgRRihNAni.

\subsection{saMshodhana-dhEya-samarthanaM}
\subsubsection{aadarshe saahasasya maahatyaM}
bhavataH kule laghusu saMshodhakeShu asti ekaH bhavaan| kevala asmaat saahasaat eva asi tvaM aadarshaH dakShiNa-karnaaTaka-braahmaNebhyaH cha bhaargavebhyaH| chaaNakya-kiirtiH smaritavyaH taiH|

\subsubsection{Resource expended during training is not wasted}
Resources  spent on training are part of the society's stochastic resource allocation game, which fully expects different shades of success among graduates.

\subsubsection{Fewer, but good discoveries}
Maybe your skills will never match that of prodigies like Terrance Tao; you can still make non-trivial contributions. Maybe, after research training and sincere attempts, you are unable to make many truly great discoveries: but maybe you can make a few (even one) great discovery!


\subsection{mArgAH yOjitAH}
saMskRRiti-rakShaNaM svasaMskArANAM asa\~Nkuchita-anusaraNEna bahumAtrayA sAdhyatE.

\subsubsection{saMshOdhaka-AchAryaH}
uttame saMshodhane cha bandhuunaaM saMskRRiti-uddhaare cha saMshodhana-vidyaarthiinaaM protsaahane mama prabhaavasya cha shaktiinaaM cha ruchiinaaM uttamaM upayogaM bhavati agreShu 10 varSheShu iti j~naayate yadi ahaM mama vartamaana saMshodhana-shikShaNaM parigaNaami chet| tat-pashchaat anyaM sukaaryaM kartuM api shaknomi|

Cosma Shalizi sadRRishaM saMshOdhanaM kurvan, saMshOdhana-shikShAM dadan, vividhAn-pustakAn paThan, bhAShaNAn shRRiNvan bauddhikaM jIvanaM yApanaM ruchikaraM.

\subsubsection{udyama-abhiyAntrikaH}
In an ideal engineering job, you work with a wonderful team to build something novel and vital - even if it is nowhere close to the toughest scientific problems - that vastly exceeds what any of you could do by yourself.

Examples: google search, google plus, google extending html standards.

You can stay closer to end users and produce vitally useful products: nvidia interacts with game developers to see what is needed next.

\paragraph{Roles}
Be a technical manager/ director or an individual contributor in such a team. Even in peripheral roles such as providing security, observing performance/ market trends for opportunities, you are indirectly contributing to/ enabling the kAryakula's products.

\subsubsection{saMshOdhanaM svatantraM}
\paragraph{sAmsthAnika-samarthanasya anAvashyakatA}
\subitem When Ramanujan lay unemployed, he was able to spend a lot of productive time on mathematics.

\subitem Consider Veeresh KR's career of discovery.

vijJNAnE Darwin/ Newton sadRRishaH bhUtvA anya-sahayOgibhiH vinA uttamaM anvEShaNaM sAdhyaM.

bhautikE cha sAmsthAnikE EkAntE sthitaH api mathoverflow AdInAM upayOgEna

\paragraph{kaShTAH}
parantu tasmAt anEkaiH svatantra-vijJNAnibhiH sadRRishaH duShita-vichArAn api satyaM iti marshituM sAdhyaM.

\subsubsection{gRRiha-chAlakaH}
shrutyAH udyOgE vyasthAyAM apatya-pOShaNaM cha gRRiha-pAlanaM ityAdIn kAryAn kUrvan gRRihE saMshOdhana-sadRRishaM svatantraM kAryaM sAdhyaM.

\subsection{tiraskRRita-mArgAH}
\subsubsection{Entrepreuneur}
Many steep emotional highs and lows, demands heavy time investment, financially risky.

\subsubsection{Non-profits}
Not much opportunity for growth within the organization, usually; demands very much hard work for low pay - as in graduate school.

\subsubsection{Financial analysis}
Wall Street has jobs for people who are good with numbers.

Finance jobs are very hard work.
Finance is not science.

There is compensation for all the hassle.

dharma-sAdhanAya adakSha-mArgaH iti tiraskRRitaH.

\section{madhyama-kAlInAH}
\subsection{dhEyAH}
\subsubsection{Statistics saMshOdhanaM}
Maturation of research skills is imminent: confidence of others and yourself in your abilities comes from a successfully and autonomously executed research project. Follow it up. Get a PhD.

How to become really good at making important discoveries? Converge to the optimal answer rapidly, and live that answer.

jagati jJNAnaH apAraH vibhinnaH. machine learning viShaye jJNAnaM cha kaushalaM prApte mayA. yadi sAdhyaM lOka-kalyANE upayojayituM ichChAmi.

\paragraph{samarthanaM}
anEna dIrgha-kAlAya yOjitAn saMshOdhaka-AchArya-abhiyAntrika-ityAdIn sadhana-mArgAn prAptuM shaknOmi.

\subsection{mArgAH}
\subsubsection{saMshOdhana-anubhavaH dIrghaH}
See training section of career skills strategy.

Discover interesting (but not necessarily revolutionary) results, earn a strong reputation.

\subsubsection{udyama-nEtRRi-saMparkaH}
Work as an analyst or research engineer, and interact heavily with C-level leadership of companies.

\section{vartamAna-kAlInaH}
\subsection{dhEyAH saMshOdhana-bala-prAptiH}
Maturation of research skills is imminent. Follow it up. Broaden and deepen knowledge in machine learning/ algorithms analysis.

\subsection{saMshOdhana-anusaraNasya samarthanaM}
yasmin asti abhiruchiH, tasmin vardhitavyaM sAmarthyaM.

Also see similar section in problem solving strategy.

\subsection{mArgaH}
Do volunteer research. mArgaM sukShmatayA avagantuM google docs vIkShitavyaM.

\subsubsection{PhD prApti-vyavasthA}
For ideas about research topics, see research topics file. shruti amErikAyAM asti. ataH amErikAyAM bhUyAt PhD adhyAyaH.

\subsubsection{udyama-abhiyAntrikaH}
Work as a research engineer or an analyst at a krAnti-kEndra like Google.

\section{Research: Common Motivations}
The main motivation would come from a sense of child-like joy.

\subsection{Impressing peers}
"young men strive to achieve high status among their peer group". 

\subsection{Ignoring money}
"Science is a wonderful thing if one does not have to earn one's living at it." -- Albert Einstein 

'Jobs in science pay far less than jobs in the professions and business held by women of similar ability. A lot of men are irrational, romantic, stubborn, and unwilling to admit that they've made a big mistake. ... there are more men than women who have chosen to stick with the choice that they made at age 18 to become a professor of science or mathematics.' -- Philip Greenspun.

\subsection{Immigrants}
"The desperate need for graduate student labor and lack of Americans who are interested in PhD programs in science and engineering means that you'll have no trouble getting a visa. When you finish your degree, a small amount of paperwork will suffice to ensure your continued place in the legal American work force. Science may be one of the lowest paid fields for high IQ people in the U.S., but it pays a lot better than most jobs in China or India."

\subsection{Ignoring old age and family}
"If you interview old people and ask "What are the greatest sources of satisfaction and happiness in your life?" almost always the answer "my children" comes back. At the age when people are choosing careers, the idea of having children is often unappealing and certainly few have the idea that one should choose a "kid-friendly" career. Old people, on average, also have higher income requirements than young people." - P Greenspun.

\subsection{Ignoring job security}
"If you have a narrow education and have been earning an academic salary, it is much tougher to change careers at age 45 or 50 than for someone who was in a job where the earnings are higher and begin at a younger age."

\chapter{Job Strategy}
\section{sAMsthAnika-udyOga-guNa-mEyanaM}
Goodness of jobs is a function of a] the extant to which they allow you time and resources to achieve your personal goals, and b] your evaluation of the worth of the remaining parts of the job. The best job is one where you are paid and enabled with resources and colleagues to achieve exactly what you like to achieve.

\section{saMshOdhaka-Acharya-mArgaH}
\subsection{mArga-kramaH}
Consider going to the industry before returning to the academia - With a greater focus on research, one may have a more productive time in the industry. Also, it pays more, and one can negotiate a higher starting salary. However, variety in research may be lesser during the years in the industry as a result.

\subsection{sulabha-saMsthAnAni}
\subitem Research scientist at a good department, or a professor in another country like Ireland, Australia, New Zealand, Netherlands or the United Kingdom.
\subsubitem Eg: Aneesh Chaudry.
\subitem Find a place which atleast has e-access to important journals and good bandwidth for internet access.

Transcend artificial barriers of "department". When experts from many disciplines collaborate, fresh insights are found. [Reference]
\subitem Eg: David Haussler's work spanned impressive tracts of Computational learning theory and bioinformatics.

\section{Getting jobs}
\subsection{Timing}
Get a job before you graduate - Employers have a bad impression about unemployed people.

\subsection{Job search venues}
Good places to look for employment are CRA, kdNuggets, linkedin (very useful - in getting internal references), indeed.com, monster.com.

Have the flexibility to change the geographical destination in pursuit of the objective.

\subsubsection{Contacts}
Most jobs are not even advertised. 4/5 of all jobs go to personal contacts. Research positions are mainly obtained through contacts.

Important in both getting the interviews and others arguing for you.

\subsection{Internships}
Internship is when they check you out.

\subsection{Interview and appraisal}
Brand yourself. Be prepared with an answer to 'tell me about yourself'.

Know the prospective employer: what makes them special?

\subsection{Dealing with rejection}
The stable marriage problem is a recurring theme in hiring. So, don't be overly discouraged upon rejection.

\section{Job information}
\subsection{External advice}
Always take advice of a considered selection of people in making big decisions.

\subitem Talk to professors, professionals and career counselors. People like talking about what they do. It is called "information interviewing".

\subitem Talk to company recruiters.

\subsection{Talk to current employees}
People like talking about what they do. This is especially true if they know that you're not after a job with their organization. It is called "information interviewing". There will be a signficant percentage of successful interviews.
\subitem Dress suitably.
\subitem Be prepared - do not ask questions which could have been answered by other means. Ask:
\subsubitem How did you get your job?
\subsubitem How did you decide you wanted to pursue this specific career?
\subsubitem Did you have a mentor who helped you? Did you have role models?
\subsubitem What satisfactions do you get from your position?
\subsubitem Did the organizational culture turn out to be what you expected?
\subsubitem How were the most recently employed people hired?
\subsubitem How would you suggest someone with my background to look for a job?

\section{antarjAle kAryaM}
antarjAla-mAdhyamena svatantraM kAryaM sAdhyaM. 

abhiyAntrika-kAryAya santi kShEtrANi - Eg: Freelance coding, social open-source coding websites (eg: github etc..).

saMshOdhana-kAryAya api santi kShEtrANi, udAharaNAya - mathoverflow, theoretical computer science stackexchange.

jJNAna-prAptyai cha shikShaNAya api santi kShEtrANi jJNAna-prApti-sUtrE vivRRitAni.

sAmUhika-kAryaM antarjAle samAja-sUtre vimRRiShyatE.

\part{sAMsthAnika-kAryaM}
\chapter{Industry}
\section{Research labs in industry and institutes}
Industry encourages, in general, short term research; whereas academia should ideally encourage long term, abstract research.

\subsection{Choices}
Microsoft lab: Freedom to pick your own projects.

Theory group of AT\&T

IBM's theory group.

IBM India, located in Delhi and Bangalore, requires you to work on company projects for 3 days in the week.

DE shaw does good computational chemistry.

\subsection{Strategy}
Intern or Post-doc first.

IBM Almaden offers a Raviv fellowship

\subsection{Rewards}
\begin{itemize}
\item SRI international has an impressive list of accomplishments, which are reminiscent of Bell Labs. However, Bell Labs seems to have lapsed into obscurity.
\item Working in the industry could provide a means of building up credentials required for entering the academia later. 
\subitem Eg: Russell Grainer. Brent Waters.
\subitem Eg: Andrew McCallum (h index 34 as of 2007! Born in 1967): PhD University of Rochester(1995) to Research Fellow U Rochester to senior scientist in industry and adjunct prof at CMU(2000) to VP R\&D in industry (2002) to Research Associate Professor UMass (2003) to Associate Professor (pre-tenure) UMass

\item It can result in fresh research directions:
\subitem Inderjit Dhillon worked in IBM Almaden before moving to UT. It was there that he found important applications of linear algebra in data-mining.
\item Research labs with a lacuna in a certain research area present excellent opportunities:
\subitem In the Samsung lab in San Jose, data-mining expertise was lacking. Hyuk was able to fill the gap and demonstrate excellent results, impressing the research staff there. Just applying his algorithms led him to fascinating results. The internship lead to an employment offer. A key part in this happened when Hyuk was able to communicate the important ideas of his research to the non-machine learning audience effectively, while making good use of the white-boards available. The audience was very impressed with the mathematics.
\end{itemize}

\subsection{Risks}
\begin{itemize}
\item Industry researchers have to sign non-disclosure agreements.
\item There is a high failure rate in the biotech industry.
\item External coercion of research interests. Necessity of having to sell research projects to the company.
\item Many places look for people who are willing to code.
\item J Kemp, who did his dissertation in an extremely abstract and mathematical field in computer science had a tough time finding an industrial job.
\end{itemize}

\section{Engineering in Industry and consulting companies}
If you do go into engineering, make an effort to find a great team to work with where either your direct manager or other leaders in the company have a clear compelling vision for what you are going to do.

\subsection{Risks}
It's not feasible to pursue one career and continue to do a little academic research on the side. You may be able to discuss things with colleagues whose main job is research and contribute thoughts.

Engineering can often turn into a confused mass of uninteresting drudgery.

Managers are often simply engineers who were good at their original job but lack true management skills.

\subsection{Choices}
The Georgia Tech Research Institute (GTRI) is the nonprofit applied research arm of the Georgia Institute of Technology in Atlanta, Georgia. Many universities are building facilities dedicated to biology research. GTRI has offices in Ireland, France and Atlanta. Smaller than SWRI.

SWRI consults with industry. A private non profit. To pursue your own projects, write internal grants. Else work on specific, application oriented projects. Publishing may require removing information identifiable to clients.

\subsection{Hard Interview questions}
find median of a $10^{12}$ numbers distributed among 1k machines with minimal communication. Solution: Iteratively refined histogram.

Find 3 numbers whose sum is 0 in less than $O(n^3)$ time. Solution: Consider the problem of finding 2 numbers which sum to a certain number.

\chapter{Academia}
\section{Mission}
Teaching, research and service towards these goals are the main activities.

\subsection{Research}
Industry encourages, in general, short term research; whereas academia should ideally encourage long term, abstract research.

\section{Location}
Working in Europe can mean having to write fewer grant proposals, guaranteed funding for a few students and job safety.

bhaarate saMshodhana-kendraaH nuutanaaH api kraanti-kendraaH|

Many opportunities in India: IISc, IISER's, IIT's, CMI, IMRes.

\section{Hunting jobs}
Good places to look for employment are CRA and kdNuggets.

The money magazine indicated in August 2007 that the average salary of a college professor is \$81500, and the annual job openings are 95300.

tenure : tenure-track : no tenure = 25:10:65.

\subsection{Departments}
Consider departments such as Operations research, Business management, Electrical and computer engineering too.

\subsection{Job crunch}
\begin{itemize}
\item PhD production:
\item The number of PhD's produced in USA has been growing exponentially since 1880. This increase stopped suddenly in 1970. "The big crunch".
\item "Over-production of PhD's leads to intense competition and a hellish lifestyle, but does not end up selecting the best scientists - only the most fiercely competitive."
\item Professors produce 15 descendants on average, whereas they only need to produce 1 descendant in order to sate academic demand. Yet, PhD's are trained only for academic research.

\item Applicant : tenure track job = 300:1.
\item The number of people who graduate from “top 10” computer science programs every year is approximately 250. Conversely, the number of faculty positions that get filled at “top 50” research universities is about 25. [Ref]
\item Many CS departments first try to hire in applied areas and failing that will only then start to consider theory candidates. [Reference]
\item "Hot fields" are an exception. Reputed department jobs are even more competitive. The ratio here is 1000:1. "No job openings except for some bioinformatics or statistics."
\item Universities tend not to replace retiring faculties due to the grant-crunch; and hire adjunct faculty.
\item Age:
\item Significantly older graduates are more likely to be rejected.
\item In Italian universities, there is Reverse Age Discrimination.
\end{itemize}


\subsection{Getting a Tenure track job}
\begin{itemize}
\item Advisor's reputation and influence:
\subitem PlagueRat says that this is the most important factor.
\subitem Very important according to the book "Getting what you came for". In the academia, this is much more important than the department reputation.

\item Department and University reputation:
\subitem Casual Observation: If a person graduates from a department ranked X, his best chance is to obtain a tenure at departments ranked X+-10.
\subitem A survey of CMU CS faculty reveals this:
\subsubitem graduates from one of the top 15 departments : graduates from elsewhere = 13:2
\subitem A survey of UMass graduates who acquired university tenure showed that, on average, an academic aspirant is most likely to end up in a department ranked below 45.
\subitem Leslie Valiant, the excellent researcher who does path breaking theory research in diverse areas without caring about paper count, graduated from the University of Warwick. Sariel Har Peled graduated from Tel Aviv University.

\item Post-Doctoral research or a second PhD:
\subitem Is it from an exotic place like the Max Planck institute or Rockefeller or Cold Spring Harbor?
\subitem The Santa Fe Institute may be a good place: "sheer concentrated intellectual stimulation — not to mention views like this from your office window — there is probably no better position for an independent-minded young scientist with interdisciplinary interests."
\subitem Is the post-doc sponsor an influential person: such as the editor or the originator of a certain popular journal?
\subitem What kind of recommendation letter can you get from him?
\subitem Are you getting 2-3 good journal papers a year?
\subitem What subject was the postdoc in?
\subitem Examples:
\subsubitem Cosma Shalizi of CMU did two postdocs.
\subsubitem Eric P Xing of CMU has a PhD (Biol) from Rutgers and a PhD (CS) from Berkeley.
\subsubitem Turing awardee Andrew Chi-Chih Yao has a Physics PhD from Harvard and a CS PhD from UIUC.
\subsubitem Prof. Drennan got a PhD from the Univresity of Michigan, and did a post-doc at Caltech, before joining the MIT chemistry faculty.
\subitem In some disciplines, post-doctoral research has become a required credential.
\subitem According to Nikhil, MIT physics post-docs earn 50000\$.
\subitem Risks:
\subsubitem Most post-docs cannot find a tenure track job. So, post-docs in their 40's and 50's exist. Then, they are forced to leave the academia because they cannot land another post-doc.
\subsubitem Post-docs are voluntary slaves, as the rat said. They work for 70 hrs a week for 40 hrs pay.
\subsubitem "Post-docs" from government labs are not appreciated.
\subsubitem The problem may not be in your interest-area.
\subsubitem Low salary: \$25000 to \$30000 per year.

\item Prior research work in the industry: Examples: Suresh Venkatasubramanian, Madhu Sudan,

\item Publications:
\subitem Are they two or three per year? Are they published in Cell, Nature, Science or PNAS?
\subitem Risk:
\subsubitem Obscure, short papers in obscure journals are graded negatively.

\item Performance at a lesser ranked department:
\subitem UMass graduate Prof. Sandholm had a remarkable career at WUSTL before joining gatech as an associate professor.
\subitem CMU graduate and the mathematical machine learning genius UMass professor Sridhar Mahadevan worked at University of South Florida for many years before moving to UMass.
\subitem Risk:
\subsubitem Have you been assistant professor in two or more institutions?

\item Availability of transferable grant.

\item Recommendation letter:
Do you have a stellar recommendation letter from a luminary?

\item Specialization match.
\subitem Did the department advertise for someone with your specialization?
\subitem Can you act as a bridge between two faculty and become co-investigators?
\subitem Departments prefer candidates with theoretical rigor.

\item Teaching experience.
\subitem How have students rated you?
\subitem Risks: Were you a college teacher for long after graduation?

\item Age.
\subitem Risks:
\subsubitem Applicants significantly older than average are rejected.
\item Absence of career breaks:
\subitem Risks:
\subsubitem Were you truck driver for 10 years before going for a PhD?

\item Absence of negative reputation:
\subitem Risk:
\subsubitem Bad gossip, whether substantial or not, marks you.

\item PhD completion time.

\item How do you present yourself?
\subitem Do you give effective talks?

\item Politics:
\subitem Do you have contacts among the target department's faculty?
\subitem All tenured faculty members need to approve the granting of tenure. If one faculty member is not in favor of granting tenure, there is a little trouble. But, if two faculty members are of that opinion, it can doom the case for tenure.

\item Honors:
\subitem Did you get the "best-dissertation" award? [People who received the UTexas CS Best dissertation award seem more likely to have a good academic career.]

\item Choice of target department:
\subitem Jonathan Kelner joined Applied Math department after PhD in Computer Science and postdoc at IAS.

\item Low Salary: \$42,000 (Entry-level lawyers can make \$150000 a year.)

\item Off-beat and new departments (dedicated to new interdisciplinary areas) do not always attract the attention of the majority of job seekers.
\end{itemize}


\section{Risks}
\subsection{Average trajectory in USA}
From P Greenspun: The average trajectory for a successful scientist is the following:

age 18-22: paying high tuition fees at an undergraduate college

age 22-30: graduate school, possibly with a bit of work, living on a stipend of \$1800 per month

age 30-35: working as a post-doc for \$30,000 to \$35,000 per year
age 36-43: professor at a good, but not great, university for \$65,000 per year

age 44: with (if lucky) young children at home, fired by the university ("denied tenure" is the more polite term for the folks that universities discard), begins searching for a job in a market where employers primarily wish to hire folks in their early 30s 

\subsection{Salary}
\begin{itemize}

\item In a certain university department in Houston with a masters program but no PhD program, salary seems to be around 60K.
\item Universities pay 9 months salary - the remaining 3 months' salary comes from grants. However, for initial 2 years, 12 months' salary may be provided.
\item In a public research university, salary seems to be about 80K (9-month). 12 month salary is much higher.
\item In a private research university, salary seems to be about 100K (9-month). There's usually more room for negotiation at private universities and can go up much more.
\item Salary in a research lab seems to be \$120K.
\item Competing offers and counter-offers are effective in eliciting salary hikes.
\end{itemize}

\subsection{Grant crunch}
\begin{itemize}

\item This leads to diminished rewards for academic career in the future.
\item In post-cold-war era, government funding enthusiasm has decreased. Private companies do not invest in new basic-science labs too.
\item 25\% ("summer salary") to 100\% salary of a Principal Investigator comes from grants. In some institutions, if grant is not renewed, you loose your job, sometimes even if you are tenured.
\item Funding success rate = 15-20\%. (Figure inflated by the presence of superstars.)
\item NIH Grant renewal rate = 30\%. Rate of second renewal = 60\%.
\end{itemize}

\subsection{Dishonesty and personality problems}
Faced with increasing pressures, academic dishonesty has increased. Funding decisions are "peer reviewed".

\subsection{Paucity of students}
Number of students wanting to get a PhD in Computer Science is dropping.

\subsection{Work-load}
Many conflicting demands on your time.

In some places, 6 hours of courses are required to be taught every week.

In a certain university department in Houston with a masters program but no PhD program, the teaching load seems to be 2 undergraduate courses and 1 graduate course. But often senior seminar courses and labs are included in this number.

\subsubsection{Personal life}
"My case is extreme, but my wife and I only saw each other once a month for 15 years."

"The average full professor, someone who has been teaching for, say, fifteen years or longer, is making five times less than the average president at most institutions; works 60 - 70 hour weeks, uses holidays to do research, and tries desperately to find time to be a good spouse, father, mother, or partner. The life of the mind may seem cushy, but it is not." 

\subsection{Common bitterness}
"The tenured Nobel Prize winners are pretty happy, but they are a small proportion of the total. The average scientist that I encounter expresses bitterness about (a) low pay, (b) not getting enough credit or references to his or her work, (c) not knowing where the next job is coming from, (d) not having enough money or job security to get married and/or have children. If these folks were experiencing day-to-day joy at their bench, I wouldn't expect them to hold onto so much bitterness and envy. " - P Greenspun.



\subsection{Other factors}
Outspoken atheism can be persecuted.

\section{Adjunct work}
There exists a hierarchy of salaries, at the top of which, one can earn as much as an associate professor. (\$52000)

\subsection{Risks}
Dead end, not entry level.

\section{Research professor}
Depends on the university, but often there are positions that are 100\% research (often including supervision of grad students or postdocs), funded by grants. It is often easier to get, but you still need publications, not just a degree. There may be a tradeoff between autonomy and security. If you are essentially a lab manager for someone who's really good at getting grants, your job security may be reasonable, comparable to private industry, but less than with tenure.  If you do a good job, you may be able to get another position at the same university if your boss loses his funding.  If you have to raise your own salary via grants, you have autonomy but will probably be short of money.

\section{Tenured work}
Academic superstars command 6-figure salaries.

"The UK has largely given up the tenure system. This means on the one hand that lecturers have permanent positions as soon as they pass a probation." [Ref]

(ivies do not have "tenure-track" per se)

\subsection{Tenure requirements}
\begin{itemize}

\item Research quality:
\item Randy Pausch got tenure one year early by working hard.
\item publication rate
\item established ability to draw grants
\item 2 monographs in nine years.
\item When Don Knuth applied to be a associate professor at Stanford, the committee was uneasy because of his age, but were convinced by the credibility represented by the "Art of Computer Programming" book.
\item 3 review stages
\item 10 outside letters of recommendation
\item undergraduate students' feedback
\item personal statements
\item service
\item respect and standing in the department, especially with respect to the powerful.
\end{itemize}

\subsection{Risks}

Tenure denial rate ranges from 10\% to 100\%. Tenure may be denied if the committee, or the chairman, or the dean, decides against you. This could happen if you fail to perform, or if you have enemies, or if an influential person wants to create vacancies to draw someone in.

Not pursuing deep goals: The great thing about tenure is that it means your research can be driven by your actual interests instead of the ever-changing winds of fashion. The problem is, by the time many people get tenure, they've become such slaves of fashion that they no longer know what it means to follow their own interests.

\section{High schools and community colleges}
Science teachers are in demand.

Community colleges often hire people with masters degree.


\end{document}
