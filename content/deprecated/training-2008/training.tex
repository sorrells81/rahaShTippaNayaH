\documentclass[oneside, article]{memoir}
\input{../../../work/packages}
\input{../../../work/packagesMemoir}
\usepackage{fontspec, xunicode}
%\setmainfont[Script=Devanagari]{Chandas}
\setmainfont[Script=Devanagari]{Kalimati}

\input{../../../work/packagesMemoir}
\usepackage{fontspec, xunicode}
%\setmainfont[Script=Devanagari]{Chandas}
\setmainfont[Script=Devanagari]{Kalimati}

\input{../../../work/packagesMemoir}
\usepackage{fontspec, xunicode}
%\setmainfont[Script=Devanagari]{Chandas}
\setmainfont[Script=Devanagari]{Kalimati}

\input{../../../work/packagesMemoir}
\title{Training strategy}
\author{vishvAs}

\begin{document}
\maketitle


\chapter{High level strategy}
\section{Deliberate practice}
Get frequent, accurate feedback - either by examining your own skills (eg: reading papers), or by listening to/ talking to others.

\section{Learning with courses}
\begin{itemize}
\item Also see the strategy for learning from lectures in career skills strategy document.
\item Read before attending the lecture. Read after the lecture. Tackle exercises.
\item Befriend and acquire contact details of at least 3 students.
\item Bringing photocopies of relevant pages of the book to the class helps in quick understanding and orderly recording.
\item Regularly attend office hours.
\item You can contribute to the discussion and lectures in the classroom in exactly two ways: By asking questions, or by answering questions. Contribute cleverly in both ways, and you will be considered an attentive student.
\item Take easy courses, which are short on labor, and which are of appropriate level. Some students take courses even until they graduate. This ensures that knowledge acquired does not dissipate due to disuse. Research is more important.
\item Consider ut CIS survey ratings, ratemyprofessor.com, utlife.com ratings before taking courses.
\end{itemize}

\subsection{Target items to learn}
Acquire proper prorgamming skills: Reason about programs well; learn the theory to do this.

\subsection{Exam strategy}
Read up homework solutions and results; write them in cheat-sheet. Solve many problems as warmup. Beware the calculation mistakes. Use separate sheets for separate questions. Don't over hydrate yourself. Sleep well. Solve easy, short questions first; solve many questions parallelly.

\subitem Strategy for open book exams.


\chapter{Advisor search}
\section{Qualities of good advisors}
"A good advisor will give you a hot topic to work on where you can get results that people will find interesting. A good advisor will be so famous that merely being their student will cause people to be interested in you. A good advisor will go to bat for you when it comes time for you to get a job. A good advisor will be politically well-connected and lubricate your way straight to the holy groves of academe. A good advisor will also work your butt off and scare the crap out of you by expecting you to know about millions of things - don't let that put you off." [Ref]

Some advisors are good mentors, interested in development of the student's abilities and career. Others are just employers, whose agenda does not necessarily include a student's growth.



\subsection{Knowledge gathering}
\subitem About the advisor's personality and working style:
\subsubitem Does he have experience advising students? Does he know how to behave with students?
\subsubitem Do his students like him? Do you respect him? (He does not have to be a nice person.)
\subsubitem What problems do his student face with him?
\subsubitem Does he give sufficient advice to his students? Is he accessible?
\subsubitem How much autonomy is granted to students? Are students under heavy pressure?
\subsubitem When the time for independent work arrives, \\will everything be smooth?
\subsubitem How experienced in advising students is he?
Older professors are more skilled at advising students. Younger professors are more eager for research.
\subitem About the advisor's research:
\subsubitem What is he famous for?
\subsubitem Does he have money?
\subsubitem Is he eager and enthusiastic about his research?
\subsubitem Does he still do research independently?
\subitem About the topic being studied:
\subsubitem What are the important problems being studied?
\subsubitem What is the fraction of time they spend with pen and paper?
\subsubitem What kind of mathematics is involved?
\subsubitem What is the long term applicability of the skills learnt due to the research work?
\subitem About the research group:
\subsubitem Is the group collaborative?
\subitem About the career path:
\subsubitem Where have his students gone?

\section{Narrative for time spent in umass}
I tried computational biology research at UMass. My undergraduate education, from which I graduated 4 years earlier was very dissatsifactory; I had almost no exposure to research; let alone research in complexity theory, algorithms, AI and machine learning. At UMass, I discovered that protein structure prediction was too experimental for my taste and that theory and machine learning appealed to me more, with their use of mathematics. Hence, I decided to properly explore computer science and try research in theoretical computer science.

\chapter{Benefitting from advisors}
\section{Meeting research advisors}
The less time spent in understanding your work, the more time the advisor can spend in providing feedback, the better the meeting. See the oral presentations section in career strategies.

Frequency: Meet at least once a week, perhaps even more frequently during summer.

\subsection{Things to discuss}
Discuss thesis points and research.

Ask for things that you need.

\subsection{Avoid buildup of hidden strains}
Ask general questions. Ask specifics follow-ups to general responses:
\subitem Do you report progress regularly enough? Is the information adequate?
\subitem Are you devoting enough time for your research versus your other duties?
\subitem Are there ways you can improve how you work together? Does the advisor feel that you are responsive enough to her criticisms.
\subitem Are you maturing professionally?

\section{Dealing with the advisor}
Don't treat advisor as a stranger. Keep the channels of communication open.


\subsection{Giving the (potential) advisor proper metrics}
Finding the right things natural; aka raw instinct: ie, excellence in learning from examples.

Display attacks on problems, increased knowledge and understanding.

"We will talk about possible problems or possible ways to find a good problem for you. If you obtain results, then you get yourself an advisor." [Fan Chung Graham]

Confidence, critical taste.

\subsection{Report progress}
Report progress regularly and build a relationship within which problems can be solved, should they arise. Be honest.

Meet advisor one-on-one regularly - at least once a month. You can meet with secondary members less regularly - perhaps once in two months.

Solicit feedback every semester from advisor. "What are my strengths and weaknesses?"

Send monthly progress reports to your advisor and committee members.

\subsection{Research strategy for working with indrajita}
Focus on getting the research right.

Keep him happy, but don't necessarily follow his instructions to the letter. Often, it is very sub-optimal to follow his instructions in detail; over the long run it could result in ruination.

Take initiative and plan your own research.

Read 2 or 3 papers each week, and discuss these with him.

Find your own, fresh problems and work on them.

\subsection{Collaborate with eager researchers}
Collaborate extensively with pradIpa ravikumAra, perhaps end up with him as a co-advisor or even as primary advisor.

Retain TA job. Thus, you will retain flexibility in your choice of advisors.

\part{Grad School strategy}
\chapter{Strategies}
\section{Research}
\subsection{Researcher, not student}
Think of yourself, not as a student, but as a researcher.

Research assistantship with indrajita must be viewed as a temprarily desirable employment, where you gather credentials and qualifications to gain superior employment as soon as possible.

\subsection{Acquiring research capability}
Some time (1 year) is required to gain adequate strength and comfort in a research area. I am yet to gain that in any area, including data-mining, colt, graphical and statistical learning theory.


\subsection{Establish research credibility}
PhD is a way of establishing research credibility. Be sharply focused on the results.

\subitem Publish well, and in good quantity. Write a monograph or two. This, and acquiring recognition in the research community is the main thing, not the thesis, which is a mere formality.

\subsection{Advisor search}
See research advisor search strategy. See there for positives and negatives of various (potential) advisors.



\section{General strategy}
\subsection{Finish quickly}
\subitem Look for the thesis topic - now.
\subitem The fact is that nobody knows everything. Don't wait. Just get started.
\subitem Outsource work which can be outsourced. Example : Hire a statistics student or an undergrad.

\subsection{Teaching}
\subitem Do teach for a quarter or a semester at maximum, but avoid investing extra time into it. Do not be distracted! Research is more important than making extra money!

\subsection{Behave}
\subitem Have a blemish free reputation. The advisor should be unreservedly proud of you, and must think of you as a junior colleague.
\subitem Do not gossip about professors. Never say bad things within the university or research community.
\subitem Dress neatly, be thoroughly professional, modest and cheerful.
\subitem Relationship with the advisor:
\subsubitem Arrange your working hours to coincide with those of your advisor.
\subitem Remain friendly and cheerful.

\section{Risks in grad school}
\subitem Sour relationship with the advisor.

\subsubitem Possible reasons may include sexual or racial harassment, job insecurity, a sense of duty in sabotaging an unworthy candidate. These aren't the rule, but neither are they rare exceptions.
\subitem Poverty
\subitem Low status
\subitem Long hours
\subitem Total length of the ordeal
\subitem Social isolation
\subitem Difficult interactions with advisor and committee.


\section{TA strategy}

\subsection{Grading}
\subitem Have a list of things to check.

\subitem Grading programs:
\subsubitem Make scripts.

\subitem Grading papers:
\subsubitem Grade the entire bundle, enter scores later.
\subsubitem Do grading at night, at home for example, when creativity and productivity are low.

\chapter{Relationships to cultivate}
\section{Potential choices for research advice}
\subsection{In CS}
\subsubsection{Pradeep Ravikumar}
Works on statistical machine learning: inference, graphical model selection.

\paragraph*{Reasons}
Assistant professors still deeply involve themselves in the problem at hand. They provide pertinent suggestions when it comes to experimentation, help you write papers. Older professors tend to be much more hands-off.

He maintains good connections in the machine learning/ statistics community. 

Connections matter later in getting opportunities.

\paragraph*{Strategy}
Read a couple of his papers, especially High-dimensional Ising model selection using l1-regularized logistic regression and talk to him. Understand it with prateeka's help. Thereby you will have an advantage over others.

Approach him and express interest in working on a project with him which will lead to a publication.

\subsubsection{Kristen Grauman}
\subitem Prateeka has said that she is an excellent advisor.
\subitem Started off prateeka in research.

\subsubsection{Joydeep}
Works in data mining. Reputed to be a nice person.

\subsubsection{Bill Press}
Highly accomplished cross-disciplinary researcher. He is active, capable of doing mathamtics and programming. Has 3 students: CAM or BME. Was looking for biology students who would validate his theoretical predictions.

\subsubsection{Anna}
Very hard to get results. Interesting area.

\subsubsection{Matt Lease}
Work not very mathematical, very applicaiton oriented.

\subsubsection{Brent}
A hot area. Does applied, not theoretical cryptography.

Commanded a student to work on a certain problem, without getting any input. Results seem easier compared to complexity theory. But, cryptography, being useful for those interested in preserving secrecy: like corporations and governments, does not seem to be interesting yet.

\subsection{Potentially bad choices for research advice}
\subsubsection{Adam and Greg}
Have refused to advise me, based on their (mis?) judgement of my current level of theoretical ability. However, in our later interactions, I see in Adam implicit doubt and regret over this.

\paragraph*{Narrative for failure of collaboration with Adam} Combination of bad timing and impatience. Bad timing: Did not know basic things when I met him: such as Chebyshev's inequality, Chernoff bound, Hoeffding inequality, martingale, use of 'linearity of expecation' was not second nature; so he was skeptical. Impatience: I did not want to wait long enough for him to evaluate me; so I checked out other research areas.

\subsubsection{indrajita}
Works in applications of linear algebra and optimization in data mining. Seems interested in having me as his student.

His personality is explored elsewhere. Thence, I find that I do not admire his working style and personality (Asserts / displays dominance far too often), and that his research interests, especially the ones for which he has funding, are not the ones closest to mine. But, he is highly influential in the department; and pradIpa appears afraid to offend him. Also, pradIpa is considering very interesting projects with indrajita's lab. So, the idea is to be coadvised by indrajita.

indrajitasya dehabhAShAM cha durguNAn anusaran asmi iti bhayaH.

\paragraph*{Possible narrative for failure of attempt to get advice}
\subparagraph{indrajita-saMshodhanAlayasya tyAgaH}
\subparagraph{asmAt lAbhaH}
tasya kAryashailiH anuchitaH. kRRita-kAryeShu pragatyAH viShaye tApanAya asya asatya kathanAt cha bhayotpAdana-yatnAt muktiH prApyate. kaH Api kAryeShu kushalaH api nirbhayena vishrAntyA sthAtuM na shaknoti. punarpunaH anAvashyakaM api svAmitvaM pratipAdayati. I observed his interactions with me and other students; Our personalities and working styles are not compatible.

tasya dIkShA-virodhe kArya-karaNe (udAharaNArthaM saMshodhanAlayAt anyaiH saha kArya-karaNe) asvAtantryAt cha akaushalAt muktiH.

prAyaH anyebhyaH AchAryebhyaH uttarAH sUchanAH prApyante.

tat-paraM ahaM tasmai na roche, taM avalambituM na shaknomi iti jJNAtaM saMbhAShaNebhyaH; yat mayA samyak na prakaTitaM tat-dIkShA-avalambakasya cha kAryeShu tat-sImita-kA\~NkShasya vinayaM.

His research interests are not the ones closest to mine: Options seem to exist for exploring more abstract questions by more theoretical methods.

\subparagraph{asmAt naShTaH}
saha-saMshodhakaiH saha saMbhAShaNasya cha tebhyaH jJNAnArjanasya naShTaH.



\subsubsection{David}
Confirmed negative impression, due to tussle with Adam. Also, has a sufficient number of students. Hard to get results.

\subsubsection{Vijaya}
Recent students have had trouble adjusting to her working style. Not known if she is actively pursuing independent research.

\subsubsection{Gouda}
Intemperate, many theoretical but mostly impractical ideas.

\subsubsection{Ray Mooney}
Works in NLP, rather than in machine learning. Is known to be slightly averse to theory and mathematics, but focuses on heuristics and system building in problem-solving.

\subsection{In ECE}

\subsubsection{Sujay Sanghvi}
Works on optimization, matrix decompositions, graphical models.

\subsubsection{Constantine C}
Going by his webpage, it is likely that he will expect me to take his course first.

\subsubsection{Harris Vikalo}
works in estimation theory.

\subsubsection{Gustavo De Veciana}
Going by Jaechul, he will expect me to take his course first.

\subsection{In Math}

\subsubsection{Lexing}

He is known to be looking for students, and has funding. He is currently co advising one student working in quantitative finance. His area of research is in fast algorithms for problems from continuous mathematics. Research area very interesting, with wide ranging applications in Science. This could be an opportunity to learn a great deal of beautiful mathematics seen in his papers.

\subitem Impression about Lexing: Enthusiastic in teaching, conveys excitement and close memory of important algs, shows off during teaching: makes radical difference in student interest and perception.

\subsection{In BME}

Orley Alter: works on Genomic Signal processing. Well known in the biology community. She uses much Linear algebra: mainly applies well known, simple linear algebra techniques in the biology community where they are largely unknown. She has 2 students from ECE, is known to recently have refused one.

Imaging research center: Russell Poldrack. Manish Saggar, student of Riisto works with him.

\subsection{In TACC}
Chris of TACC and Robert proposed a numerical analysis project to interest Jesse. Works on sparse linear algebra, whereas Robert works on dense linear algebra.

\subsection{Business school}
carlos carvalho organized a workshop on bayesian non-parametrics, and asked insightful questions.

\subsection{Biology}
\subsubsection{edward marcotte}
Known to be a prodigous and excellent researcher.

Eager for results.

\chapter{The thesis}

\section{Thesis committee}
\begin{itemize}
\item Thesis committee will help guide you and help you find a job. It will consist of core members, who are convinced during the proposal, and secondary members, who come in later. Manage the thesis committee:

\subitem Advisor will be the chairperson, and indeed, the most important person. He has the power to resolve differences among committee members, though in some cases, this is impossible.

\subitem Populate the thesis committee with members who are genuinely motivated to improve your thesis, and are not out to score points against other committee members, or you. 
\subsubitem Their skills should complement each other, so that you can draw on them for help. Young professors have a reputation for being particularly harsh.
\subsubitem Don't pick luminaries if they are inflexible. Flexibility is especially important while choosing secondary committee members.
\subsubitem Ask other students and recent graduates about their experiences.
\subsubitem Make sure they aren't about to go on a sabbatical or retire.
\subitem Draw them into the thesis process as soon and as thoroughly as possible.
\subitem Develop the ability to do independent work. Don't bother them.
\subitem Solicit, and be responsive to their advice. Give in to minor points. Not many people are likely to read your thesis. Circulate your understanding of their advice, to be sure.

\subsubitem Going away and writing your complete thesis can lead to practical and political problems.
\subsubitem Get chapters reviewed as you go along. Circulate polished drafts. Delay in reading drafts is a common cause for delay in completion. Try gently to get a commitment on the deadline.
\subitem You can change the committee. Changes in committee are easier earlier in the process. If someone cannot attend, consider taking extra time to choose substitutes.
\subitem Try to get the thesis committee to meet annually to review progress, and to iron out differences. This way, approvals can be "on record".
\subitem Faculty members are not obliged to work on your committee. You should convince them that they can help you with little work on their part. It is best for the advisor to approach them.
\end{itemize}


\section{Problems between committee members}
Try to get the advisor to solve it amicably. Otherwise, call a meeting of the full committee. Final resort: alter your committee.

\section{Problems between you and a committee member}
\subitem Act quickly before the impression solidifies. Talk openly.
\subitem If problem is with a secondary committee member, seek advisor's help.
\subitem If even that fails, seek mediation from department chairperson or university ombudsman.
\subitem Final resort: alter your committee. Do it tactfully, let him save face, don't tell him or anyone about your reasons.
\subitem Repercussions minor if the problem is with a secondary committee member.
                        Relationship with an advisor is a serious matter. It is a risk to constitute the committee with him after that. It is like divorce.


\section{Thesis proposal}

The Thesis proposal is a contract with the committee, a sales pitch and a research plan. If there are changes, there can be renegotiation.:


Start writing a thesis proposal immediately upon finding possible thesis topics. If the topic turns out to be intractable, now is the time to change it.

Ask everyone you can for their opinions.

It can give you a head start in writing your thesis. Many parts of the proposal can make it intact to the thesis. It is desirable to make it specific, factual and detailed, specifically in the sections which will be later adapted into the thesis. This can also reduce far-reaching changes induced by the committee.

The proposal is but a proposal. So, don't spend too much time on it.

\section{Thesis writing}
\subitem Many people fall into a state of epic procrastination when it comes to writing the thesis.
\subsubitem Do not think that you have to read everything related to the topic before you start writing. The brain has bounded short-term memory capacity.
\subsubitem Do not find refuge in displacement activity. Keep an eye on the metrics.
\subsubitem Writing skills are improved and maintained only by regular practice.

\subitem Generally, including time for research, one can produce around 3 pages of thesis material per day. So, a 200 page thesis draft can take 3 months.

\subitem Material which can be directly used in the thesis:
\subsubitem The thesis proposal
\subsubitem Course papers - Choose topics of material importance to you.
\subsubitem Research papers: Write up results and analyses as experiments are completed.

\subitem Understand the treatment and structure in outstanding theses.

\subitem Thesis = Front Matter + Body + Back Matter.
\subsubitem Title should be full of key-words used by researchers while searching for references.
\subsubitem In acknowledgment section, mention all important people, including committee members. Keep it professional. Do not gush.

\subitem Submit chapters as they are written.
\subsubitem While making substantial revisions to the thesis, consult more than one thesis member - Their advice can often be contradictory.

\subitem Keep it as short as possible. Don't polish thesis endlessly. Even the search committee will not read a 600 page thesis - They look at published papers.

\section{References}
   1. Getting What You Came For: The Smart Student's Guide to Earning an M.A. or a Ph.D. (Paperback) by Robert Peters

\end{document}
