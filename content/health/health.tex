\documentclass[oneside, article]{memoir}
\input{../../../work/packages}
\input{../../../work/packagesMemoir}
\usepackage{fontspec, xunicode}
%\setmainfont[Script=Devanagari]{Chandas}
\setmainfont[Script=Devanagari]{Kalimati}

\input{../../../work/packagesMemoir}
\usepackage{fontspec, xunicode}
%\setmainfont[Script=Devanagari]{Chandas}
\setmainfont[Script=Devanagari]{Kalimati}

\input{../../../work/packagesMemoir}
\usepackage{fontspec, xunicode}
%\setmainfont[Script=Devanagari]{Chandas}
\setmainfont[Script=Devanagari]{Kalimati}

\input{../../../work/packagesMemoir}
\input{../../../work/macros}
\title{Health strategy}
\author{vishvAs}

\begin{document}
\maketitle

\chapter{Brain health}
\section{Genetics}
Some special genes can predispose you or increase resistance to diseases, including cognitive ones like alzheimers. As explained in the human society survey, genes play a limited role in general in the heritability of intellect.

\section{Blood supply to the brain}
Fatty build up in blood vessels is linked with brain impairment, and deadly strokes and heart attacks. The brain relies on a good blood supply to keep its functions and processes in top order.

\subsection{Hydration}
Hierarchical regression models demonstrated that lower hydration status was related to slowed psychomotor processing speed and poorer attention/memory performance, after controlling for demographic variables and blood pressure. 

\subsubsection{Negative agents}
Drinking heavy amounts of alcohol shrinks your brain.

\subsection{HDL levels}
One important determinant of blood vessel health is cholesterol levels, with the LDL subtype of cholesterol being bad for your arteries and HDL being good.

Many dietary items which increase HDL naturally include.

\subsubsection{Negative agents}
Smoking reduces HDL.

\paragraph*{Trans fats}
Trans fatty acids (commonly termed trans fats) are a type of unsaturated fat. Most trans fats consumed today, however, are industrially created as a side effect of partial hydrogenation of plant oils. Partial hydrogenation changes a fat's molecular structure (raising its melting point and reducing rancidity), but this process also results in a portion of the changed fat becoming trans fat. In trans fat molecules, the double bonds between carbon atoms (characteristic of all unsaturated fats) are in the trans rather than the cis configuration, resulting in a straighter, rather than a kinked shape. As a result, trans fats are less fluid and have a higher melting point than the corresponding cis fats. The primary health risk identified for trans fat consumption is an elevated risk of coronary heart disease (CHD).  [Ref] In other words, it blocks blood vessels, including those which plumb the brain.

\subsection{Capillary damage}
\subsubsection{Oxidative damage}
Since the brain uses about 25\% of the body's oxygen, the brain is highly susceptible to oxidative damage.

\subsubsection{Diabetes}
Diabetes damages the small blood vessels in the brain, and eventually rots these vessels to the point where they entirely close off. When this happens, the brain tissue fed by the blood vessel dies (i.e. a stroke). The diabetic brain therefore frequently looks like Swiss cheese, with lots of little holes scattered all over the place.

\subsubsection{High blood pressure}
Also damages the plumbing.

\section{Nutrition}
See food strategy.

\section{Activities: Stimulation and rest}
\subsection{Physical activity}
Physical activity may be beneficial to cognition during early and middle periods of the human lifespan and may continue to protect against age-related loss of cognitive function during older adulthood. The tasks, which measured subjects' reaction time and response accuracy when presented with congruent and incongruent visual patterns, involve cognitive processes known as executive control function (ECF).

Exercise causes the frontal lobes to increase in size. But other regions benefit from exercise in many secondary ways. "Wherever you have the birth of new brain cells, you have the birth of new capillaries."

\subsection{Cognitive exercise}
The Religious Orders Study, which began in 1993 and includes more than 1,000 nuns, priests and brothers across the country, has found that those who engage more often in reading, puzzles and processing information have a 47 percent lower risk of Alzheimer's disease than those who do little or none.

As Begley points out, many scientists now pooh-pooh the "use-it-or-lose-it" theory of mental functioning. Instead, they argue that it is "cognitive reserve" built up largely before the age of 30, not ongoing mental training, that benefits aging adults.

\subsection{Negative agents}
Games such as Mahjjong epilipsy.

\subsubsection{Prolonged stress}
The more hours you put in at work, the more likely you are to have high blood pressure. High Blood Pressure ravages the small blood vessels that feed the brain, and over time leads to many little holes in your gray and white matter that are quite obvious on MRI brain scans. [Ref]

In medical students studying for exams, the medial prefrontal cortex shrinks during cram sessions but grows back after a month off. [Ref]

\subsubsection{Sleep deprivation}
Some aspects of memory consolidation only happen with more than six hours of sleep. [Ref]

Teenagers who stay up late on school nights and make up for it by sleeping late on weekends are more likely to perform poorly in the classroom. This is because, on weekends, they are waking up at a time that is later than their internal body clock expects. [Ref] Sleep debt can be repaid, though it won't happen in one extended snooze marathon. Tacking on an extra hour or two of sleep a night is the way to catch up. [Ref]

By depriving rats of sleep for 72 hours, the researchers found that those animals consequently had increased amounts of the stress hormone corticosterone, and produced significantly fewer new brain cells in the hippocampus. [Ref]



\section{Physical damages}
About an American football player: He committed suicide November 2006 at the age of 44. The results of his brain autopsy have just been announced, and the pathologist from the University of Pittsburgh concluded that his brain cells had the appearance of an 85-year-old man with Alzheimer's disease.

They found brain damage in virtually every Everest climber but also in many climbers of lesser peaks who returned unaware that they had injured their brain.

On shockwave injuries: If the skull doesn't break, sometimes this can lead to the energy of the impact being more fully absorbed by the brain, often leading to shearing and tearing of the white matter pathways as the brain 'bounces around' inside.

\section{Aging effects}
In general, memory declines with age. But, for a few people, it remains clear and functional. [National Geographic, Nov 2007]

\subsection{High spread}
Claude E Shannon died of Alzheimer's disease. It's estimated that 5-10\% of the population aged 65 years or older has dementia.

\subsection{White matter decline}
White matter naturally degrades as we age—causing disrupted communication between brain regions and memory deficits—after conducting a battery of cognitive tests and brain scans on 93 healthy volunteers, ages 18 to 93. [Ref] The observed weakening of these brain network interactions (associated with default state and executive functions) was significantly correlated with a measured decline of cognitive functions as a result of aging. [Ref] 

\end{document}
