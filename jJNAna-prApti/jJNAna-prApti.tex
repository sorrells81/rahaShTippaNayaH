\documentclass[oneside, article]{memoir}
% \input{../../../work/packages}
\input{../../../work/packagesMemoir}
\usepackage{fontspec, xunicode}
%\setmainfont[Script=Devanagari]{Chandas}
\setmainfont[Script=Devanagari]{Kalimati}

\input{../../../work/packagesMemoir}
\usepackage{fontspec, xunicode}
%\setmainfont[Script=Devanagari]{Chandas}
\setmainfont[Script=Devanagari]{Kalimati}

\input{../../../work/packagesMemoir}
\usepackage{fontspec, xunicode}
%\setmainfont[Script=Devanagari]{Chandas}
\setmainfont[Script=Devanagari]{Kalimati}

\input{../../../work/packages}
\input{../../../work/packagesMemoir}
\usepackage{fontspec, xunicode}
%\setmainfont[Script=Devanagari]{Chandas}
\setmainfont[Script=Devanagari]{Kalimati}

\input{../../../work/packagesMemoir}
\usepackage{fontspec, xunicode}
%\setmainfont[Script=Devanagari]{Chandas}
\setmainfont[Script=Devanagari]{Kalimati}

\input{../../../work/packagesMemoir}
\usepackage{fontspec, xunicode}
%\setmainfont[Script=Devanagari]{Chandas}
\setmainfont[Script=Devanagari]{Kalimati}

\input{../../../work/packagesMemoir}
\input{../../../work/macros}

\title{Knowledge gathering and assimilation}
\author{vishvAs}

\begin{document}
\maketitle
\tableofcontents

\part{Introduction}
\chapter{Prelude}
jJNAna-prAptyai savitarka-buddhiH AvashyakI.

\section{Views}
\section{The goal: in case of research}
The ultimate objective is to be fruitful in the exploration of the worlds of scientific and mathematical ideas (aka research).

This pertains to tricks and tactics useful in
\begin{itemize}
 \item extracting and memorizing, understanding important information from books, articles, lectures;
 \item acquiring proof/ thought strategies and skills used therein.
\end{itemize}

\subsection{Signs of understanding}
Ability to reduce everything to elementary school mathematics signifies true understanding.

Ability to solve textbook problems is another important sign.

Ability to reproduce complex proofs without notes, thinking and remembering on the spot.

\part{Gathering knowledge}
\chapter{General ideas}
\section{Knowledge gathering for research}
\subsection{Picking knowledge to gather}
Deep understanding about a narrow area should be balanced with broad knowledge.

\subsection{Deepening knowledge and research skills}
Read about great discoveries, so that you can replicate great discoveries.

Then check your understanding and skills. See that section.

\subsubsection{Critiquing others' work}
Use the research article review checklist.

\subitem Critique the problem.
\subitem Critique the posited solution.
\subitem Discuss your critiques with others; thereby improve critiquing skills; gather their observations.
\subitem Critique the novelty of the problem and the solution.


\subsection{Acquisition of wide knowledge}
Know a million theorems. Keep playing around with all sorts of ideas, techniques and tools. Don't be scared of experts and their jargon.

Attend talks. There is a high correlation between those students who are doing the broadest and deepest work and those who are regularly attending seminars.

Talk to other graduate students. A lot. Organize reading groups. Be enthusiastic about reading and presenting papers. (Like PJ.) Also talk to post-docs, faculty, visitors, and people you run into on the street. They can give you ideas unrelated to what you were thinking. Don't worry too much about impressing them. Don't be scared to ask basic questions - and don't be surprised when nobody knows the answers.

\subsection{Good progress}
Without fail, read 3 papers every week.

\section{Checking understanding}
\subsection{Challenges and exercises}
A common Atma-va\~NchanaM is to fool oneself into thinking that one's skills and knowledge in a subject is sufficient for research. Hence, challenges of others, exercises and self-testing is extremely important. Experience shows these.

Thence you recognize holes in your knowledge and skills. The more holes you find in your knowledge and skills, the better it is.

\subsection{Questioning and inquiry}
NEVER be lazy about asking questions. Ask yourself as many questions (especially about the object of study) as possible.

\subsection{Challenges}
Other people ask questions to satisfy their own curiosity, and to desmonstrate a flaw in your knowledge and skills. Both of these are good.

Exams, homeworks, exercises are others.

Talk to people. Clearly state your ideas, so that they may criticise them.

Thence, challenges should be encouraged and embraced.

\chapter{Learning from technical interactions}
\section{Talking to others}
\subsection{Importance}
\subsubsection{In Learning}
Talk to others about subjects you wish to learn. In the act of expressing your understanding of the subject, you and others will find and correct mistakes. Furthermore, you get new knowledge.

\subsubsection{In research}
This is also an important source of high-quality technical collaborations: with a new perspective, the problem (yours or the one faced by another person) is tackled more easily.

\subsection{Procedure}
Don't worry too much about impressing them. You ask questions, make mistakes or you will be hiding your stupidity for the rest of your life.

Ask for questions and clarifications.

\subsection{On the internet}
Very useful technical exchanges can happen on the internet Eg: math stackexchange, stackoverflow.

Research interactions are considered in the Career Survey.

Socially curated news from researchers is considered in the 'reading' part, while online lectures are considered in the 'lecture' section.

\section{Learning from lectures}
\subsection{Importance and purpose}
\subsubsection{The learning process}
(Ravi Vakil) You'll hear various words, whose definitions you're not so sure about. At some point you'll be able to make a sentence using those words; you won't know what the words mean, but you'll know the sentence is correct. Later you'll learn the meanings and fill in the details more efficiently.

\subsubsection{Relevance to research}
It is amazing what can become relevant to your research. You won't believe it until it happens to you.

\subsection{Approximation of private tuition}
The purpose of attending a lecture is to learn from the lecturer's insights, opinions and ways of solving problems. Private tuition is far better than a lecture. But lecture exists due to lack of sufficient number of tutors. If you cannot follow what is going on, and if it is not too disruptive, ask. If you are lost, ask the lecturer to do an example.

\subsection{Difficulty}
An acquired skill: Learning to get information out of research seminars is an acquired skill, usually acquired much later than the skill of reading mathematics.

\subsection{Maintaining attention}
Try to ask one question at as many seminars as possible, either during the talk, or privately afterwards. The act of trying to formulating an interesting question focuses the mind.

Try pretending that you will be scored based on your understanding of the lecture.

Sit in the front of the class. During lectures, one must respond with gestures in order to indicate interest.

\subsubsection{Following the talk}
Try to follow the thread of the talk, and when you get thrown, try to get back on again. This isn't always possible, and admittedly often the fault lies with the speaker. The initial parts of the talk is very critical.

Check claims; consider limits, exercise intuition, use calculation.

Take text or slides to the lecture, if available.

\subsection{Post-talk discussions}
Befriend a couple of other attendants. Talk to them and to professors attending the talk, question them about the lecture, acquire their insights.

Try to answer the questions: What question(s) is the speaker trying to answer? Why should we care about them? What flavor of results has the speaker proved? Do I have a small example of the phenonenon under discussion? See if you can get one lesson from the talk. Criticize the problem and the solution; see how you can apply ideas thence to your work.

Try to extract three words from the talk and learn their definitions.  Assimilate notes immediately after the lecture: avoid backlogs.

\subsection{On the internet}
Video lectures and university course lectures are useful.

\chapter{Reading}
\section{Focus on ideas vs text}
\subsection{Focus on ideas}
In general, one must read books/ passages with the intention of mining it for problems, ideas and knowledge, and for training the brain in solution strategies, not to read it from cover to cover. So, ignore unimportant information.

Given a research paper, one must get to the interesting information as rapidly as possible.

\subsection{Focus on word-smithery}
Sentence level focus is sensible in rare cases: eg: relishing exquisite word-smithery.

\subsection{Focii in reading fiction}
One may focus on various ways of being (character, intellect, desires, behavior), uncommon circumstances (physical laws) and socio-economic-political conditions. One may compare these characters and conditions with those observed in actual life.

\section{Choosing what to read}
To find knowledge and interesting ideas, some works are more suitable than others.

\subsection{General knowledge}
News is described elsewhere.

Course-notes posted on the websites of university instructors and encyclopedias like the wikipedia are very useful in getting some good knowledge.

\subsection{Research papers}
Theses are often more detailed and friendly than journal papers. Journal papers are often better written than conference papers. Single author papers are usually more carefully written. So, reading them in great depth is a good way to enter a field.

\section{Parsing and understanding}
\subsection{History}
Comprehending passages reached a peak while preparing for the GRE general test. There reading comprehension questions tested ability to determine the general purpose of a given passage, find answer to specific questions using information presented, make (and critique) inferences (verbal reasoning).

\subsection{Seeking ideas}
\subsubsection{Grossness and separation}
To seek ideas efficiently and effectively, one needs to view the material at various levels of grossness. For this reason, it is important to maintain some distance from the text.

\subsubsection{Getting an overview}
Some portions of the text offer a more high level view than others. These ought to be read first.

Eg: The abstract and concluding paragraphs of an article, the first and last sentences of a paragraph. Sentences marked by structure indicators/ transition words like 'although' and 'but'.

Determine structure from titles first. Glance at figures and captions.

\subsubsection{Repeated summarization}
Summarize to yourself repeatedly to understand and remember what the passage is trying to say. Don't just verify correctness, ask "Why?". Guess the author's motives whilst reading. (Eg: Why did he choose that value for the interval?)

\subsubsection{Understand the author}
The author's viewpoint, which may be different from the ideas you identify, is often important and informative.

To understand this viewpoint, observe where the author spends much of his time, how the text builds up support for a certain bunch of ideas.

One flaw is to be fooled by red-herrings - ideas which are not central to the author's thesis.

\subsubsection{Iterative Progress}
Reading someone's deep work can be overwhelming. It is important to not let this happen. Find one small idea to conquer and digest at a time, and conquer it. Then repeat.

To ensure steady progress, keep track of your current position in the book/ article, so that you may continue from that point at a later date: One can recollect forgotten knowledge using one's notes.

\subsection{Considering ideas}
\subsubsection{Interestingness}
Pick one of the ideas the author is trying to communicate, see if you are interested in it. Repeat this until you find an interesting idea.

\subsubsection{Understandability}
Check if the interesting idea is something you can understand by attempting to understand it. If you cannot understand the idea, determine what prerequisites you must satisfy to be able to understand it. If you are so inclined, satisfy the prerequisites and repeat this step with that idea.

\subsubsection{Grok significance of ideas}
Read examples/ counterexamples or find your own. Simulate operations in toy models to understand general relationships.

Consider relationships with previous knowledge. Besides being very natural and enjoyable, such exercises and associations ensure that the brain does better at retaining important information.


\subsection{Dealing with dense passages}
Read with writing material. Complicated passages may require diagrams and external symbol manipulation to understand. Also, while reading mathematics, writing the important parts and leaving out the fluff helps immensely. Also, tracing mathematical formulae, which are dense in information, ensures that all essential information is noticed.

Considering the ideas presented themselves is described elsewhere.

\subsubsection{Graphical process}
In each sentence/ paragraph, scan for important entities (nouns which are objects, subjects, actors) and actions(verbs).

Understand the relationship amongst these entities by visualizing them as a graph. This can be a slow process.

\subsection{Signs of understanding, retention}
Writing short summaries and doing exercises for the purpose of retention are described elsewhere. This can be interleaved with the process of parsing.

\section{News}
\subsection{Content type}
This could be research news, news from friends through social media, newspapers etc..

They often tend to be short messages leading to other links. Sometimes, they are followed by discussions.

\subsubsection{Research news}
Besides internet journals and pre-print collections, twitter feeds and blogs of researchers tend to be a very good source of research news.

Lectures and course-notes are considered separately.

\subsection{Curation of articles and comments}
News can be chosen, curated, arranged and presented either by a small group of professional editors (Eg: BBC, Time, etc..); or by a democratic process where a huge group of readers vote articles up and down (Eg: Reddit, Digg).

Comments may also be curated by an editorial process (presented in the form of letters to the editor), or they may be curated and rated by a democratic system of votes (as in the case of Reddit, Ted talks, Youtube).

\subsection{Selectivity, filtering}
To get good return for the time spent following news, filtering and curation is essential. Good sources, from experience are: reddit, ted talks, podcasts such as Planet Money, BBC.

Sometimes this must be done manually. For example, in Facebook, classify people depending on the topics they post about, and zealously unfollow news feeds of people who consistently reduce the quality of your combined news feed.

\chapter{Retain important knowledge/ skills, discover problems}
General properties of memory and various techniques for strengthening memory is considered elsewhere.

\section{Retention of important knowledge}
\subsection{Reconstruction ability}
One must be able to reconstruct important knowledge acquired so far, relevant to the subject being studied, from scratch. This must be done with minimal external reference, and with the power of one's memory and abilities. This is helpful in preparing the mind for studies and in solidifying knowledge. For thoroughness of topics covered, a reference sheet may be used.

Being able to reproduce complex proofs without notes is an important skill. Eg: Pradeep presenting many proofs of statistical learning theory, VCD, packing number etc.. PratIka presenting ICML papers.

\subsection{Notes/ surveys}
\subsubsection{Identify core ideas}
A complex concept is just an arrangement of a few simple ideas. This is important while making notes.

Learning involves repeated summarization.

\subsubsection{Succinct summarization}
Record important patterns of ideas and solution strategies, with examples/ counter-examples and important results. View surveys as embedding the graph of ideas in two dimensions. The idea is to embed a network of ideas on a sheet.

Capture the most central ideas; ideas which evoke the strongest and the most unusual reactions. Ignore the rest. Be able to summarize the essentials of a paper in a few paragraphs/ sentences. This is useful as a mnemonic, as a useful device while re-reading the details, and as a block to be used in reasoning.

\subsubsection{Vividness}
In the surveys, note important and surpising things, which evoke a reaction other than the usual; make things vivid. Consider making them colorful. \exclaim{Exclaim in notes!}

Ignore and skip unimportant information.

For closeness to the subject, it is important to have favorite examples.

\section{Exercises}
Often work out small details while reading.

ALWAYS attempt for a reasonable time to work out the solution strategy yourself before looking up the proof or the solution to the problem. This is very important training in problem-solving in that area. It is also important in the development of intuition.

\subsection{Tackling, not solving}
The point is to tackle, not necessarily solve. This is very important training in problem-solving in that area.

\subsection{Review attempt}
Even if you read the solution, consider what you *could* have done to solve the problem.

\section{Mining for fresh problems}
It is very important to critique a paper or a talk, understand its shortcomings, suggest future improvements. Thence, incremental research occurs.

OFTEN, perturb the problem and extrapolate a solution. This is a very important skill.

\part{References}
    * Posts and comments by Gowers and others.

    * Math book suggestions by J Baez.

    * The PhD experience by Mihir Bellare

\end{document}
